\section{Conclusions and future work}

\subsection{Knotation system properties}\label{sub:knot_system_props_recap} % (fold)
Rather than list the properties of section
\ref{sub:desirable_properties_for_a_knot_notation_system_} and tick
them off, as we may do, let us take a step back. In the end it is the
act of calculating over knots -- tabulating, searching, making new
ones from old -- that places demands on the notation system used to
represent them. These are the practical considerations that give rise
to the properties listed in
\ref{sub:desirable_properties_for_a_knot_notation_system_}. Establishing
the tight correspondence between processes in the image of the
encoding and the knots they represent is ultimately in service of
showing this is a reasonable proxy for knots because calculations on
the representation have a correspondence to calculations on the
domain. Moreover, the last few decades of logic in computer science
has taught us that there is fidelity and then there is
\emph{fidelity}. It's one thing map elements of one domain to elements
of another, but it's another to map the procedures of one domain
(roughly, operations on elements) to the procedures of
another. Whether one sees a demand like this through the lens of
``functoriality'' or through a correspondence of proofs (of say,
ambient isotopy, in a proof system made of Reidmeister moves) and
programs (that transform programs), the practical side of the demand
is about making calculation \emph{tractable}, so that ``local''
operations in one domain correspond to ``local'' operations in the
other. This makes it possible to structure calculations in terms of
the structure of the elements over which the calculations are being
performed, which has far-ranging consequences on the complexity of
calculation in the representative domain. Of note, we have been
careful to restrict attention to the image of the encoding. We do not,
as of yet, have an effective characterization of processes that will
decide whether or not a process is in the image of the encoding. This
restriction notwithstanding, the image is the only system thus far
proposed that does enjoy all of the properties listed, including possessing (actually, being) a
language for classifying knots via logical properties.

\subsection{Knot queries}\label{sub:knot_queries} % (fold)
The interest in knot tabulation coming from the physical sciences is a
strong motivation to investigate the design of a specialized form of
spatial logic tailored to reasoning about processes in the image of
this encoding. Specifically, the current authors are in the process of
devising an executable knot query language that runs on top of a
translation of spatial logic predicates to XQuery, the XML query
language \cite{W3C-XQ-2873221} \cite{W3C-XQ-5631451}. In this
application, a member, $K$, of an isotopy class $[K]_{\sim}$ is stored
as an XML document $\meaningof{\meaningof{K}}_{XML}$. As the notation
indicates, the document may be calculated from the process expression,
$\meaningof{K}$. Pleasantly, all such documents will conform to an XML
schema \cite{Fallside:04:XSP} \cite{Mendelsohn:04:XSP}
\cite{Biron:04:XSP} that may be formally derived from the grammar for
processes described in section \ref{subsub:process_grammar}. Thus, the
mathematics extends to a formal specification of the software
implementation of a ``knot database'' and the compositional nature of
the specification makes tractable formal verification of the
correctness of the software with regards to the specification.
% subsection knot_queries (end)

\subsection{Braids, tangles, virtual knots, racks and quandles}\label{sub:braids_tangles_and_virtual_knots} % (fold)

As may be seen from the encoding of the Reidemeister moves, nothing in
this approach restricts it to knots. In particular, the same
techniques may be lifted to braids and tangles. More generally,
Kauffman posits an intriguing new member to the knot family by
virtualizing the crossings in a knot diagram
\cite{kauffman-2005-VKNL}. We note that while we arrived at this
encoding before becoming aware of Kauffman's work, that line of
investigation is very much aligned to the guiding intuitions of the
encoding presented here. Namely, the crossing circuit \emph{a priori}
could be any $\pi$-calculus process that respects the interface of the
wiring circuit. We conjecture that many forms of \emph{virtualization}
may be faithfully interpreted as \emph{simulation}.

Moreover, it seems likely that the rack and quandle frameworks may be
mirrored in the process calculus, as these structures can be used to
model the \emph{action} of the Reidemeister moves on a knot. The
relation between this type of action and the dynamics of the process
calculus presented here remains to be investigated.

% subsection braids_tangles_and_virtual_knots (end) 

\subsection{Other calculi, other bisimulations and geometry as behavior}\label{sub:other_calculi_other_bisimulations_and_geometry_as_behavior} % (fold)


The astute reader may have noticed that the encoding
described here is not much more than a linear notation for a minimal
graph-based representation of a knot diagram. In this sense, there appears to be
nothing particularly remarkable about the representation. Though
expressed in a seemingly idiosyncratic way, the encoding is built from the same
information that any freely downloadable program for calculating knot
polynomials uses routinely. What is remarkable about this
representational framework is that it enjoys an \emph{independent}
interpretation as the description of the behavior of concurrently
executing processes. Moreover, the notion of the equivalence of the
behavior of two processes (in the image of the encoding) coincides
exactly with the notion of knot equivalence. It is the precise
alignment of independently discovered notions that often indicates a
phenomena worth investigating.

This line of thought seems particularly strengthened when we recall
the $\pi$-calculus is just one of many `computational calculi' ---the
$\lambda$-calculus being another paradigmatic example--- that may be
thought of as a \emph{computational dynamics+algebra} and that
virtually every such calculus is susceptible to a wide range of
bisimulation and bisimulation up-to techniques
\cite{milner92techniques}. As such, we see the invariant discussed
here as one of many potential such invariants drawn from these
relatively new algebraic structures. It is in this sense that we see
it as a new kind of invariant and is the inspiration for the other
half of the title of this paper.

Finally, if the reader will permit a brief moment of philosophical
reflection, we  conclude by observing that such a connection fits
into a wider historical context. There is a long-standing enquiry and
debate into the nature of physical space. Using the now familiar
signposts, Newton's physics ---which sees space as an absolute
framework--- and Einstein's ---which sees it as arising from and
shaping interaction--- we see this connection as fitting squarely
within the Einsteinian \emph{weltanschauung}. On the other hand,  unlike the particular mathematical framework of continuity in
which Einstein worked out his programme, behavior and its implied notions of space and time are
entirely discrete in this setting, built out of names and acts of
communication. In this light we look forward to revisiting the now
well-established connection between the various knot invariants such
as the Kauffman bracket and quantum groups. Specifically, it appears
that the kinematic picture of loop quantum gravity derived from spin
networks can be faithfully encoded in a manner analogous to one used
here to encode knots, but the process structure offers an account of
dynamics somewhat different from spin foams \cite{baez-2000-543}.

% subsection other_calculi_other_bisimulations_and_geometry_as_behavior (end)


