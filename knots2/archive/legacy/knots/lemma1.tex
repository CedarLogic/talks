We first establish a lemma to our purposes, that shows we can clean up our knot diagram in a particularly suitable way.

Given a knot $K$ and a diagram $D(K)$ of $K$, we  call the application of a 
Reidemeister move $\rho$ a {\em neatening} move if the resulting diagram $D'(K)$ has a crossing number less than or
equal to the crossing number of $K$ i.e. any neatening move
of the type $\Omega_{1}$ or $\Omega_{2}$ can occur in
only one direction. Let $\hat{\rho}=\rho_{1}\cdots\rho_{n}$ be a sequence of successive Reidemeister moves on the
diagram $D(K)$ (with $n=0$ being the identity move). We call $\hat{\rho}$ a {\em neatening isotopy} if either
$n=0$ or  each
$\rho_{i}$
is a neatening move. If, in addition, at least one of the $\rho_{i}$ is a move of  type $\Omega_{1}$ or $\Omega_{2}$, we
call $\hat{\rho}$ a {\em cleaning isotopy} of $D(K)$. If $\rho_{n}$ is  move of  type $\Omega_{1}$ or $\Omega_{2}$, we
call $\hat{\rho}$ a {\em concise}  cleaning isotopy of $D(K)$

For a given diagram $D(K)$, the collection of all its neatening isotopies can be given a partial ordering: given
neatening isotopies $\hat{\rho}=\rho_{1}\cdots\rho_{n}$ and $\hat{\sigma}=\sigma_{1}\cdots\sigma_{m}$, we put
$\hat{\rho}\le\hat{\sigma}$ when $n\le m$ and $\rho_{i}=\sigma_{i}$  for all $i\le n$. This partial order restricts to the
the subcollection  of concise cleaning isotopies.

\begin{lemma}\label{simplificationLemma} For any knot  $K$
and knot diagram $D(K)$, we may reduce
$D(K)$ via a neatening isotopy to a diagram
$D'(K)$  having the property that no neatening isotopy of $D'(K)$ is a cleaning isotopy of $D'(K)$ .
\end{lemma}
\begin{proof} If the collection of cleaning isotopies of $D(K)$ is an empty one, then $D(K)$ is the desired diagram.
So assume there is at least one cleaning isotopy of $D(K)$.

The collection of concise cleaning isotopies of
$D(K)$
forms a partially ordered set. Since each
concise cleaning
isotopy reduces the
number of crossings in the diagram by at least 1, each chain in the partially order set has length bounded by the
number of crossings in $D(K)$. Select any maximal element $\hat{\rho}$ of the partially ordered set. Let $D'(K)$ be
the diagram derived from $D(K)$ by applying $\hat{\rho}$. 
\end{proof}

