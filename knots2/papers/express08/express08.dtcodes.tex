\section{Appendix: From DT-codes to processes}

What follows is an algorithm for calculating the $\omega$ indexing
function. Note that the function \emph{over} can be implemented easily using the function $\delta$, or the function $sgn$, described in section \ref{ssub:dowker_thistlethwaite_codes}. Exercise for the reader: upgrade the algorithm so that wires
are not duplicated during the process.  \newpage

%  \begin{lstlisting}
\begin{verbatim}
(* ------------------------------------------------------------------ *)
(*                                                                    *)
(* Given types Knot and Wire, the omega function is typed as follows  *)
(* omega : int -> (int -> int) -> (int -> int) -> Knot -> (Wire list) *)
(* With dt a table representing the Dowker-Thistlethwaite mapping     *)
(* from odds to evens, and dti the inverse. The function keeps an     *)
(* accumulator, acc, in which to collect the wires it calculates.     *)
(* The function assumes helper functions C and x0,x1,y0,y1. C takes   *)
(* and instance of type Knot and the odd index of the DT code and     *)
(* returns the crossing. The other helpers are accessors of the ports *)
(* of the crossing. The algorithm visits every crossing and so        *)
(* generates many wires more than once, which is why the accumulator  *)
(* is updated with union rather than cons on the recursive calls. As  *)
(* an exercise, modify the algorithm to avoid wire duplication.       *)
(*                                                                    *)
(* ------------------------------------------------------------------ *)

let omega i dt dti knot acc = 
  if (i <= (numCrossings knot)) 
     then 
       let ic = (2*i - 1) in 
         (omega 
            (i+1) dt dti knot 
            (union acc 
               [ W(x1(C(knot,ic)), 
                 (if (over dt ic-1)
                     then y0(C(knot,ic-1))
                     else y1(C(knot,ic-1)))); 
                 W(y0(C(knot,ic)), 
                 (if (over dt ic+1)
                     then x1(C(knot,ic+1))
                     else x0(C(knot,ic+1)))); 
                 W(x0(C(knot,ic)), 
                 (if (over dt (dti ((dt ic)-1)))
                     then y0(C(knot,(dti ((dt ic)-1))))
                     else y1(C(knot,(dti ((dt ic)-1)))))); 
                 W(y1(C(knot,ic)), 
                 (if (over dt (dti ((dt ic)+1)))
                     then x1(C(knot,(dti ((dt ic)+1))))
                     else x0(C(knot,(dti ((dt ic)+1)))))) ])) 
     else acc
\end{verbatim}
%     \end{lstlisting}

%%% Local Variables:
%%% Tex Master:"jktir.tex"
%%% End: