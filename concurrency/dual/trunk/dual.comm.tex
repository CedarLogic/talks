\section{Evolution and communication}

So far we have merely re-presented Milner's $\pi$-calculus, albeit in
the light of questions it raises about what shape a proper theory of
dynamics might have. In this section we begin to dissect the theory to
expose its implicit commitments to notions of scale and interaction.

As noted above, the original comm rule for Milner's $\pi$-calculus was
of the form

\begin{mathpar}
  \inferrule* [lab=comm] {} {x?(y).P | x!(z).Q \to P\{z/y\} |Q}
\end{mathpar}

Later, when Milner analyzed the scoping of input binding he realized
that the sequencing operator was given a more hygenic account as
synchronization followed by data-exchange. Compare with the
productions of the grammar: sequence is effectively a function taking
names and agents, while agents is the disjoint sum of two types:
abstractions and concretions. In the case of the $\pi$-calculus we
have abstractions of the form $A \equiv ?(y)P$ (i.e. functions from
names to processes) and concretions of the form $C \equiv (\nu u)!(z)Q$
(i.e. pairs of names and processes).

This refactoring leads to a considerably more parsimonious comm rule.

\begin{mathpar}
  \inferrule* [lab=$comm_{AC}$] {} {x.A | x.C \to A \cdot C}
\end{mathpar}

where $A \cdot C$, called pseudo-application, is then defined in this context as

\begin{equation}
  ?(y)P \cdot (\nu u)!(z)Q \triangleq (\nu u)P\{z/y\}|Q \nonumber
\end{equation}

\subsubsection{Probabilistic pay-offs}
This is much more than just a convenient refactoring. It represents a
fairly deep set of ontological categories that have natural
polymorphic interpretations, and natural extensions. To illustrate
this point consider Priami or Cardelli and Phillips' accounts of the
stochastic $\pi$-calculus \cite{DBLP:journals/cj/Priami95}
\cite{Phillips-Cardelli-Bioconcur04}. Both presentations make two
essential changes to the calculus. One is a syntactic change to the
shape of process constructors and the other is a change to the
reduction relation.

The first modification is based on a typing of sequence as $action
\times process \to process$. One way of seeing their account is that
they interpret this as a polymorphic operation in which actions may be
extended to have additional structure, $(x?(y),r) \; | \; (x!(z),r)$,
namely a rate from which a probability of interaction may be
calculated given a sampling window and global description of
contending candidates for reduction. To carry this interpretation out,
they modify the reduction relation to take an additional parameter,
namely the rate. (Priami's original account differs from the
Cardelli-Phillips account in how the probability is related to the
rate in certain contention cases -- a difference that has little
import in the discussion here.)

In an interpretation of sequence as typed $name \times (abstraction
\oplus concretion) \to process$ we see that there are two places to
consider as candidates for probabilistic or stochastic treatment:
synchronization and data-exchange. Both of these have natural physical
interpretations:

\begin{itemize} 
\item synchronization, like the docking of a protein on a gene site,
  is probabilistic, or fuzzy;
\item data-exchange, like the sharing of an electron in
  electron-bonded compounds, is probabilistic.
\end{itemize}

We can recover the Priami/Cardelli/Phillips interpretations by the
\emph{single} syntactic extension, $(x,r).A$. Like the systems of
Priami or Cardelli and Phillips we may interpret this as a polymorphic
interpretation of \emph{location} as having further articulated
structure, namely the rate, $r$. Additionally, we may explore
presentation of probabilistic data-exchange. It is outside of the
scope of this paper to present them, but there are very natural
interpretations that admit quantum computational interpretations.

The central point of this exegesis is that Milner's decomposition of
the sequence operator in \cite{milner91polyadicpi} -- which we stress
as something akin in scope to Girard's decomposition of implication
\cite{DBLP:journals/tcs/Abramsky93} -- has more pay-off than just a
hygenic account of scoping of binding operations. It also provides a
simpler account of existing probabilistic interpretations and also
points the way to additional and potentially useful probabilistic
theories. Thus, despite the lack of treatment of concrete examples of
interaction across scale, we are still accruing evidence in favor of a
certain declension of the theory; exposing a certain warp and woof to
it that favors certain usage patterns.

\subsubsection{Synchronization algebras}

Now to address the central concern of relating scale to interaction we
need to make explicit another piece of information implicit in the
standard presentation of the $\pi$-calculus. Synchronization
fundamentally depends on name equality. In the comm rule, the
sequenced abstraction may be thought of, roughly, as located at the
`?'-end of a channel, $x$, and the sequenced concretion at the
`!'-end. It is name equality, i.e. that $x = x$, that determines that
these agents are located at the opposite ends of the \emph{same}
channel and thus may communicate. We generalize this notion of
co-location by adding a predicate, $\perp : name \times name \to
bool$, explicitly determining when two names enable synchronization
and data-exchange. Our comm rule then becomes

\begin{mathpar}
  \inferrule* [lab=$comm_{AC^{\perp}}$] {x_{src} \perp x_{trgt}} {x_{src}.A | x_{trgt}.C \to A \cdot C}
\end{mathpar}

In this context we note that Milner saw this as an avenue of
investigation. In SCCS, for example, \cite{Milner83SCCS} he considered
actions with the additional structure of a group and a communication
rule that depended on action annihilation (i.e. that the channel and
co-channel were group inverses of each other). Additionally, other
investigators have exploited this possibility. Meredith and Radestock,
for example, use it to derive a reflective, two-level comm rule in
which co-location is calculated using the process dynamics itself
\cite{meredith2005rho}.

Further, in light of the previous discussion of probabilistic
accounts, we mention that it is natural to consider a co-location
predicate that is not boolean-valued, but takes a value in the unit
interval, representing (a rate that may be used in a calculation of) a
probability or likelihood of interaction. That this ultimately yields
an account of a Priami/Cardelli/Phillips-style interpretation of a
probabilistic $\pi$-calculus -- without intrusive modification of
process syntax -- is a mark in favor of following Milner's prescient
intuitions regarding the declension of the comm rule. Apart from
motivating the idea of having the co-location predicate take values in
another domain, however, we do not pursue this line of investigation,
here.

Moving steadily towards our goal of accounting for interaction across
scale, recall that Milner's decomposition of sequence provides
\emph{two} places for expanding the interpretation and application of
his calculus. We've already noted that it is fruitful to explore
polymorphic interpretations of data-exchange, and it is a polymorphic
interpretation of the pseudo-application operator, together with the
way it interacts with an expanded notion of co-location that
constitutes the central innovation of the construction we wish to
consider.

% To develop this interpretation, we consider pseudo-application in the
% context of the reflective calculus described in the previous
% section. Note that in this context the continuation of output may be
% associated with wherever the codes of the process communicated are
% dropped.

Before considering notions of spatio-temporal scale, let us consider a
common aspect of scale as system complexity in the context of network
protocols -- a domain where Milner's calculus is already established
as broadly applicable. Intuitively, suppose we had two processes that
were written to communicate over protocols that were at different
levels in a network stack, e.g. in the network setting, one might be
written to accept and emit \texttt{TCP/IP} packets while the other was
written to accept and emit \texttt{HTTP} over \texttt{TCP/IP}. It is
common practice to think of the process that communicates at the
higher layer as essentially `wrapped' by a set of communications
behaviors that have been factored from the core behavior of the
process. If $P$ is the process written at the higher layer, we can
faithfully represent that division of labor by $K[P]$, where $K$ is a
context that constitutes the communications behaviors of packaging up
\texttt{HTTP} requests and responses over \texttt{TCP/IP}.

% In the higher-order context, we might effect the communication between
% the two processes as $\lift{x}{P} | x(y).(K[\dropn{y}]|Q)$. In the
% context of pseudo-application, however, it is natural to consider a
% $K$-indexed version of the operator defined as 

% \begin{equation}
%   (y)Q \cdot_{K} \lift{}{P} \triangleq Q\{\quotep{K[P]}/y\} \nonumber
% \end{equation}

In terms of our intuitive interpretation, such $K$ represents a
protocol mediator, mediating between the layer at which $P$ engages
and the layer at which $Q$ engages. The question then becomes how to
select these mediators? This is precisely where the expansion of the
co-location predicate from boolean-valued to taking values in some
other domain comes into play. We use the pair of names (in terms of
our networking analogy, the pair of communications port abstractions
at the different layers in the networking stack) to determine the
mediating context. So, the co-location ``predicate'' becomes a
context-valued function. Thus, the general expression of our comm rule
becomes

\begin{mathpar}
  \inferrule* [lab=$comm_{AC^{\perp}K}$] {x_{src} \perp x_{trgt} = K} {x_{src}.A | x_{trgt}.C \to A \cdot_{K} C}
\end{mathpar}

Obviously, the relation $x \perp_{\Box} x \triangleq \Box$ with $x
\perp_{\Box} y$ undefined if $x \neq y$, recovers the original comm
rule. In situations other than the trivial case, this rule must be is
instantiated by an appropriate notion of pseudo-application. We
consider three variations.

\paragraph{Single-holed context-indexed pseudo-application}

In the most straightforward interpretation, we may provide a single
mediating context and make a decision about where to break symmetry
between input and output.

\begin{equation}
  (y)P \cdot_{K} [z]Q \triangleq K[P]\{z/y\}|Q \nonumber
\end{equation}

This approach works well for processes in which there is a clear
difference in ``layering'', as in our network example. Sometimes,
however, processes are not cleanly layered involving communication
events at different levels of granularity. Network monitoring
applications, for example, may often these uneven communications
events; leading to our second variation.

\paragraph{Dual-holed context-indexed pseudo-application}

In a slightly more sophisticated approach, we borrow again from Milner
and consider multi-holed contexts, though we only need dual-holed
contexts to restore symmetry. In this approach, the co-location
predicate must take values in contexts of the form, $K \equiv
K_{A}|K_{C}$, which we demand in the hypothesis to our comm rule.

\begin{mathpar}
  \inferrule* [lab=$comm_{AC^{\perp}KPar}$] {x_{src} \perp x_{trgt} = K, K \equiv
    K_{A}|K_{C}} {x_{src}.A | x_{trgt}.C \to A \cdot_{K} C}
\end{mathpar}

The pseudo-application takes advantage of the additional information
about the structure context to substitute the continuations of the
communications into their appropriate holes.

\begin{equation}
  (y)P \cdot_{K \equiv K_{A}|K_{C}} [z]Q \triangleq K_{A}[P]\{z/y\}|K_{C}[Q] \nonumber
\end{equation}

\paragraph{Higher-order context-indexed pseudo-application}

Further, in an age where code is regularly downloaded and executed
across a variety of execution platforms, we may recognize that not
every execution site is to be treated equally. In this setting it is
natural to consider a higher-order variant of the $\pi$-calculus, in
which the (codes of) processes are passed. Thus, we have abstractions
of the shape, $process_{0} \times \ldots \times process_{n} \to
process$, and concretions are simply processes.

\begin{equation}
  (y)P \cdot_{K} Q \triangleq P\{K[Q]/y\} \nonumber
\end{equation}

Fully instantiating our comm rule, we have 

\begin{mathpar}
  \inferrule* [lab=$comm_{AC^{\perp}KHO}$] {x_{src} \perp x_{trgt} = K} {x_{src}.(y)P | x_{trgt}.Q \to P\{K[Q]/y\}}
\end{mathpar}

\subsubsection{Composing algebras}

The scheme of the comm rule derived above supports a natural notion of
composition, which in turn, gives surprisingly natural notions of
morphisms, neatly side-stepping the issues raised above regarding a
categorical treatement of the composition of algebras. Suppose we have
two reduction relations, $\red_{1}$ and $\red_{2}$ given in terms of
our comm rule scheme.

\begin{mathpar}
  \inferrule* [lab=$comm_{AC^{\perp}K1}$] {x_{src} \perp_1 x_{trgt} = K} {x_{src}.A | x_{trgt}.C \to_1 A \cdot_{K}^{1} C}
  \and
  \inferrule* [lab=$comm_{AC^{\perp}K2}$] {x_{src} \perp_2 x_{trgt} = K} {x_{src}.A | x_{trgt}.C \to_2 A \cdot_{K}^{2} C}
\end{mathpar}

We may derive a composite comm rule

\begin{mathpar}
  \inferrule* [lab=$comm_{AC^{\perp}K1 \circ K2}$] {x_{src} \perp_1 x_{mid} = K_1, x_{mid} \perp_2 x_{trgt} = K_2} {x_{src}.A | x_{trgt}.C \to_1 A \cdot_{K_1 \circ K_2} C}
\end{mathpar}

where, as usual, we take $(K_1 \circ K_2)[P] \triangleq
K_2[K_1[P]]$. More generally, we may lift all of the context
constructors to constructors of algebras.

\paragraph{Typed composition of algebras}
TBD

% Consider, the modern presentation of a process algebra is essentially
% given by a triple, $\langle G, \equiv, \red \rangle$, comprised of the
% generating structure, $G$, the structural equivalence, $\equiv$ and
% the reduction relation, $\red$. Notions of morphisms from one such
% triple to another are complicated by the interplay between structure
% in $G/\equiv$ and dynamics in $\red$. We can dramatically simplify the
% situation if we require the term structure and structural equivalence
% to be perfectly preserved and only consider the composition of
% $\red$'s.

\subsection{From complexity scale to spatio-temporal scale}

TBD

% Up to this point we have primarily relied on well-established
% applications of Milner's calculus to bootstrap intuitions about the
% conceptual framework under discussion. Moreover, it should present no
% conceptual challenges to think about scale qua system complexity as
% essentially aligned with what may be captured via context. In fact,
% even contexts based on recursive definitions are compatible with
% identifying \emph{separation} in complexity scale with contextual
% decomposition. That is, it is intuitively reasonable to demand $P
% \equiv K[Q] \Rightarrow S(Q) \leq S(P)$ for any reasonable measure of
% complexity scale, $S$.

% The step from complexity to spatio-temporal scale is actually already
% present in the process calculi literature in the form of recent work
% on spatial and context logics. To cut to the chase, contexts are
% essentially processes with holes in them. Spatial/contextual formulae
% may be used to locate those holes in space and time. 

% Our context-valued co-location predicate


