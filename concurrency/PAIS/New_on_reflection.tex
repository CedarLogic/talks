% ------------------------------------------------------------------------
% AMS-LaTeX Paper ********************************************************
% ------------------------------------------------------------------------
% Submitted:      Dec 15 2003
% Final Version:  
% Accepted:       
% ------------------------------------------------------------------------
% This is a journal top-matter template file for use with AMS-LaTeX.
%%%%%%%%%%%%%%%%%%%%%%%%%%%%%%%%%%%%%%%%%%%%%%%%%%%%%%%%%%%%%%%%%%%%%%%%%%

%\documentclass{tran-l}
\documentclass{amsart}

%\usepackage[active]{srcltx} % SRC Specials for DVI Searching
\usepackage{url}

% From Allen's stable.
\usepackage{bigpage}
\usepackage{bcprules}
\usepackage{code}
\usepackage{amsfonts}
\usepackage{latexsym}
\usepackage{amssymb}
\usepackage{caption}

% SYNTAX
\newcommand{\id}[1]{\texttt{#1}}
\newcommand{\none}{\emptyset}
\newcommand{\eps}{\epsilon}
\newcommand{\set}[1]{\{#1\}}
\newcommand{\rep}[2]{\id{\{$#1$,$#2$\}}}
\newcommand{\elt}[2]{\id{$#1$[$#2$]}}
\newcommand{\infinity}{$\infty$}
\newcommand{\seq}{\mathbin{\id{,}}}
\newcommand{\all}{\mathbin{\id{\&}}}
\newcommand{\choice}{\mathbin{\id{|}}}
\newcommand{\altern}{\mathbin{\id{+}}}
\newcommand{\juxtap}{\mathbin{\id{|}}}
\newcommand{\concat}{\mathbin{\id{.}}}
\newcommand{\punify}{\mathbin{\id{:=:}}}
\newcommand{\fuse}{\mathbin{\id{=}}}
\newcommand{\entailm}{\mathbin{\vdash_{\mathfrak m}}} %matching
\newcommand{\entailp}{\mathbin{\vdash_{\mathfrak p}}} %behavioral
\newcommand{\entailv}{\mathbin{\vdash_{\mathfrak v}}} %validation
\newcommand{\congd}{\mathbin{\equiv_{\mathfrak d}}}
\newcommand{\congs}{\mathbin{\equiv_{\mathfrak s}}}
\newcommand{\congp}{\mathbin{\equiv_{\mathfrak p}}}
\newcommand{\scong}{\mathbin{\equiv}}
\newcommand{\pzero}{\mathbin{0}}
\newcommand{\typedby}{\mathbin{\:\colon}}
\newcommand{\mixedgroup}[1]{\id{mixed($#1$)}}
\newcommand{\cast}[2]{\id{CAST AS} \; #1 \; (#2)}
\newcommand{\bslsh}{\mathbin{\id{\\}}}
\newcommand{\bslshslsh}{\mathbin{\id{\\\\}}}
\newcommand{\fslsh}{\mathbin{\id{/}}}
\newcommand{\fslshslsh}{\mathbin{\id{//}}}
\newcommand{\bb}[1]{\mbox{#1}}
\newcommand{\bc}{\mathbin{\mathbf{::=}}}
\newcommand{\bm}{\mathbin{\mathbf\mid}}
\newcommand{\be}{\mathbin{=}}
\newcommand{\bd}{\mathbin{\buildrel {\rm \scriptscriptstyle def} \over \be}}
\newcommand{\category}[1]{\mbox{\bf #1}}
% Double brackets
\newcommand{\ldb}{[\![}
\newcommand{\rdb}{]\!]}

%GRAMMAR
\newlength{\ltext}
\newlength{\lmath}
\newlength{\cmath}
\newlength{\rmath}
\newlength{\rtext}

\settowidth{\ltext}{complex type name}
\settowidth{\lmath}{$xxx$}
\settowidth{\cmath}{$::=$}
\settowidth{\rmath}{\id{attributeGroup}}
\settowidth{\rtext}{repetition of $g$ between $m$ and $n$ times}

\newenvironment{grammar}{
  \[
  \begin{array}{l@{\quad}rcl@{\quad}l}
  \hspace{\ltext} & \hspace{\lmath} & \hspace{\cmath} & \hspace{\rmath} & \hspace{\rtext} \\
}{
  \end{array}\]
}

% Over-full v-boxes on even pages are due to the \v{c} in author's name
\vfuzz2pt % Don't report over-full v-boxes if over-edge is small

% THEOREM Environments ---------------------------------------------------
 \newtheorem{thm}{Theorem}[subsection]
 \newtheorem{cor}[thm]{Corollary}
 \newtheorem{lem}[thm]{Lemma}
 \newtheorem{prop}[thm]{Proposition}
 \theoremstyle{definition}
 \newtheorem{defn}[thm]{Definition}
 \theoremstyle{remark}
 \newtheorem{rem}[thm]{Remark}
 \newtheorem{example}[thm]{Example}
 \numberwithin{equation}{subsection}
% MATH -------------------------------------------------------------------
 \DeclareMathOperator{\RE}{Re}
 \DeclareMathOperator{\IM}{Im}
 \DeclareMathOperator{\ess}{ess}
 \newcommand{\veps}{\varepsilon}
 \newcommand{\To}{\longrightarrow}
 \newcommand{\h}{\mathcal{H}}
 \newcommand{\s}{\mathcal{S}}
 \newcommand{\A}{\mathcal{A}}
 \newcommand{\J}{\mathcal{J}}
 \newcommand{\M}{\mathcal{M}}
 \newcommand{\W}{\mathcal{W}}
 \newcommand{\X}{\mathcal{X}}
 \newcommand{\BOP}{\mathbf{B}}
 \newcommand{\BH}{\mathbf{B}(\mathcal{H})}
 \newcommand{\KH}{\mathcal{K}(\mathcal{H})}
 \newcommand{\Real}{\mathbb{R}}
 \newcommand{\Complex}{\mathbb{C}}
 \newcommand{\Field}{\mathbb{F}}
 \newcommand{\RPlus}{\Real^{+}}
 \newcommand{\Polar}{\mathcal{P}_{\s}}
 \newcommand{\Poly}{\mathcal{P}(E)}
 \newcommand{\EssD}{\mathcal{D}}
 \newcommand{\Lom}{\mathcal{L}}
 \newcommand{\States}{\mathcal{T}}
 \newcommand{\abs}[1]{\left\vert#1\right\vert}
% \newcommand{\set}[1]{\left\{#1\right\}}
%\newcommand{\seq}[1]{\left<#1\right>}
 \newcommand{\norm}[1]{\left\Vert#1\right\Vert}
 \newcommand{\essnorm}[1]{\norm{#1}_{\ess}}

%%% NAMES
\newcommand{\Names}{{\mathcal N}}
\newcommand{\Channels}{{\sf X}}
\newcommand{\Variables}{{\mathcal V}}
\newcommand{\Enames}{{\mathcal E}}
\newcommand{\Nonterminals}{{\mathcal S}}
\newcommand{\Pnames}{{\mathcal P}}
\newcommand{\Dnames}{{\mathcal D}}
\newcommand{\Types}{{\mathcal T}}
\newcommand{\lpquote}{\langle}
\newcommand{\rpquote}{\rangle}

\newcommand{\fcalc}{fusion calculus}
\newcommand{\xcalc}{${\mathfrak x}$-calculus}
\newcommand{\lcalc}{$\lambda$-calculus}
\newcommand{\pic}{$\pi$-calculus}
\newcommand{\hcalc}{highwire calculus}
\newcommand{\dcalc}{data calculus}
%XML should be all caps, not small caps. --cb
%\newcommand{\xml}{\textsc{xml}}
\newcommand{\xml}{XML} 

\newcommand{\papertitle}{New, on reflection}
% use static date to preserve date of actual publication
 \newcommand{\paperversion}{Draft Version 0.1 - December 30, 2003}

\newenvironment{toc}
{
\begin{list}{}{
   \setlength{\leftmargin}{0.4in}
   \setlength{\rightmargin}{0.6in}
   \setlength{\parskip}{0pt}
 } \item }
{\end{list}}

\newenvironment{narrow}
{
\begin{list}{}{
   \setlength{\leftmargin}{0.4in}
   \setlength{\rightmargin}{0.6in}
 } \item }
{\end{list}}

%%% ----------------------------------------------------------------------
\begin{document}

%These margin values appear to be relative to the bigpage package settings. --cb
\setlength{\topmargin}{0in}
\setlength{\textheight}{8.5in}
\setlength{\parskip}{6pt}

\title{\huge{\papertitle}}

\author{ L.G. Meredith }

\author{ Allen Brown }

\author{ David Richter }

\author{ Lucian Wischik }

%\address{Microsoft Corporation, Redmond, Washington, USA}

%\email{gregmer@microsoft.com}

\thanks{This work was completed under the BigTop project.}

%\subjclass{Primary 47A15; Secondary 46A32, 47D20}

\keywords{concurrency, distribution, message-passing, process calculus, service-oriented architecture, quality of service, service level agreement}

%\date{April 6, 2002.}

\dedicatory{}

%\commby{Daniel J. Rudolph}

%%% ----------------------------------------------------------------------

\begin{abstract}
\normalsize{ 

In pursuing an account of features for a practical programming
language based on mobile process algebras we have stumbled on a very
general approach to name-passing calculi. The features we were
considering, including implementation strategies for recursion; sending and
receiving complex data; macros; and durability across transactions may
all be related to \textit{serialization}, in the sense of producing an
external representation of a process. Surpisingly, one construction
provides a basis for an account for all of these features.

We illustrate the construction by applying it to the pure asynchronous
\pic-calculus, to arrive at a variant we dub the $\rho$-calculus. The
calculus is Turing-complete and yet is free of any explicit construct
for recursion or replication. Moreover, it endures no explicit
operator for the construction and binding of fresh names. Instead, it
introduces a reflective mechanism allowing for turning processes into
names and names into processes. This suffices to recapitulate all of
the functionality normally found in non-reflective process algebras
and also treat the phenomena mentioned afore..

}

\end{abstract}

\noindent
{\large \textbf{
Concur submission draft}}\\
\rule{6.25in}{0.75pt}\\\\\\

%%% ----------------------------------------------------------------------
\maketitle
%%% ----------------------------------------------------------------------

% \begin{center}
% \paperversion\\
% \end{center}

% \begin{toc}
% \tableofcontents
% \end{toc}

% \newpage
% ------------------------------------------------------------------------

\section{Introduction}

In pursuing an account of features for a practical programming
language based on mobile process algebras we have stumbled on a very
general approach to name-passing calculi. The features we were
considering are all related to \textit{serialization}, in the sense of
producing an external representation of a process. For example, one of
the key points of transaction semantics in systems like \texttt{Biztalk} is
\textit{durability}. A running process restarts from the last known
transacted outcome in the event of failure and restart. Such
durability requires (upon entering a transacted context) to be able to
serialize a representation of the state of the process and commit it
to durable store, like a database management system. It also requires
being able to deserialize that representation into an executing
process on restart.
 
Another example comes in the form of macros. A sophisticated macro
capability, such as one sees in implementations of the language
\texttt{scheme}, works by being able to turn program source into data,
operate on the data, and turn the modified data back into program
source. What unifies these two examples, in the mobile process algebra
setting, is that program source \textit{contains} program state in the
sense that there is no hidden virtual machine state that also needs to
be captured. Thus, having a solution to one means, effectively having
a solution for the other.

These two features also distinguished from each other by a certain
characteristic. In the case of recovery, program state may very well
be inspected by the recovering program to make decisions regarding
what branch to execute. That is, application level logic makes use of
reified process state. Even in the most sophisticated a macro systems,
on the other hand, the program logic is insulated from the
macro-expansion process. But, it is still the case, in the
sophisticated macro system, that the programming language in which
macros as expressed is the language in which program source is
expressed. In this sense, macros are 'meta-programs', or programs that
calculate programs, while the durability feature of systems like
\texttt{Biztalk} make some application programs also be meta-programs.
 
By way of introducing a nomenclature for discussing the results of the
paper we observe that-- although the usage is a bit stretched-- in
common parlance, a structured way of turning program into data and
data into program has been dubbed \textit{reflection}. In point of
fact there are several forms of reflection that may be distinguished
from one another.

\begin{itemize}
	\item structural;
	\item procedural;
	\item lexical;
\end{itemize}

Structural reflection is best exemplified in a language like
\texttt{Java}. Program elements, e.g. classes, may be turned into
program manipulable data structures. But, it is not the case that the
state of a virtual machine executing a \texttt{Java} program is
manipulable by a program running in that virtual machine. This is to
be contrasted with languages like \texttt{3-Lisp} and \texttt{Rosette}
where the state of the executing machinery is also made available as
program-manipulable data.

In lexical reflection, every element of the \textit{syntax} of the
language has a representation as a manipulable program data
structure. This is not the case in languages like \texttt{Java} or
\texttt{C\#}. But, it is found in \texttt{Rosette} and
\texttt{3-Lisp}. Note that while this is not generally recognized
under this term under in the literature, the phenomenon is real. Among
other things, this feature forms the basis of an architecture for
parser generation.

The approach to name-passing calculi that we have discovered
effectively allows a treatment of all three types of reflection within
the theory. 

In a similar vein, it is really beyond human capability to do
significant calculation with the \pic-calculus as it stands. Without
the ability to send and receive complex data, calculation quickly
becomes incomprehensible. One needs a higher level abstraction, for
example, to carry out any sort of ordinary arithmetic. Providing a
higher level language for data manipulation begs the question: how
does the data language relate to the control language? Why isn't there
only one language? Our approach to adding reflection to name-passing
calculi achieves a unification by making the program constructors of
the language \textit{be} the data constructors. Of course, it doesn't
specify which process constructors represent the data abstractions.

Surprisingly, it also has consequences for practical issues of space
management and the implementation of recursive features, like
$!P$. For example, if the $(\nu \; x)-$ operator is interpreted as an
operation involving memory allocation then there is a problem
satisfying the structural equivalence

\begin{equation*}
	(\nu \; x)(\nu \; x)P \scong (\nu \; x)P
\end{equation*}

The left hand side may cause a memory fault while the right hand side
does not. If, on the other hand, the operator is not interpreted as an
operation involving memory allocation, then what is the
interpretation? In point of fact, it says that the $\nu$-binder must
be \textit{lazily} related to memory allocation. Considering these
implementation strategies is at level of detail too rich for
name-passing process algebras presented to date.

Lucian Wischik was motivated by this problem (and others) with the
implementation of the $(\nu \; x)-$ operator to develop the \textit{global
\pic-calculus}. We were very much surprised to see how similar our
behavioral rules were to his when we developed this calculus. Of
course, there is a difference. The difference is that we say exactly
which set of names are the global names and provide a rationale for
picking this set: the names are the reified processes.

Another point that bears mentioning is the treatment of recursive
constructs, like $!P$. In most practical implementations this is
achieved by making program source (at some level of representation)
multiply instantiable. The mobile process algebras provide only a
specification of this behavior (in the form of the structural
congruence rules for $!P$, or in some presentations, in the
dynamics). They steadfastly remain silent on implementation
strategies, and rightly so. 

Our approach leads to a theory that at a sufficiently rich detail to
describe implementation strategies for this specification. In some
sense, the situation is analogous to the difference between the
$\lambda$-calculus and the \pic-calculus. The latter is a theory with
a richer, or more intensional structure, allowing a finer grain of
distinctions to be made. Likewise, the calculus at which we arrive is
a calculus that allows a more intensional characterization of the
features the name-passing calculi take for granted.

In this paper we illustrate the construction for a pure asynchronous
calculus. But, we stress that this construction works for name-passing
calculi, in general. 

%%% ----------------------------------------------------------------------

\section{The calculus}

By way of motivation we wish to make an observation about the
asynchronous \pic-calculus: it is a theory parametric in a theory of
\textit{names}. Standard presentations assume a supply of names and a
theory of equality on them. This point is made abundantly clear in
such presentations by the grammar describing terms of the language:
there is no production for names. Names are taken to be terminals in
the grammar. (Ed note: put in footnote about how \texttt{Biztalk} takes
advantage of this.)

We will adopt a different tac by closing the theory processes, making
processes serve as their own theory of names. So, the first point of
departure from a standard presentation of the asynchronous
\pic-calculus is a production for names in the grammar. A name will
be a \textit{quoted} process, $\lpquote \category{P} \rpquote$. 

Of course, if a name is a quoted process, it would be very handy to
have a way of \textit{evaluating} such an entity. Thus, one of term
constructors for processes takes a name, say $x$, and extracts the
process from it, $\rpquote x \lpquote$. 

Note that name construction is not process construction. So, if one
wants to be able to generate a name from a given process, there must
be an operator in the term language that creates a name from a
process. This is the motivation for the production $\texttt{lift} ; x \concat \category{P}$. This turns out to play a role analogous to
$(\nu \; x)P$. As such, it turns out not to be necessary to provide
the $\nu$-binder. We will illustrate the implementation of $\nu$ in
the sequel. 

On the basis of responses to talks about this material at this point
we issue a warning to the reader: it will turn out that quoted
processes are opaque to substitutions. But, this operator is not. This
will provide a key difference between $\texttt{lift} \; x \concat \category{P}$ and outputing a process.

It is part of folklore that replication is not required in a
higher-order process algebra. While our algebra is \textit{not}
higher-order in the traditional sense (there are not formal process
variables of a different type from names) it has all the features of a
higher-order process algebra. Thus, it turns out that there is no need
for a term for recursion. To illustrate this we present an
implementation of $!P$ in this calculus.

In standard presentations of the asynchronous \pic-calculus, the
statement of the communications rule critically depends on the
equality that comes with the theory of names. The situation is the
same here. For output and input processes to communicate it is
important to know how to determine whether the subject guards are the
same. But, another interesting point of departure between the calculi
is reflected in the fact that in the former, the theory of equality of
names is opaque. It must be give apriori, and this determination is
deemed independent of the theory of processes. In the latter, the
equality between names is not opaque, but given explicitly, and is
determined by the structure of processes. More about this point later,
though.

\begin{grammar}
\mbox{$\rho$-calculus}		& \category{P}			& \bc	& \pzero \\
				&					& \bm	& \category{x}\id{[}\category{x}\id{]} \\
				&					& \bm	& \category{x}\id{(}\category{x}\id{)} \concat \category{P} \\
				&					& \bm	& \id{lift} \; \category{x} \concat \category{P} \\
				&					& \bm	& \category{P} \juxtap \category{P} \\
				&					& \bm	& \rpquote \category{x} \lpquote \\
				& \category{x}  		& \bc	& \lpquote \category{P} \rpquote
\end{grammar}

\subsection{Substitution}

As is standard we use $\category{P}\id{\{}\vec{\category{x}_1} /
\vec{\category{x}_0} \id{\}}$ to denote capture-avoiding substitution
of the names $\vec{\category{x}_1}$ for the names
$\vec{\category{x}_0}$. But, some care must be taken regarding
substitutions and quoted processes, i.e. $\lpquote \category{P}
\rpquote$. Specifically, we must first equip out names with a theory
of equality. We take this to be the smallest equivalence relation
generated by the following equation.

\begin{eqnarray*}
\lpquote\rpquote\category{x}\lpquote\rpquote	& \scong & \category{x} \\
\end{eqnarray*}

Now, let $Proc$ denote the set of processes and
$\lpquote Proc \rpquote$ denote the set of names. Then, substitutions
are defined as maps, $s :\lpquote Proc \rpquote \rightarrow \lpquote
Proc \rpquote$. These lift, uniquely, to maps $\rpquote s \lpquote :
Proc \rightarrow Proc$ by the following equations.

\begin{eqnarray*}
(\category{0}) \rpquote \id{\{}\lpquote \category{Q} \rpquote / \lpquote \category{P} \rpquote \id{\}} \lpquote         & = &    0 \\
(\category{R} \juxtap \category{S}) \rpquote \id{\{} \lpquote \category{Q} \rpquote / \lpquote \category{P} \rpquote \id{\}} \lpquote      
		& = &    
		(\category{R})\rpquote \id{\{} \lpquote \category{Q} \rpquote / \lpquote \category{P} \rpquote \id{\}} \lpquote \juxtap (\category{S})\rpquote \id{\{} \lpquote \category{Q} \rpquote / \lpquote \category{P} \rpquote \id{\}} \lpquote \\
(\category{x} \id{(} \category{y}\id{)}\concat\category{R}) \rpquote \id{\{} \lpquote \category{Q} \rpquote / \lpquote \category{P} \rpquote \id{\}} \lpquote    
		& = &    
		(\category{x})\id{\{} \lpquote \category{Q} \rpquote / \lpquote \category{P} \rpquote \id{\}} \id{(}\category{y}\id{)}\concat( (\category{R}) \rpquote \id{\{} \lpquote \category{Q} \rpquote / \lpquote \category{P} \rpquote \id{\}} \lpquote ) \\
(\category{x}\id{[}\category{y}\id{]}) \rpquote \id{\{} \lpquote \category{Q} \rpquote / \lpquote \category{P} \rpquote \id{\}} \lpquote
		& = &    
		(\category{x})\id{\{} \lpquote \category{Q} \rpquote / \lpquote \category{P} \rpquote \id{\}} \id{[}(\category{y})\id{\{} \lpquote \category{Q} \rpquote / \lpquote \category{P} \rpquote \id{\}} \id{]} \\
(\id{lift} \; \category{x}\concat\category{R}) \rpquote \id{\{} \lpquote \category{Q} \rpquote / \lpquote \category{P} \rpquote \id{\}} \lpquote  
		& = &    
		\id{lift} \; (\category{x})\id{\{} \lpquote \category{Q} \rpquote / \lpquote \category{P} \rpquote \id{\}}\concat( (\category{R} ) \rpquote \id{\{} \lpquote \category{Q} \rpquote / \lpquote \category{P} \rpquote \id{\}} \lpquote ) \\
(\rpquote \category{x} \lpquote) \rpquote \id{\{} \lpquote \category{Q} \rpquote / \lpquote \category{P} \rpquote \id{\}} \lpquote       
		& = & 
		\left\{ 
			\begin{array}{ccc} 
				\category{Q} & & \category{x} \scong \lpquote \category{P} \rpquote \\
                              	\rpquote \category{x} \lpquote & & otherwise \\
			\end{array}
		\right.
\end{eqnarray*}
 

where

\begin{equation*}
(\category{x})\id{\{} \lpquote \category{Q} \rpquote / \lpquote \category{P} \rpquote \id{\}}            = 
		\left\{ 
			\begin{array}{ccc}
				\lpquote \category{Q} \rpquote & & \category{x} \scong \lpquote \category{P} \rpquote \\
                                \category{x} & & otherwise \\
			\end{array}
		\right.
\end{equation*}

Anticipating something of what's to come, these definitions should
make it clear that there is a significant difference between
$\id{lift} \; \category{x} \concat \category{P}$ and
$\category{x}\id{[}\lpquote\category{P}\rpquote\id{]}$. Specifically, consider the following pair of processes:

\begin{eqnarray*}
	\id{lift} \; w \concat y \id{[}z\id{]} \rpquote \id{\{}u / z \id{\}} \lpquote
		& = &
		\id{lift} \; w \concat y \id{[}u\id{]} \\
	w\id{[} \lpquote y\id{[}z\id{]} \rpquote \id{]} \rpquote \id{\{}u / z \id{\}} \lpquote
		& = &
		w\id{[} \lpquote y\id{[}z\id{]} \rpquote \id{]}
\end{eqnarray*}

Thus, the $\id{lift}$ operator will be seen as a way to dynamically
construct processes before reifying them as names.

\subsection{Operational Semantics}
The {\em structural congruence} of processes, noted $\scong$, is the least congruence that satisfies the following laws:
\begin{eqnarray*}
\category{P} \juxtap \pzero	&  \scong \; \category{P} \; \scong & \pzero \juxtap \category{P} \\
\category{P}_0 \juxtap \category{P}_1	& \scong & \category{P}_1 \juxtap \category{P}_0 \\
(\category{P}_0 \juxtap \category{P}_1) \juxtap \category{P}_2	& \scong & \category{P}_0 \juxtap (\category{P}_1 \juxtap \category{P}_2) \\
\end{eqnarray*}
\\
The reduction rules for $\rho$-calculus are
\infrule[Comm]
{  }
{{{\category{x}_{0}\id{[}\category{x}_{2}\id{]}}} \juxtap {\category{x}_{0}\id{(}\category{x}_{1}\id{)} \concat \category{P}}
\longrightarrow \category{P}\id{\{}\category{x}_{2} \id{/} \category{x}_{1} \id{\}}}

\infrule[Lift]
{  }
{{\id{lift} \; \category{x} \concat \category{P}}
\longrightarrow \category{x}\id{[}\lpquote\category{P}\rpquote\id{]}}

In addition, we have the following context rules:

\infrule[Par]{\category{P}_0 \longrightarrow \category{P}_1}{{\category{P}_0 \juxtap \category{P}_2} \longrightarrow {\category{P}_1 \juxtap \category{P}_2}}

\infrule[Equiv]{{\category{P}_2 \scong \category{P}_0} \andalso {\category{P}_0 \longrightarrow \category{P}_1} \andalso {\category{P}_1 \scong \category{P}_3}}{\category{P}_2 \longrightarrow \category{P}_3}

\section{Replication}

It is actually a part of folklore that recursion can be implemented in a higher-order process algebra. Here we give the construction explicitly for the $\rho$-calculus.

\begin{equation*}
!P \triangleq \id{lift} \; x \concat (x\id{(}y\id{)}\concat(\rpquote y\lpquote \juxtap x\id{[}y\id{]}) \juxtap P) \juxtap x\id{(}z\id{)}\concat(\rpquote z\lpquote \juxtap x\id{[}z\id{]})
\end{equation*}

\section{Name services}

The way the name service below works (it is possible to define other more
efficient ones) is to let the process that wants to use it give it a
list of names. The first one is sent on the port $i$; the rest or port
$x$; and then the client process signals it's done by sending a
message on $f$ with a callback port $b$ to receive the answer. The
service, $N$, hands back a name guaranteed to be distinct from all of
those it has received.

 
\begin{eqnarray*}
N(i,x,f,r)	& = &	f\id{(}b\id{)}\concat r\id{(}s\id{)} \concat \id{lift} \; b \concat \lpquote \pzero \rpquote\id{[}\lpquote \pzero \rpquote \id{]} \juxtap s\id{[}\lpquote \pzero \rpquote \id{]} \\
		&   &	\altern x\id{(}y\id{)} \concat r \id{(} s \id{)} \concat (\id{lift} \; r \concat(y\id{[}\lpquote \pzero \rpquote \id{]} \juxtap s \id{[} \lpquote \pzero \rpquote \id{]}) \juxtap N(i,x,f,r) )\\
		&   &	\altern i\id{(} y \id{)} \concat (\id{lift} \; r \concat y\id{[} \lpquote \pzero \rpquote \id{]} \juxtap N(i,x,f,r) )
\end{eqnarray*}
 

Remember, $!\category{P}$ and recursion are inter-definable. Recursion is used here
because it is slightly more readable. The calculus doesn't have
choice, to implement the choice one needs to add one more port to the
parameter list of $N$. This has elided that because it is an unimportant
complication. 

Notice that $N$ counts and remembers. Thus, implementing $N$ is tantamount
to showing that the calculus is Turing complete.

\section{Docs-as-procs}


We can now state specifically, how data becomes typed processes. Modifying the grammar like this

\begin{grammar}
\mbox{$\rho$-calculus}		& \category{P}			& \bc	& \pzero \\
				&					& \bm	& \category{x}\id{[}\category{y}\id{]} \\
				&					& \bm	& \category{x}\id{(}\category{z}\id{)} \concat \category{P} \\
				&					& \bm	& \id{lift} \; \category{x} \concat \category{P} \\
				&					& \bm	& \category{P} \juxtap \category{P} \\
				&					& \bm	& \rpquote \category{x} \lpquote \\
				& \category{x}  		& \bc	& \lpquote \category{P} \rpquote \\
				& \category{y}  		& \bc	& \lpquote \category{P} \rpquote, \category{P} \id{:} \category{T} \\
				& \category{z}  		& \bc	& \lpquote \category{P} \rpquote, \category{P} \id{:} \category{U}
\end{grammar}
 

where $\category{T}$ is required to be a type in a system that selects processes that are
pattern recognizers and $\category{U}$ is required to be a type in a system that
selects processes that represent values.

 
\section{Durability across transactions}

Consider the following program transformation which takes a channel, $z$, and turns the 

\begin{eqnarray*}
	x \id{(} y \id{)} \concat P 
		& = &
		 x \id{(} y \id{)} \concat \id{(} \id{lift} \; z \concat P \juxtap z \id{(}w \id{)}. \rpquote w \lpquote \id{)} \\
\end{eqnarray*}

This provides the basis for storing

\section{Conclusions and future work}

Quite enough done.

% ------------------------------------------------------------------------
%GATHER{Xbib.bib}   % For Gather Purpose Only
%GATHER{Paper.bbl}  % For Gather Purpose Only
\bibliographystyle{amsplain}
\bibliography{altofilo}

% ------------------------------------------------------------------------

% ------------------------------------------------------------------------

\end{document}
% ------------------------------------------------------------------------
