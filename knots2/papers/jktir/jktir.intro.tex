\section{Introduction}\label{sec:introduction} % (fold)
Knot and link tabulation continues to be a lively area of scientific
research and promises to be useful to areas of science such as quantum
computing and DNA unpacking \cite{Quantum1} \cite{Quantum2} \cite{DNA1}.  The past
20 years have seen major advances in knot classification, the
development of knot invariants, and computational methods for
tabulating knots and links.  There are tables of the alternating prime
knots of up to 23 crossings \cite{Hoste2005The-enumeration}
\cite{Rankin2004Enumerating-I} \cite{Rankin2004Enumerating-II} \cite{KnotsTo16}. While
current algorithms provide complete tables of knots (and links),
winnowing these tables of duplicates is a time consuming
task. Moreover, as knot tables have proved of use to researchers in
genetics, and may prove to be to researchers in quantum computation,
the need to search these tables in meaningful ways presents
itself. For example, of the 4,976,016,485 prime, non-oriented,
alternating knots with minimal crossing number of 22, which contain
the tangle corresponding to 5/3 (if any)?  Of course, knot invariants
are useful to distinguish knots, but few provide the basis for a formal
language of knot properties with which to identify classes of knots \emph{via}  logical
expressions in the language. 

\subsection{Summary of contributions and outline of paper}

Here we present a newly-found strong knot invariant that does give
rise to such a formal language of knot properties. Knots are
invariantly associated to expressions in Milner's $\pi$-calculus
\cite{milner91polyadicpi}, a member of a family of dynamic calculi
known as the mobile process calculi. These calculi were developed for
the analysis of concurrent computation \cite{ParrowWalker}
\cite{SangiorgiWalker}.  The invariant introduced here interprets knot equivalence up
to ambient isotopy as an equivalence of process dynamics known as
bisimulation \cite{DBLP:conf/lics/Sangiorgi04} \cite{Sangiorgi95} in
the concurrency theory literature. Of critical importance, and
somewhat surprisingly, these two notions of equivalence correspond
exactly on the image of the encoding: the main result of this paper is
that two knots are ambient isotopic if, and only if, their images
under the encoding are weakly bisimilar.

%\begin{theorem*}[main]
%There is a process calculus $\mathcal{P}$ and a procedure for encoding knots into $\mathcal{P}$ such that the encoding is well-defined and injective on the collection of ambient isotopy classes of knots.  
%\end{theorem}


Building on this result, we observe that dual to the process calculi
are a family of logics, the Hennessy-Milner logics (HMLs), providing a
logical language capable of the classification of processes via logical
properties. Factoring this capability through our encoding leads to the
development of a logical language identifying classes of knots with
logical properties. In particular, the spatial logics discovered
by Caires and Cardelli \cite{CairesC03} constitute
a particularly interesting sub-family of the HMLs having logical
connectives enabling us to take full advantage of key features of our
encoding. We illustrate the application of the logic in \cite{Caires04}
primarily \emph{via} examples of predicates that select features of
knots.

The paper is organized as follows. First, we provide a brief overview
of knot presentations relevant to this paper, given in the context of
the history of knot tabulation. Next, we give a brief overview of
process calculi, highlighting achievements in that field pertinent to
this paper. Section \ref{sec:knots_as_processes} begins the technical
exegesis with an introduction to Milner's polyadic $\pi$-calculus via
demonstration of process expressions that, much like the combinators
of Conway's tangle calculus, reflect aspects of knot structure
(crossings \emph{etc.}). The section concludes with an example
encoding of the trefoil knot. Section
\ref{sub:the_syntax_and_semantics_of_the_notation_system} gives a
condensed but complete formal presentation of the process
calculus. The main theorem is proved in section
\ref{sub:main_thm_proof_sketch}. This is followed by an account of
spatial logic via knots in section \ref{sub:characteristic_formulae}
and how this can be applied in the study and inquiry of knot tables,
with potential application in knot tabulation algorithms
themselves. Finally, new techniques are judged not solely by the
questions they answer but by the new questions to which they give
rise. In the conclusion, we identify areas of further investigation.
% section introduction (end)