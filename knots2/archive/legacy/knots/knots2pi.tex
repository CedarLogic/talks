% ------------------------------------------------------------------------
% LNCS LaTeX Paper ******************************************************
% ------------------------------------------------------------------------
% Submitted:      July 12 2005
% Final Version:  
% Accepted:       
% ------------------------------------------------------------------------
% This is a journal top-matter template file for use with AMS-LaTeX.
%%%%%%%%%%%%%%%%%%%%%%%%%%%%%%%%%%%%%%%%%%%%%%%%%%%%%%%%%%%%%%%%%%%%%%%%%%

%\documentclass{tran-l}
%\documentclass[twocolumn]{amsart}
%\documentclass[]{amsart}
\documentclass[]{llncs}

%\documentclass[]{entcs}
%\usepackage[]{prentcsmacro}

%\usepackage[active]{srcltx} % SRC Specials for DVI Searching
\usepackage {mathpartir}
%\usepackage {listings}
%\usepackage {array}
\usepackage{url}

% From Allen's stable.
%\usepackage{bigpage}
\usepackage{bcprules}
\usepackage{code}
\usepackage{amsfonts}
\usepackage{latexsym}
\usepackage{amssymb}
%\usepackage{caption}
\usepackage{multicol}

% Double brackets
\newcommand{\ldb}{[\![}
\newcommand{\rdb}{]\!]}
\newcommand{\ldrb}{(\!(}
\newcommand{\rdrb}{)\!)}
\newcommand{\lliftb}{\langle\!|}
\newcommand{\rliftb}{|\!\rangle}
% \newcommand{\lpquote}{\langle}
% \newcommand{\rpquote}{\rangle}
% \newcommand{\lpquote}{\lceil}
% \newcommand{\rpquote}{\rceil}
\newcommand{\lpquote}{\ulcorner}
\newcommand{\rpquote}{\urcorner}
\newcommand{\newkw}{\nu}

% SYNTAX
\newcommand{\id}[1]{\texttt{#1}}
\newcommand{\none}{\emptyset}
\newcommand{\eps}{\epsilon}
\newcommand{\set}[1]{\{#1\}}
\newcommand{\rep}[2]{\id{\{$#1$,$#2$\}}}
\newcommand{\elt}[2]{\id{$#1$[$#2$]}}
\newcommand{\infinity}{$\infty$}

\newcommand{\pzero}{\mathbin{0}}
\newcommand{\seq}{\mathbin{\id{,}}}
\newcommand{\all}{\mathbin{\id{\&}}}
\newcommand{\choice}{\mathbin{\id{|}}}
\newcommand{\altern}{\mathbin{\id{+}}}
\newcommand{\juxtap}{\mathbin{\id{|}}}
\newcommand{\concat}{\mathbin{.}}
\newcommand{\punify}{\mathbin{\id{:=:}}}
\newcommand{\fuse}{\mathbin{\id{=}}}
\newcommand{\scong}{\mathbin{\equiv}}
\newcommand{\nameeq}{\mathbin{\equiv_N}}
\newcommand{\alphaeq}{\mathbin{\equiv_{\alpha}}}
\newcommand{\names}[1]{\mathbin{\mathcal{N}(#1)}}
\newcommand{\freenames}[1]{\mathbin{\mathcal{FN}(#1)}}
\newcommand{\boundnames}[1]{\mathbin{\mathcal{BN}(#1)}}
%\newcommand{\lift}[2]{\texttt{lift} \; #1 \concat #2}
\newcommand{\binpar}[2]{#1 \juxtap #2}
\newcommand{\outputp}[2]{#1 \id{[} #2 \id{]}}
\newcommand{\prefix}[3]{#1 \id{(} #2 \id{)} \concat #3}
\newcommand{\lift}[2]{#1 \lliftb #2 \rliftb}
\newcommand{\quotep}[1]{\lpquote #1 \rpquote}
\newcommand{\dropn}[1]{\rpquote #1 \lpquote}

\newcommand{\newp}[2]{\id{(}\newkw \; #1 \id{)} #2}
\newcommand{\bangp}[1]{\id{!} #1}

\newcommand{\substp}[2]{\id{\{} \quotep{#1} / \quotep{#2} \id{\}}}
\newcommand{\substn}[2]{\id{\{} #1 / #2 \id{\}}}

\newcommand{\psubstp}[2]{\widehat{\substp{#1}{#2}}}
\newcommand{\psubstn}[2]{\widehat{\substn{#1}{#2}}}

\newcommand{\applyp}[2]{#1 \langle #2 \rangle}
\newcommand{\absp}[2]{\id{(} #1 \id{)} #2}

\newcommand{\transitions}[3]{\mathbin{#1 \stackrel{#2}{\longrightarrow} #3}}
\newcommand{\meaningof}[1]{\ldb #1 \rdb}
\newcommand{\pmeaningof}[1]{\ldb #1 \rdb}
\newcommand{\nmeaningof}[1]{\ldrb #1 \rdrb}

\newcommand{\Proc}{\mathbin{Proc}}
\newcommand{\QProc}{\quotep{\mathbin{Proc}}}

\newcommand{\entailm}{\mathbin{\vdash_{\mathfrak m}}} %matching
\newcommand{\entailp}{\mathbin{\vdash_{\mathfrak p}}} %behavioral
\newcommand{\entailv}{\mathbin{\vdash_{\mathfrak v}}} %validation
\newcommand{\congd}{\mathbin{\equiv_{\mathfrak d}}}
\newcommand{\congs}{\mathbin{\equiv_{\mathfrak s}}}
\newcommand{\congp}{\mathbin{\equiv_{\mathfrak p}}}
%\newcommand{\logequiv}{\mathbin{\leftrightarrow}}

\newcommand{\barb}[2]{\mathbin{#1 \downarrow_{#2}}}
\newcommand{\dbarb}[2]{\mathbin{#1 \Downarrow_{#2}}}

% From pi-duce paper
\newcommand{\red}{\rightarrow}
\newcommand{\wred}{\Rightarrow}
\newcommand{\redhat}{\hat{\longrightarrow}}
\newcommand{\lred}[1]{\stackrel{#1}{\longrightarrow}} %transitions
\newcommand{\wlred}[1]{\stackrel{#1}{\Longrightarrow}}

\newcommand{\opm}[2]{\overline{#1} [ #2 ]} % monadic
\newcommand{\ipm}[2]{{#1} ( #2 )} 
\newcommand{\ipmv}[2]{{#1} ( #2 )} % monadic
\newcommand{\parop}{\;|\;} % parallel operator
\newcommand{\patmatch}[3]{#2 \in #3 \Rightarrow #1}
\newcommand{\sdot}{\, . \,} % Space around '.'
\newcommand{\bang}{!\,}
%\newcommand{\fuse}[1]{\langle #1 \rangle} 
\newcommand{\fusion}[2]{#1 = #2} % fusion prefix/action
\newcommand{\rec}[2]{\mbox{\textsf{rec}} \, #1. \, #2}
\newcommand{\match}[2]{\mbox{\textsf{match}} \; #1 \; \mbox{\textsf{with}} \; #2}
\newcommand{\sep}{:}
\newcommand{\val}[2]{\mbox{\textsf{val}} \; #1 \; \mbox{\textsf{as}} \; #2}

\newcommand{\rel}[1]{\;{\mathcal #1}\;} %relation
\newcommand{\bisim}{\stackrel{.}{\sim}_b} %bisimilar
\newcommand{\wb}{\approx_b} %weak bisimilar
\newcommand{\bbisim}{\stackrel{\centerdot}{\sim}} %barbed bisimilar
\newcommand{\wbbisim}{\stackrel{\centerdot}{\approx}} %weak barbed bisimilar
\newcommand{\bxless}{\lesssim} %expansion less (amssymb required)
\newcommand{\bxgtr}{\gtrsim} %expansion greater (amssymb required)
\newcommand{\beq}{\sim} %barbed congruent
\newcommand{\fwbeq}{\stackrel{\circ}{\approx}} %weak barbed congruent
\newcommand{\wbeq}{\approx} %weak barbed congruent
\newcommand{\sheq}{\simeq} %symbolic hypereq
\newcommand{\wbc}{\approx_{cb}}

% rho logic

\newcommand{\ptrue}{\mathbin{true}}
\newcommand{\psatisfies}[2]{#1 \models #2}
\newcommand{\pdropf}[1]{\rpquote #1 \lpquote}
\newcommand{\plift}[2]{#1 \lliftb #2 \rliftb}
\newcommand{\pprefix}[3]{\langle #1 ? #2 \rangle #3}
\newcommand{\pgfp}[2]{\textsf{rec} \; #1 \mathbin{.} #2}
\newcommand{\pquant}[3]{\forall #1 \mathbin{:} #2 \mathbin{.} #3}
\newcommand{\pquantuntyped}[2]{\forall #1 \mathbin{.} #2}
\newcommand{\riff}{\Leftrightarrow}

\newcommand{\PFormula}{\mathbin{PForm}}
\newcommand{\QFormula}{\mathbin{QForm}}
\newcommand{\PropVar}{\mathbin{\mathcal{V}}}

% End piduce contribution

\newcommand{\typedby}{\mathbin{\:\colon}}
\newcommand{\mixedgroup}[1]{\id{mixed($#1$)}}
\newcommand{\cast}[2]{\id{CAST AS} \; #1 \; (#2)}
\newcommand{\bslsh}{\mathbin{\id{\\}}}
\newcommand{\bslshslsh}{\mathbin{\id{\\\\}}}
\newcommand{\fslsh}{\mathbin{\id{/}}}
\newcommand{\fslshslsh}{\mathbin{\id{//}}}
\newcommand{\bb}[1]{\mbox{#1}}
\newcommand{\bc}{\mathbin{\mathbf{::=}}}
\newcommand{\bm}{\mathbin{\mathbf\mid}}
\newcommand{\be}{\mathbin{=}}
\newcommand{\bd}{\mathbin{\buildrel {\rm \scriptscriptstyle def} \over \be}}
\newcommand{\category}[1]{\mbox{\bf #1}}

%GRAMMAR
\newlength{\ltext}
\newlength{\lmath}
\newlength{\cmath}
\newlength{\rmath}
\newlength{\rtext}

\settowidth{\ltext}{complex type name}
\settowidth{\lmath}{$xxx$}
\settowidth{\cmath}{$::=$}
\settowidth{\rmath}{\id{attributeGroup}}
\settowidth{\rtext}{repetition of $g$ between $m$ and $n$ times}

\newenvironment{grammar}{
  \[
  \begin{array}{l@{\quad}rcl@{\quad}l}
  \hspace{\ltext} & \hspace{\lmath} & \hspace{\cmath} & \hspace{\rmath} & \hspace{\rtext} \\
}{
  \end{array}\]
}

% Over-full v-boxes on even pages are due to the \v{c} in author's name
%\vfuzz2pt % Don't report over-full v-boxes if over-edge is small

% THEOREM Environments ---------------------------------------------------
% MATH -------------------------------------------------------------------
 \newcommand{\veps}{\varepsilon}
 \newcommand{\To}{\longrightarrow}
 \newcommand{\h}{\mathcal{H}}
 \newcommand{\s}{\mathcal{S}}
 \newcommand{\A}{\mathcal{A}}
 \newcommand{\J}{\mathcal{J}}
 \newcommand{\M}{\mathcal{M}}
 \newcommand{\W}{\mathcal{W}}
 \newcommand{\X}{\mathcal{X}}
 \newcommand{\BOP}{\mathbf{B}}
 \newcommand{\BH}{\mathbf{B}(\mathcal{H})}
 \newcommand{\KH}{\mathcal{K}(\mathcal{H})}
 \newcommand{\Real}{\mathbb{R}}
 \newcommand{\Complex}{\mathbb{C}}
 \newcommand{\Field}{\mathbb{F}}
 \newcommand{\RPlus}{\Real^{+}}
 \newcommand{\Polar}{\mathcal{P}_{\s}}
 \newcommand{\Poly}{\mathcal{P}(E)}
 \newcommand{\EssD}{\mathcal{D}}
 \newcommand{\Lom}{\mathcal{L}}
 \newcommand{\States}{\mathcal{T}}
 \newcommand{\abs}[1]{\left\vert#1\right\vert}
% \newcommand{\set}[1]{\left\{#1\right\}}
%\newcommand{\seq}[1]{\left<#1\right>}
 \newcommand{\norm}[1]{\left\Vert#1\right\Vert}
 \newcommand{\essnorm}[1]{\norm{#1}_{\ess}}

%%% NAMES
\newcommand{\Names}{{\mathcal N}}
\newcommand{\Channels}{{\sf X}}
\newcommand{\Variables}{{\mathcal V}}
\newcommand{\Enames}{{\mathcal E}}
\newcommand{\Nonterminals}{{\mathcal S}}
\newcommand{\Pnames}{{\mathcal P}}
\newcommand{\Dnames}{{\mathcal D}}
\newcommand{\Types}{{\mathcal T}}

\newcommand{\fcalc}{fusion calculus}
\newcommand{\xcalc}{${\mathfrak x}$-calculus}
\newcommand{\lcalc}{$\lambda$-calculus}
\newcommand{\pic}{$\pi$-calculus}
\newcommand{\spic}{spi-calculus}
\newcommand{\rhoc}{$\rho$-calculus}
\newcommand{\rhol}{$\rho$-logic}
\newcommand{\hcalc}{highwire calculus}
\newcommand{\dcalc}{data calculus}
%XML should be all caps, not small caps. --cb
%\newcommand{\xml}{\textsc{xml}}
\newcommand{\xml}{XML} 

\newcommand{\papertitle}{Knots as processes: a new kind of invariant}
% use static date to preserve date of actual publication
\newcommand{\paperversion}{Draft Version 0.1 - December 19, 2004}

\newenvironment{toc}
{
\begin{list}{}{
   \setlength{\leftmargin}{0.4in}
   \setlength{\rightmargin}{0.6in}
   \setlength{\parskip}{0pt}
 } \item }
{\end{list}}

\newenvironment{narrow}
{
\begin{list}{}{
   \setlength{\leftmargin}{0.4in}
   \setlength{\rightmargin}{0.6in}
 } \item }
{\end{list}}

\def\lastname{Meredith and Snyder}
%%% ----------------------------------------------------------------------
\begin{document}

%\begin{frontmatter}
\title{Knots as processes: a new kind of invariant}
\titlerunning{Namespace logic}

\author{ L.G. Meredith\inst{1} \and David F. Snyder\inst{2} }
\institute{ CXO, Biosimilarity\\ 505 N72nd St, Seattle, WA 98103, USA, \\
  \email{ lgreg.meredith@gmail.com } \\
  \and Associate Professor of Mathematics\\ Dept of Mathematics\\ Texas State University\\ 601 University Drive \\ San Marcos, TX 78666 \\
  \email{ dsnyder@txstate.edu }
} 

\maketitle              % typeset the title of the contribution

%%% ----------------------------------------------------------------------

\begin{abstract}

          We exhibit an encoding of knots into processes in the
          $\pi$-calculus such that knots are ambient isotopic if and
          only their encodings are weakly bisimilar.

\end{abstract}

% \begin{keyword}
% concurrency, message-passing, process calculus, reflection, program logic
% \end{keyword}

%\end{frontmatter}

\section{Introduction and motivation}

Compositionality.

\subsection{Overview and summary of contributions}

\section{The $\pi$-calculus in a nutshell}

\subsection{\pic}

\begin{grammar}
\mbox{\pic} & {P, Q} & \bc & \pzero \\
& & \bm & {x}\id{[}{y}\id{]} \\
& & \bm & {x}\id{(}{y}\id{)} \concat {P} \\
& & \bm & \newp{{x}}{{P}} \\
& & \bm & {P} \juxtap {Q} \\
& & \bm & \bangp{{P}}
\end{grammar} 

Note well: names are quoted \emph{$\rho$-calculus} processes.

\subsection{Structural congruence}

\begin{definition}
The {\em structural congruence}, $\equiv$, between processes is 
the least congruence closed with respect to
alpha-renaming, satisfying the abelian monoid laws for 
parallel (associativity, commutativity and $\pzero$ 
as identity), and the following axioms:
\begin{enumerate}
\item the scope laws:
\begin{eqnarray}
 \newp{x}\pzero  & \equiv & \pzero, \nonumber\\
 \newp{x}{\newp{x}{P}} & \equiv & \newp{x}{P}, \nonumber\\
 \newp{x}{\newp{y}{P}} & \equiv & \newp{y}{\newp{x}{P}}, \nonumber\\
 \binpar{P}{\newp{x}{Q}} & \equiv & \newp{x}{\binpar{P}{Q}}, \; \mbox{\textit{if} }x \not\in \freenames{P} \nonumber
\end{eqnarray}
\item
the recursion law:
\begin{eqnarray}
 \bangp{P} \equiv \binpar{P}{\bangp{P}} \nonumber
\end{eqnarray}
\end{enumerate}
\end{definition}

\subsection{Operational semantics} The operational semantics is standard.

\infrule[Comm]
{ }
{{{{x}\id{[}{z}\id{]}}} \juxtap {{x}\id{(}{y}\id{)} \concat {P}}
\red {P}\id{\{}{z} \id{/} {y} \id{\}}}

In addition, we have the following context rules:

\infrule[Par]{{P} \red {P}'}{{{P} \juxtap {Q}} \red {{P}' \juxtap {Q}}}

\infrule[New]{{P} \red {P}'}{{\newp{{x}}{{P}}} \red {\newp{{x}}{{P}'}}}

\infrule[Equiv]{{{P} \scong {P}'} \andalso {{P}' \red {Q}'} \andalso {{Q}' \scong {Q}}}{{P} \red {Q}}

Again, we write $\wred$ for $\red^*$, and rely on context to
distinguish when $\red$ means reduction in the {\pic} and when it
means reduction in the {\rhoc}. The set of {\pic} processes will be
denoted by $\Proc_{\pi}$.

\section{Stuff about knots}

A knot is usually represented by a knot projection, which satisfies
some basic general position properties.

\subsection{Reidemeister moves}

There are three Reidemeister moves that may be performed upon a knot
projection, which we denote by $\Omega_1, \Omega_2, \Omega_3$. These
are illustrated in the figure (Figure \ref{RMoves}).

\begin{figure}[center]
\vspace{2.5in}
\caption{The three Reidemesiter moves}
\label{RMoves}
\end{figure}

\subsection{The Dowker-Thistlethwaite code}

The Dowker-Thistlethwaite code (``DT'') of a knot $K$  is obtained as follows:


Choose a point on $K$ and begin traversing  $K$ , counting each 
 crossing you pass through. If $K$
has $n$  crossings, then (since  every crossing is visited twice) the
count ends at $2n$. Label each crossing with the value of the counter
when it is visited (each crossing is labeled twice).
 Finally, when labeling a crossing with
 an even number,  prepend with the label
with a minus sign if traversing ``under'' the crossing.
All crossings end up being labeled by a pair of integers whose absolute values
run, \emph{in toto}, from $1$ to $2n$. It is easy to see that each
 crossing is labeled with 
one odd integer and  one even integer. For each odd integer $j$
 between $1$ and $2n-1$ inclusive, let
 $\textrm{pairedWith}(j)$ be the even integer with which it is paired.
The DT code is 
the sequence $1, \textrm{pairedWith}(1), 3, 
\textrm{pairedWith}(3), \ldots, 2n-1, \textrm{pairedWith}(2n-1)$.




\section{The encoding}


\section{Ambient isotopy as weak bisimilarity}

It is well-known that two knot presentations are ambient isotopic if
and only if there is a sequence of Reidemeister moves transforming one
presentation to the other. There are three types of Reidemeister moves
and within each type we have a move and its inverse, as shown in the
figure.


We first establish a lemma to our purposes, that shows we can clean up our knot diagram in a particularly suitable way.

Given a knot $K$ and a diagram $D(K)$ of $K$, we  call the application of a 
Reidemeister move $\rho$ a {\em neatening} move if the resulting diagram $D'(K)$ has a crossing number less than or
equal to the crossing number of $K$ i.e. any neatening move
of the type $\Omega_{1}$ or $\Omega_{2}$ can occur in
only one direction. Let $\hat{\rho}=\rho_{1}\cdots\rho_{n}$ be a sequence of successive Reidemeister moves on the
diagram $D(K)$ (with $n=0$ being the identity move). We call $\hat{\rho}$ a {\em neatening isotopy} if either
$n=0$ or  each
$\rho_{i}$
is a neatening move. If, in addition, at least one of the $\rho_{i}$ is a move of  type $\Omega_{1}$ or $\Omega_{2}$, we
call $\hat{\rho}$ a {\em cleaning isotopy} of $D(K)$. If $\rho_{n}$ is  move of  type $\Omega_{1}$ or $\Omega_{2}$, we
call $\hat{\rho}$ a {\em concise}  cleaning isotopy of $D(K)$

For a given diagram $D(K)$, the collection of all its neatening isotopies can be given a partial ordering: given
neatening isotopies $\hat{\rho}=\rho_{1}\cdots\rho_{n}$ and $\hat{\sigma}=\sigma_{1}\cdots\sigma_{m}$, we put
$\hat{\rho}\le\hat{\sigma}$ when $n\le m$ and $\rho_{i}=\sigma_{i}$  for all $i\le n$. This partial order restricts to the
the subcollection  of concise cleaning isotopies.

\begin{lemma}\label{simplificationLemma} For any knot  $K$
and knot diagram $D(K)$, we may reduce
$D(K)$ via a neatening isotopy to a diagram
$D'(K)$  having the property that no neatening isotopy of $D'(K)$ is a cleaning isotopy of $D'(K)$ .
\end{lemma}
\begin{proof} If the collection of cleaning isotopies of $D(K)$ is an empty one, then $D(K)$ is the desired diagram.
So assume there is at least one cleaning isotopy of $D(K)$.

The collection of concise cleaning isotopies of
$D(K)$
forms a partially ordered set. Since each
concise cleaning
isotopy reduces the
number of crossings in the diagram by at least 1, each chain in the partially order set has length bounded by the
number of crossings in $D(K)$. Select any maximal element $\hat{\rho}$ of the partially ordered set. Let $D'(K)$ be
the diagram derived from $D(K)$ by applying $\hat{\rho}$. 
\end{proof}



\subsection{Encoding the Reidemeister moves}



\section{Conclusions and future work}

Compositionality.

Generalizing crossings.

Braids and tangles.

Geometry and concurrency has been investigated in one direction. This
work suggests that the other might also be fruitful: algebras with dynamics as invariants for
spatial systems.

\paragraph{Acknowledgments.}
The author wishes to acknowledge his longstanding debt to Samson
Abramksy for his foundational insights into the Curry-Howard
isomorphism.

% ------------------------------------------------------------------------
%GATHER{Xbib.bib}   % For Gather Purpose Only
%GATHER{Paper.bbl}  % For Gather Purpose Only
\bibliographystyle{plain}
\bibliography{knots2pi}

% ------------------------------------------------------------------------

% ------------------------------------------------------------------------

\end{document}
% ------------------------------------------------------------------------
