\section{Introduction}\label{sec:introduction} % (fold)
For the past several centuries there has been no serious competitor to
the ``Newtonian'' account of dynamics in the physical sciences. Here
we mean the mathematical apparati of the integral and derivative and
the principles of applications of those apparati to modeling dynamical
systems, such as the Lagrangian and Hamiltonian \cite{375178}. As a
result the predominant share of accounts of dynamical systems and
situations have had to be formulated in terms of the Newtonian
machinery. The present author views this as an intellectually
dangerous position to occupy. Everything, despite it's intrinsic
shape, turns into a nail to be hit with this hammer. For example, even
challenges to classical accounts of physics as deep as General
Relativity \cite{Gravitation} and Quantum mechanics \cite{Dirac1930}
are still \emph{essentially} couched in the methods and procedures
derived from the Newtonian integral and derivative \footnote{The
  covariant derivative still bears the imprimature of the original
  notion. Inner product in the wave-function formulation is still an
  integral}. Recently, however, the theory of computation has matured
to the point where we have candidates for theories of dynamics that
offer very different perspectives on reasoning about dynamical systems
and situations. It is time to see just how good they are by addressing
the dynamics of interest in the physical sciences. Testing these
candidates against very successful accounts of dynamical situations,
like quantum mechanics, is going to give us some sense of maturity,
quality and range of applicability of computational notions of
dynamics.

To address this program we realize and carry out the calculations of
quantum mechanics over a novel formal theory of dynamics, a formal
theory of dynamics that corresponds to a theory of concurrent
computation with \emph{reflection}
\cite{bcsmith:phdthesis},\cite{319871}. It has the advantage that the
underlying theory is already `quantized', but supports analogues all
of the continuuous operations. Strikingly, this underlying theory has
recently been connected with a notion of metric
\cite{DBLP:conf/fossacs/BreugelMOW03} that coincides with the metric
induced by the inner product.

\subsection{Summary of contributions and outline of paper}

The plan is to develop an interpretation of the operations of quantum
mechanics normally interpreted by Hilbert spaces and operators
\cite{MathematicalFoundationsQM}. The primary novelty of the approach
is to do this over a theory of computation. Note that this is very
different than the usual quantum computation program which develops
notions of computation over quantum mechanics \cite{544199}. Rather,
we are developing a story that aligns with Wheeler's slogan: It from
Bit \cite{HowComeTime}. To do this we will first provide an account of
the theory of computation at play here. Then we will dive into a
calculation-driven interpretation of the operations of quantum
mechanics.

The reason we take this approach is that -- until very recently
(\cite{DBLP:conf/lics/Abramsky04},
\cite{DBLP:conf/calco/Abramsky05})-- there hasn't been an axiomatic
account of quantum mechanics. As a result there has been no sharp
delineation of the mathematical theory supporting interpretation of
the physical theory and the physical theory, itself. So, ambient
features of the maths are free to be exploited (or suppressed) without
a real accounting of their physical relevance. There is no sharp
statement ``here's the physical theory'' qua \emph{theory} and
``here's the mathematical interpretation'' enabling a judgment of how
faithful the interpretation is -- apart from experimental
observation. When there is an axiomatic account we can judge how well
a given mathematical formalism supports an interpretation of the
axioms, independent of experimentation. Likewise, we can judge how
well we have captured our physical evidence and experience with our
axiomatics, independent of any specific mathematical implementation,
with accidental detail that may or may not have physical significance.

In lieu of a fully fleshed out and vetted axiomatic account of quantum
mechanics, interpreting the operational notions in service of modeling
physical systems will have to suffice. In other words, we are not in
the business of providing a model of Hilbert spaces and operators. We
are in the business of providing a model of quantum mechanics because
we are motivated by testing our notions of dynamics against physical
theory; and, the predictive calculations of the physical theory must
serve as the best formulation -- shy of a fully fleshed out axiomatic
account -- of the physical theory itself (as they have for scientific
theories since time immemorial). Put another way, despite a
whole-hearted commitment to an It-from-Bit ontology, we are firmly
aligned with the shut-up-and-calculate camp as the best way to obtain
results either from the physical perspective or as a quality assurance
measure of our fledgling theory of dynamics.

In detail, we present a reflective process calculus. Then we develop
intuitive correspondences between the notions available in this
calculus and the usual physical notions supporting quantum mechanical
calculations. 

\begin{table}[htp]
  \center{
    \fbox{
      \begin{tabular}{c|c}
        quantum mechanics & process calculus \\
        \hline
        scalar & name \\
        state vector & process \\
        dual & contextual duals \\
        matrix & formal sums of process-context-dual pairs \\
        orthogonality & process annihilation \\
        inner product & execution-formula + quoting
      \end{tabular}
    }
  }
  \caption{QM - process calculi correspondences}
\end{table}

Then we tighten up these intuitions to operational definitions. We
employ the Dirac notation as the best proxy we can find for an
abstract syntax of the quantum mechanical notions. The definitions we
develop put us in contact with equational constraints coming from the
theory that we demonstrate the definitions and calculations satisfy.

This puts us in a position to shut up and calculate for the
Stern-Gerlach experimental set up, showing how these predictive
calculations become calculations on processes in our theory of a
reflective process calculus. The particular example is of interest
because it is a composite experiment, consisting of iterated
measurements. The framework developed here is particularly suited to
reasoning about composite physical systems and situations.

Penultimately, we demonstrate that the notion of metric coming from
the inner product coincides with the notion of metric available from
the theory of bisimulation. This demonstration gives us the right to
think of space as arising from behavior. Finally, we consider where we
might go from the new vantage point we have obtained.

% section introduction (end)