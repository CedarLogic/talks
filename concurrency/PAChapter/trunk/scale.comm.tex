\section{Evolution and communication}

As noted above, the original comm rule for Milner's $\pi$-calculus was
of the form

\begin{mathpar}
  \inferrule* [lab=comm] {} {x?(y).P | x!(z).Q \to P\{z/y\} |Q}
\end{mathpar}

Later, when Milner analyzed the scoping of input binding he realized
that the sequencing operator was given a more hygenic account as
synchronization followed by data-exchange. Compare with the
productions of the grammar: sequence is effectively a function taking
names and agents, while agents is the disjoint sum of two types:
abstractions and concretions. In the case of the $\pi$-calculus we
have abstractions of the form $A \equiv ?(y)P$ (i.e. functions from
names to processes) and concretions of the form $C \equiv (\nu u)!(z)Q$
(i.e. pairs of names and processes).

This refactoring leads to a considerably more parsimonious comm rule.

\begin{mathpar}
  \inferrule* [lab=$comm_{AC}$] {} {x.A | x.C \to A \cdot C}
\end{mathpar}

where $A \cdot C$, called pseudo-application, is then defined in this context as

\begin{equation}
  ?(y)P \cdot (\nu u)!(z)Q \triangleq (\nu u)P\{z/y\}|Q \nonumber
\end{equation}

\subsubsection{Synchronization algebras}

We need to make explicit another piece of information implicit in the
standard presentation of the $\pi$-calculus. Synchronization
fundamentally depends on name equality. In the comm rule, the
sequenced abstraction may be thought of, roughly, as located at the
`?'-end of a channel, $x$, and the sequenced concretion at the
`!'-end. It is name equality, i.e. that $x = x$, that determines that
these agents are located at the opposite ends of the \emph{same}
channel and thus may communicate. We generalize this notion of
co-location by adding a predicate, $\perp : name \times name \to
bool$, explicitly determining when two names enable synchronization
and data-exchange. Our comm rule then becomes

\begin{mathpar}
  \inferrule* [lab=$comm_{AC^{\perp}}$] {x_{src} \perp x_{trgt}} {x_{src}.A | x_{trgt}.C \to A \cdot C}
\end{mathpar}

In this context we note that Milner saw this as an avenue of
investigation. In SCCS, for example, \cite{Milner83SCCS} he considered
actions with the additional structure of a group and a communication
rule that depended on action annihilation (i.e. that the channel and
co-channel were group inverses of each other). Additionally, other
investigators have exploited this possibility. Meredith and Radestock,
for example, use it to derive a reflective, two-level comm rule in
which co-location is calculated using the process dynamics itself
\cite{meredith2005rho}.

Intuitively, suppose we had two processes that were written to
communicate over protocols that were at different levels in a network
stack, e.g. in the network setting, one might be written to accept and
emit \texttt{TCP/IP} packets while the other was written to accept and
emit \texttt{HTTP} over \texttt{TCP/IP}. It is common practice to
think of the process that communicates at the higher layer as
essentially `wrapped' by a set of communications behaviors that have
been factored from the core behavior of the process. If $P$ is the
process written at the higher layer, we can faithfully represent that
division of labor by $K[P]$, where $K$ is a context that constitutes
the communications behaviors of packaging up \texttt{HTTP} requests
and responses over \texttt{TCP/IP}.

% In the higher-order context, we might effect the communication between
% the two processes as $\lift{x}{P} | x(y).(K[\dropn{y}]|Q)$. In the
% context of pseudo-application, however, it is natural to consider a
% $K$-indexed version of the operator defined as 

% \begin{equation}
%   (y)Q \cdot_{K} \lift{}{P} \triangleq Q\{\quotep{K[P]}/y\} \nonumber
% \end{equation}

In terms of our intuitive interpretation, such $K$ represents a
protocol mediator, mediating between the layer at which $P$ engages
and the layer at which $Q$ engages. The question then becomes how to
select these mediators? This is precisely where the expansion of the
co-location predicate from boolean-valued to taking values in some
other domain comes into play. We use the pair of names (in terms of
our networking analogy, the pair of communications port abstractions
at the different layers in the networking stack) to determine the
mediating context. So, the co-location ``predicate'' becomes a
context-valued function. Thus, the general expression of our comm rule
becomes

\begin{mathpar}
  \inferrule* [lab=$comm_{AC^{\perp}K}$] {x_{src} \perp x_{trgt} = K} {x_{src}.A | x_{trgt}.C \to A \cdot_{K} C}
\end{mathpar}

Obviously, the relation $x \perp_{\Box} x \triangleq \Box$ with $x
\perp_{\Box} y$ undefined if $x \neq y$, recovers the original comm
rule. In situations other than the trivial case, this rule must be is
instantiated by an appropriate notion of pseudo-application. 

