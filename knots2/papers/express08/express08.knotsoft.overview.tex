\section{A brief overview of knot software}\label{sec:a_brief_overview_of_knot_software} % (fold)
This is a brief overview of knot software, providing further context as to the relevance
of this proposal. A general shortcoming in knot tables is that one
cannot search the tables based on a particular criterion, such as:
find all the genus 7 knots having minimal crossing number between
15 and 17.

The online Table of Knot Invariants \cite{KnotInvariants} lets the user choose
one of the knot tables (or subtables) for knots of 12 or fewer
crossings and choose from a wide variety of invariants and notations
that they wish to see. The resulting table is given with the desired
invariants and notations listed for each knot.

The online KnotAtlas \cite{KnotAtlas} contains Mathematica� code (the
package KnotTheory \cite{KnotTheory}) and a visual database of knots and
links up to 11 crossings.

Rob Scharein's KnotPlot is extensive and
extensible software that has a multiplicity of capabilities  \cite{KnotPlot} \cite{SchareinPhD}. It
contains a visual database, for example, of all knots up to 10
crossings. KnotPlot also supports tangle calculus and many other
features. KnotPlot is open source software and can be used to generate
knot representations suitable for import into the PIs' software
system. Scharein has built the Knot Server \cite{KnotServer}, which is
intended to data and invariants of knots within the tables of
KnotPlot, though at this writing the calculation of invariants is not
functional.

N. Imafuji and M. Ochiai \cite{Imafuji2002Computer-aided-} developed Knot2000 (K2K), a
Mathematica�-based package that allows for extensive computation with
knots and links. S. Jablan and R. Sazdanovic \cite{LinKnot} have used this
package and their own webMathematica� code to develop the excellent
on-line site LinKnot that introduces computational methods in knot
tabulation and discusses other aspects of knot theory as well.

The website Knotilus \cite{Knotilus} of Rankin et al. has extensive tables
of knots and links up to 23 crossings, with browsing of these tables
possible via a known Gauss or Dowker-Thistlethwaite code, or via a
classification scheme, with a java-based applet for displaying (and
drawing) knots.

Gruber \cite{RationalKnotsTable} has posted online tables of rational knots of up to
16 crossings, though some errors in the tables exist, only a few
invariants are calculated, and the tables are not searchable.
% section a_brief_overview_of_knot_software (end)