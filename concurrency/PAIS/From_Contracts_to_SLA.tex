% ------------------------------------------------------------------------
% AMS-LaTeX Paper ********************************************************
% ------------------------------------------------------------------------
% Submitted:      Dec 15 2003
% Final Version:  
% Accepted:       
% ------------------------------------------------------------------------
% This is a journal top-matter template file for use with AMS-LaTeX.
%%%%%%%%%%%%%%%%%%%%%%%%%%%%%%%%%%%%%%%%%%%%%%%%%%%%%%%%%%%%%%%%%%%%%%%%%%

%\documentclass{tran-l}
\documentclass{amsart}

%\usepackage[active]{srcltx} % SRC Specials for DVI Searching
\usepackage{url}

% From Allen's stable.
\usepackage{bigpage}
\usepackage{bcprules}
\usepackage{code}
\usepackage{amsfonts}
\usepackage{latexsym}
\usepackage{amssymb}
\usepackage{caption}

% SYNTAX
\newcommand{\id}[1]{\texttt{#1}}
\newcommand{\none}{\emptyset}
\newcommand{\eps}{\epsilon}
\newcommand{\set}[1]{\{#1\}}
\newcommand{\rep}[2]{\id{\{$#1$,$#2$\}}}
\newcommand{\elt}[2]{\id{$#1$[$#2$]}}
\newcommand{\infinity}{$\infty$}
\newcommand{\seq}{\mathbin{\id{,}}}
\newcommand{\all}{\mathbin{\id{\&}}}
\newcommand{\choice}{\mathbin{\id{|}}}
\newcommand{\altern}{\mathbin{\id{+}}}
\newcommand{\juxtap}{\mathbin{\id{|}}}
\newcommand{\concat}{\mathbin{\id{.}}}
\newcommand{\punify}{\mathbin{\id{:=:}}}
\newcommand{\fuse}{\mathbin{\id{=}}}
\newcommand{\entailm}{\mathbin{\vdash_{\mathfrak m}}} %matching
\newcommand{\entailp}{\mathbin{\vdash_{\mathfrak p}}} %behavioral
\newcommand{\entailv}{\mathbin{\vdash_{\mathfrak v}}} %validation
\newcommand{\congd}{\mathbin{\equiv_{\mathfrak d}}}
\newcommand{\congs}{\mathbin{\equiv_{\mathfrak s}}}
\newcommand{\congp}{\mathbin{\equiv_{\mathfrak p}}}
\newcommand{\typedby}{\mathbin{\:\colon}}
\newcommand{\mixedgroup}[1]{\id{mixed($#1$)}}
\newcommand{\cast}[2]{\id{CAST AS} \; #1 \; (#2)}
\newcommand{\bslsh}{\mathbin{\id{\\}}}
\newcommand{\bslshslsh}{\mathbin{\id{\\\\}}}
\newcommand{\fslsh}{\mathbin{\id{/}}}
\newcommand{\fslshslsh}{\mathbin{\id{//}}}
\newcommand{\bb}[1]{\mbox{#1}}
\newcommand{\bc}{\mathbin{\mathbf{::=}}}
\newcommand{\bm}{\mathbin{\mathbf\mid}}
\newcommand{\be}{\mathbin{=}}
\newcommand{\bd}{\mathbin{\buildrel {\rm \scriptscriptstyle def} \over \be}}
\newcommand{\category}[1]{\mbox{\bf #1}}

%GRAMMAR
\newlength{\ltext}
\newlength{\lmath}
\newlength{\cmath}
\newlength{\rmath}
\newlength{\rtext}

\settowidth{\ltext}{complex type name}
\settowidth{\lmath}{$xxx$}
\settowidth{\cmath}{$::=$}
\settowidth{\rmath}{\id{attributeGroup}}
\settowidth{\rtext}{repetition of $g$ between $m$ and $n$ times}

\newenvironment{grammar}{
  \[
  \begin{array}{l@{\quad}rcl@{\quad}l}
  \hspace{\ltext} & \hspace{\lmath} & \hspace{\cmath} & \hspace{\rmath} & \hspace{\rtext} \\
}{
  \end{array}\]
}

% Over-full v-boxes on even pages are due to the \v{c} in author's name
\vfuzz2pt % Don't report over-full v-boxes if over-edge is small

% THEOREM Environments ---------------------------------------------------
 \newtheorem{thm}{Theorem}[subsection]
 \newtheorem{cor}[thm]{Corollary}
 \newtheorem{lem}[thm]{Lemma}
 \newtheorem{prop}[thm]{Proposition}
 \theoremstyle{definition}
 \newtheorem{defn}[thm]{Definition}
 \theoremstyle{remark}
 \newtheorem{rem}[thm]{Remark}
 \newtheorem{example}[thm]{Example}
 \numberwithin{equation}{subsection}
% MATH -------------------------------------------------------------------
 \DeclareMathOperator{\RE}{Re}
 \DeclareMathOperator{\IM}{Im}
 \DeclareMathOperator{\ess}{ess}
 \newcommand{\veps}{\varepsilon}
 \newcommand{\To}{\longrightarrow}
 \newcommand{\h}{\mathcal{H}}
 \newcommand{\s}{\mathcal{S}}
 \newcommand{\A}{\mathcal{A}}
 \newcommand{\J}{\mathcal{J}}
 \newcommand{\M}{\mathcal{M}}
 \newcommand{\W}{\mathcal{W}}
 \newcommand{\X}{\mathcal{X}}
 \newcommand{\BOP}{\mathbf{B}}
 \newcommand{\BH}{\mathbf{B}(\mathcal{H})}
 \newcommand{\KH}{\mathcal{K}(\mathcal{H})}
 \newcommand{\Real}{\mathbb{R}}
 \newcommand{\Complex}{\mathbb{C}}
 \newcommand{\Field}{\mathbb{F}}
 \newcommand{\RPlus}{\Real^{+}}
 \newcommand{\Polar}{\mathcal{P}_{\s}}
 \newcommand{\Poly}{\mathcal{P}(E)}
 \newcommand{\EssD}{\mathcal{D}}
 \newcommand{\Lom}{\mathcal{L}}
 \newcommand{\States}{\mathcal{T}}
 \newcommand{\abs}[1]{\left\vert#1\right\vert}
% \newcommand{\set}[1]{\left\{#1\right\}}
%\newcommand{\seq}[1]{\left<#1\right>}
 \newcommand{\norm}[1]{\left\Vert#1\right\Vert}
 \newcommand{\essnorm}[1]{\norm{#1}_{\ess}}

%%% NAMES
\newcommand{\Names}{{\mathcal N}}
\newcommand{\Channels}{{\sf X}}
\newcommand{\Variables}{{\mathcal V}}
\newcommand{\Enames}{{\mathcal E}}
\newcommand{\Nonterminals}{{\mathcal S}}
\newcommand{\Pnames}{{\mathcal P}}
\newcommand{\Dnames}{{\mathcal D}}
\newcommand{\Types}{{\mathcal T}}

\newcommand{\fcalc}{fusion calculus}
\newcommand{\xcalc}{${\mathfrak x}$-calculus}
\newcommand{\lcalc}{$\lambda$-calculus}
\newcommand{\pic}{$\pi$-calculus}
\newcommand{\hcalc}{highwire calculus}
\newcommand{\dcalc}{data calculus}
%XML should be all caps, not small caps. --cb
%\newcommand{\xml}{\textsc{xml}}
\newcommand{\xml}{XML} 

\newcommand{\papertitle}{From contract to service level agreement}
% use static date to preserve date of actual publication
 \newcommand{\paperversion}{Draft Version 0.1 - December 15, 2003}

\newenvironment{toc}
{
\begin{list}{}{
   \setlength{\leftmargin}{0.4in}
   \setlength{\rightmargin}{0.6in}
   \setlength{\parskip}{0pt}
 } \item }
{\end{list}}

\newenvironment{narrow}
{
\begin{list}{}{
   \setlength{\leftmargin}{0.4in}
   \setlength{\rightmargin}{0.6in}
 } \item }
{\end{list}}

%%% ----------------------------------------------------------------------
\begin{document}

%These margin values appear to be relative to the bigpage package settings. --cb
\setlength{\topmargin}{0in}
\setlength{\textheight}{8.5in}
\setlength{\parskip}{6pt}

\title{\huge{\papertitle}}

\author{ L.G. Meredith }

%\address{Microsoft Corporation, Redmond, Washington, USA}

%\email{gregmer@microsoft.com}

\thanks{This work was completed under the BigTop project.}

%\subjclass{Primary 47A15; Secondary 46A32, 47D20}

\keywords{concurrency, distribution, message-passing, process calculus, service-oriented architecture, quality of service, service level agreement}

%\date{April 6, 2002.}

\dedicatory{}

%\commby{Daniel J. Rudolph}

%%% ----------------------------------------------------------------------

\begin{abstract}
\normalsize{ 

In a previous paper we described how behavioral types can form the
basis of a notion of contract for a service-oriented world like the
Internet. But, that paper stopped short of addressing the kinds of
information that would make a contract a serious candidate for the
nomination service level agreement, as that term is understood in the
network services world. In this paper we show how the notion of
contract smoothly extends to include not only qualitative but
\emph{quantitative} behavioral information. Specifically, we
illustrate how various notions of quality of service, including rates,
capacity and uptime can be added to the notion of contract that is
based on the behavioral type. This makes the notion of contract put
forward here among the first proposals for a \emph{fully-mechanized}
notion of service level agreement.

It turns out that the mobile process algebra framework, where services
are processes, and contracts are behavioral types, greatly facilitates
the addition of this information. We make particular use of two
recently developed techniques for mobile process algebras: stochastic
transitions and enhanced operational semantics. Using these techniques
we illustrate decision procedures for verifying that the deployment of
a service implementation will meet the contractually advertised
quantitative promises.

}
\end{abstract}

\noindent
{\large \textbf{
BillG Think Week}}\\
\rule{6.25in}{0.75pt}\\\\\\

%%% ----------------------------------------------------------------------
\maketitle
%%% ----------------------------------------------------------------------

\begin{center}
\paperversion\\
\end{center}

\begin{toc}
\tableofcontents
\end{toc}

\newpage
% ------------------------------------------------------------------------

\section{Introduction}

If the Internet demands service-oriented architectures of its
entrepreneurial developers, then this new class of service providers
should take a page from the original network service providers'
handbook and consider the service level agreement.

The paper is structured as follows. First, we briefly review, by
example, how the service-oriented approach is developed in the process
algebra setting, introducing both a service and its contract to serve
as a running example throughout the paper. Next, we discuss the
different kinds of quantitative information that can be added to the
contract, rates, capacity and uptime and illustrate in terms of our
running example how this information is added to the contract. Then,
we discuss the mechanisms for checking implementation against
advertised performance for each of the different kinds of quantitative
information. Finally, we conclude with discussion of how this work may
be extended and some of the aspects of service management, e.g.,
service discovery may be impacted.

%%% ----------------------------------------------------------------------

\section{Service oriented-architectures in the $\pi$-calculus by example}

We start with the description of a generic service. This service
accepts a request for a session on a well-known port $www$, where service
request includes a reply port. On the reply port it sends four ports
private to this session with the client:

\begin{description}

\item[$init$] one to receive a request to initialize service;

\item[$worklo$] one to receive a request to do some low-priority work; 

\item[$workhi$] one to receive a request to do some high-priority work; and,

\item[$fin$] one to receive a request to end the session.

\end{description}

The implementation of this service in the $\pi$-calculus is

\begin{eqnarray*}
    \id{!}www\id{?}(r) \id{.} \id{[} init,worklo,workhi,fin \id{]} r \id{!}\id{(}init,worklo,workhi,fin\id{)} & & \\
	\juxtap \id{[} t \id{]} ( \id{!}t \id{?} \id{(}\id{)} \id{.} ( worklo \id{?} \id{(}\id{)} \id{.} t \id{!} \id{(}\id{)} \altern workhi \id{?} \id{(}\id{)} \id{.} t \id{!} \id{(}\id{)} \altern fin \id{?} \id{(}\id{)} \id{.} t \id{!} \id{(}\id{)} ) \juxtap init \id{?} \id{(}\id{)} \id{.} t \id{!} \id{(}\id{)})
\end{eqnarray*}

\section{Behavioral types as contracts}

\section{Adding quantitative information to contracts}

\subsection{Rates}

What is meant by rates?

Priami, et al, have addressed, on the one hand, stochastic transitions
for the $\pi$-calculus, and performance prediction for the same
calculus on the other. The former was introduced to handle an entirely
different class of applications -- modeling biology with
$\pi$-calculus. The presentation of the latter technique considered
the need to introduce rates, explicitly, an onerous burden on the
developer and developed a crucial idea, enhanced operational
semantics, in order to get around this.

But, having both, explicit statement of rates on the one hand, and
calculation of rates on the other represents a significant
opportunity. We add rates to the abstraction of the service embodied
in the behavioral type, i.e. we add an explicit statement of rates to
the contract, and calculate rates from the process constituting the
implementation of the service. Then we compare.

\subsection{Capacity}

\subsection{Uptime}

\section{Checking quantitative promises against deployment}

\subsection{Enhanced operational semantics}

\subsection{Cost functions and performance prediction}

\subsection{Shannon information of a port}

\subsection{Uptime and transitions to error}

\section{Conclusions}

% ------------------------------------------------------------------------
%GATHER{Xbib.bib}   % For Gather Purpose Only
%GATHER{Paper.bbl}  % For Gather Purpose Only
\bibliographystyle{amsplain}
\bibliography{altofilo}

% ------------------------------------------------------------------------

\section{Appendix A: $\pi$-calculus}

\begin{grammar}
\mbox{$\pi$-calculus}		& \category{process}		& \bc	& \category{channel}\id{!}\category{datum} \\
				&					& \bm	& \category{formation} \\
				&					& \bm	& \category{new} \\
				&					& \bm	& \category{replicate} \\									
				& \category{formation}		& \bc	& \category{juxtaposition} \\
				&					& \bm & \category{concatenation} \concat \category{process} \\
				&					& \bm & \category{alternation} \\
				& \category{juxtaposition}	& \bc	& \id{[} ( \category{process} ( \id{,} \category{process} )^* )? \id{]} \\
				&					& \bm & \category{juxtaposition} \juxtap \category{juxtaposition} \\
				& \category{concatenation}	& \bc	& \id{(} ( \category{io} ( \id{,} \category{io} )^* )? \id{)} \\
				&					& \bm & \category{concatenation} \concat \category{concatenation} \\
				& \category{alternation}	& \bc	& \id{\{} ( \category{io}\concat\category{process} ( {} \id{,} {\category{io}\concat\category{process}} )^* )? \id{\}} \\
				&					& \bm & \category{alternation} \altern \category{alternation} \\
				& \category{io}			& \bc	& \category{channel}\id{?}\category{datum} \\
				& \category{new}			& \bc	& \category{channels}\;\;\category{process} \\
	                        & \category{channels}		& \bc	& \id{[} \category{channel} ( \juxtap \category{channel} )^* \id{]} \\
				&					& \bm & \category{channels} \juxtap \category{channels} \\
				& \category{channel}		& \bc	& \category{name} \\
				& \category{replicate}		& \bc	& \id{!}\category{process} \\
				& \category{datum}		& \bc	& \id{(} ( \category{channel} ( \id{,} \category{channel} )^* )? \id{)}
\end{grammar}

We let $ar(\category{datum})$ denote the length of $\category{datum}$;
thus, if $\category{datum} = \id{(} \category{channel}_0 \id{,}
\cdots \id{,} \category{channel}_{n-1} \id{)} $ then $ar(
\category{datum} ) = n$. Moreover, we let $\category{datum}[i]$ denote the $i$th element of $\category{datum}$. 

As is standard, we use $\sigma$ to range over substitutions from $\category{channel}$ to $\category{channel}$.

\subsection{Operational Semantics}
The {\em structural congruence} of processes, noted $\congp$, is the least congruence that satisfies the monoid laws for $\concat$ with identity $\id{()}$, the abelian monoid laws for $\juxtap$ with identity $\id{[]}$, and $\altern$ with identity $\id{\{\}}$,
and the following additional laws:
\begin{eqnarray*}
\id{()}												& \congp	& \id{[]} \\
\id{[]}												& \congp	& \id{\{\}} \\
\id{[}\category{channel}_0\id{]} \juxtap \id{[}\category{channel}_0\id{]} \juxtap \category{channels}_0\;\;\category{process}_0	& \congp	& \id{[}\category{channel}_0\id{]} \juxtap \category{channels}_0\;\;\category{process}_0 \\
\id{[}\category{channel}_0\id{]}\category{process}_0 \juxtap \category{process}_1									& \congp	& \id{[}\category{channel}_0\id{]}\id{(}\category{process}_0 \juxtap \category{process}_1 \id{)} \\
	& & \mbox{ where } \category{channel}_0\not\in fn(\category{process}_1) \\
\Pi_{i=1}^m{\category{juxtaposition}_i} \juxtap \Pi_{i=m+1}^n{\category{juxtaposition}_i} & \congp & \Pi_{i=1}^n{\category{juxtaposition}_i} \\
\bigodot_{i=1}^m{\category{concatenation}_i} \concat \bigodot_{i=m+1}^n{\category{concatenation}_i} \concat \category{process}_0 & \congp & \bigodot_{i=1}^n{\category{concatenation}_i} \concat \category{process}_0 \\
\Sigma_{i=1}^m{\category{alternation}_i} \altern \Sigma_{i=m+1}^n{\category{alternation}_i} & \congp & \Sigma_{i=1}^n{\category{alternation}_i} \\
\id{!}\category{process}_0	& \congp & \category{process}_0\juxtap\id{!}\category{process}_0
\end{eqnarray*}
\\
The reduction rule for $\pi$-calculus is
\infrule[Comm]
{ ar( \category{datum}_0 ) = ar( \category{datum}_1 ), \; \forall i,0 \leq i \leq ar( \category{datum}_1 ). \sigma( \category{datum}_1[i] ) = \category{datum}_0[i] }
{{{\category{channel}_c\id{!}\category{datum}_0}} \juxtap {\id{(}{\category{channel}_c\id{?}{\category{datum}_1}\id{)} \concat \category{process}_0}} \\
\longrightarrow \id{[} \sigma(\category{process}_0)\id{]}}

In addition, we have the following context rules:

\infrule[Par]{\category{process}_0 \longrightarrow \category{process}_1}{{\id{[}\category{process}_0 \juxtap \category{process}_2 \id{]}} \longrightarrow {\id{[}\category{process}_1 \juxtap \category{process}_2 \id{]}}}

\infrule[New]{\category{process}_0 \longrightarrow \category{process}_1}{\id{[}\category{channel}_0\id{]}\category{process}_0 \longrightarrow \id{[}\category{channel}_0\id{]}\category{process}_1}

\infrule[Equiv]{{\category{process}_2 \congp \category{process}_0} \andalso {\category{process}_0 \longrightarrow \category{process}_1} \andalso {\category{process}_1 \congp \category{process}_3}}{\category{process}_2 \longrightarrow \category{process}_3}


% ------------------------------------------------------------------------

\end{document}
% ------------------------------------------------------------------------
