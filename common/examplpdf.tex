\documentclass{entcs} \usepackage{entcsmacro}
\input pdfcolor.tex 

% The following is enclosed to allow easy detection of differences in
% ascii coding.
% Upper-case    A B C D E F G H I J K L M N O P Q R S T U V W X Y Z
% Lower-case    a b c d e f g h i j k l m n o p q r s t u v w x y z
% Digits        0 1 2 3 4 5 6 7 8 9
% Exclamation   !           Double quote "          Hash (number) #
% Dollar        $           Percent      %          Ampersand     &
% Acute accent  '           Left paren   (          Right paren   )
% Asterisk      *           Plus         +          Comma         ,
% Minus         -           Point        .          Solidus       /
% Colon         :           Semicolon    ;          Less than     <
% Equals        =3D           Greater than >          Question mark ?
% At            @           Left bracket [          Backslash     \
% Right bracket ]           Circumflex   ^          Underscore    _
% Grave accent  `           Left brace   {          Vertical bar  |
% Right brace   }           Tilde        ~

% A couple of exemplary definitions:

\newcommand{\Nat}{{\mathbb N}}
\newcommand{\Real}{{\mathbb R}}
\def\lastname{Please list Your Lastname Here}
\begin{document}
\begin{frontmatter}
  \title{An Example Paper} \author{My
    Name\thanksref{ALL}\thanksref{myemail}}
  \address{My Department\\ My University\\
    My City, My Country} \author{My Co-author\thanksref{coemail}}
  \address{My Co-author's Department\\My Co-author's University\\
    My Co-author's City, My Co-author's Country} \thanks[ALL]{Thanks
    to everyone who should be thanked} \thanks[myemail]{Email:
    \href{mailto:myuserid@mydept.myinst.myedu} {\texttt{\normalshape
        myuserid@mydept.myinst.myedu}}} \thanks[coemail]{Email:
    \href{mailto:couserid@codept.coinst.coedu} {\texttt{\normalshape
        couserid@codept.coinst.coedu}}}
\begin{abstract} 
  This is a short example to show the basics of using the ENTCS style
  macro files.  Ample examples of how files should look may be found
  among the published volumes of the series at the ENTCS Home Page
  \texttt{http://www.elsevier.nl/locate/entcs}.
\end{abstract}
\begin{keyword}
  Please list keywords from your paper here, separated by commas.
\end{keyword}
\end{frontmatter}
\section{Introduction}\label{intro}
This short note provides a guide to using the ENTCS macro package for
preparing papers for publication in your conference
\emph{Proceedings}. The \emph{Proceedings} may be printed and hard
copies distributed to participants at the meeting; this is an option
to Conference Organizers may choose to exercise.  The
\emph{Proceedings} also will be par of a volume in the series
\emph{Electronic Notes in Theoretical Computer Science} (ENTCS), which
is published under the auspices of Elsevier Science B.~V., the
publishers of \emph{Theoretical Computer Science}. It's home page is
\href{http://www.elsevier.nl/locate/entcs}
{\texttt{http://www.elsevier.nl/locate/entcs}}

The ENTCS macro package consists of two files:
\begin{description}
\item[\texttt{entcs.cls},] the basic style file, and
\item[\texttt{entcsmacro.sty},] a macro file containing the
  definitions of some of the theorem-like environments and a few other
  tidbits.
\end{description}
The formatting these style files impose should \emph{not} be altered
-- the reason for using them is to attain a uniform format for all
papers in the \emph{Proceedings} of which your paper is a part.

Additional macro files can be added using \verb+\usepackage{...}+.
The file \texttt{entcs\-macro.sty} \emph{must} be included in the
list, as is done at the start of the source file for this paper.

The ENTCS package requires a relatively up-to-date \LaTeX\ system in
order to be successfully used. This is reflected in two other packages
that are called by entcs.cls, which must be available on your machine.
These are:
\begin{itemize}
\item The \texttt{hyperref} package. This package allows the use of
  hyperlinks in files prepared using \LaTeX 2e, one of the main
  features of Adobe's Acrobat$^{\tiny
  \copyright}$ Reader software. Be sure that you 
  have at least version 6.69d of this package. 
\item The \texttt{ifpdf} package. This is used by hyperref to
  differentiate between the use of pdf\LaTeX\ and \LaTeX 2e, followed
  by dvips and then ps2pdf. 
\end{itemize}

The file \texttt{instraut.dvi} contains information about the use of
\LaTeX to prepare files for online publication by Elsevier. This file
refers to the older version of \LaTeX\ that is no longer suppported,
and that is inadequate for preparing \texttt{.pdf} files for online
publication. Reading this file should answer most of the basic
questions about \LaTeX\ that might arise.

\section{Frontmatter}
The biggest difference between a ``usual'' \LaTeX\ style such as
\texttt{article.sty} and the ENTCS package is that the ENTCS macro
package requires the title, author's name or names, abstract, keywords
and ``thanks'' all to be included within the \texttt{frontmatter}
environment.  At the beginning of the source file for this paper,
you'll notice this.  Also, you'll notice that the usual
\verb+\maketitle+ is absent; it no longer is needed.  The ENTCS style
package automatically generates the title, author's name and address,
and related material at the beginning of the paper. Note also that
hyperref has been disabled in this part of the entcs.cls file, so
references to footnotes aren't linked to the appropriate footenotes or
addresses. This is an old problem with \LaTeX, involving the fact that
the references within the frontmatter aren't passed cleanly to the
linking software. 

For those who have used the ENTCS package before, the one new thing to
note is the inclusion of \emph{Keywords}; these are now required by
Elsevier -- they're also required by ACM's \emph{Computing Reviews}
which reviews ENTCS publications.

The ENTCS macro package provides two alternatives to listing authors
names and addresses.  These are described in detail in the file
\texttt{instraut.dvi}. Basically, listing each author and his or her
address in turn, is the simplest method.  But, if there are several
authors and two or more share the same address (but not all authors
are at this address), then the method of listing authors first, and
then the addresses, and of referencing addresses to authors should be
used.

Also, notice that acknowledgment of support (the contents of
\verb+\thanks+) should be done by a separate listing of
\verb+\thanks[NSF]{To the NSF}+ with the optional argument --
\verb+[NSF]+ -- being used for \verb+\thanksref+ which is attached to
those authors acknowledging such support.  It is important that the
\verb+\thanks+ not be included within the scope of \verb+\author{}+ or
of \verb+\title{}+, but it must be within the scope of the environment
\texttt{frontmatter}.

More details about added terms such as \verb+\collab+ can be
found in \texttt{inst.dvi}, if they are needed.

Also, notice that the command \verb+\lastname{My Lastname}+ has been
included \emph{before} the \texttt{frontmatter} begins.  This command
should contain the last names of the authors of the paper.  If there
are no more than three authors, then they should be listed with the
word ``and'' between the last two; if more than three authors
collaborated on the paper, then the first author only should be
listed, together with \verb+\emph{et al}+.  This command creates the
headline for each page after page 1.

Finally, please be sure to include an abstract for your paper.

\section{Sectioning and Environments}
Since ENTCS is published through the auspices of Elsevier Science
B.~V., their style files have been used to create the ENTCS macro
package.  Here's a proof that this package is not much different than
most of the ones one encounters:

\begin{definition}
  A file is \emph{derived} from another if it is obtained with only a
  few modifications from the original file.
\end{definition}

\begin{theorem}
  The file \texttt{\normalshape entcs.cls} is derived from
  \texttt{\normalshape elsart.sty}.
\end{theorem}
\begin{proof}
  This is clear from the similarity of the output to the output from
  Elsevier's style files.
\end{proof}

If one wants to start a proof with a descriptive word, such as
``sketch'', then one can use the \verb+\begin{proof*}...\end{proof*}+
environment, as in

\begin{proof*}{Proof (Sketch)}
  This can be derived from simple observations.
\end{proof*}

The main differences between the file \texttt{entcs.cls} and the
\texttt{elsartr.cls} file used by Elsevier are the more precise format
we use -- Elsevier's generic files are meant for preliminary editing,
and more precise formatting is imposed using a macro file designed for
the specific Elsevier journal in which the paper will eventually
appear. The \texttt{entcs.cls} and \texttt{entcsmacro.sty} files
format papers uniformly so that they all are easily recognizable as
being from the series \emph{Electronic Notes in Theoretical Computer
  Science}.

All of the usual features of \LaTeX\ are available with these
style files -- it is only the formatting that has been rigorously
defined. Thus, one has available the sectioning commands
\verb+\section,\subsection, \paragraph+ and \verb+\subparagraph.+
The numbering scheme used is one under which Theorem 1.2.3 is the
third numbered item in second subsection of the first section of
the paper.  In order to facilitate cross-references, all of the
named environments given below are numbered, and all use the same
number scheme.

The file \texttt{entcsmacro.sty} contains additional information that
is needed to typeset a paper.  It also has the definitions of the
$\cal AMS$ \texttt{euler} and \texttt{blackboard bold} fonts builtin.
If you want to use symbols for the natural numbers, the reals, etc.,
then we prefer that you use the blackboard bold fonts, and not plain
bold fonts. This is accomplished by using the \verb+\mathbb+ font, as
in $\Nat$ or $\Real$.

The names of theorem-like environments are provided in
\texttt{entcsmacro.sty}.  With the exception of the environment
Algorithm, the names of all of these are full name, rather than a
shortened version. The environments provided and their names are
\begin{itemize}
\item \verb+\begin{theorem} ... \end{theorem}+ for Theorems,
\item \verb+\begin{lemma} ... \end{lemma}+ for Lemmas,
\item \verb+\begin{corollary} ... \end{corollary}+ for Corollaries,
\item \verb+\begin{proposition} ... \end{proposition}+ for
  Propositions,
\item \verb+\begin{criterion} ... \end{criterion}+ for Criteria,
\item \verb+\begin{alg} ... \end{alg}+ for Algorithms,
\item \verb+\begin{definition} ... \end{definition}+ for Definitions,
\item \verb+\begin{conjecture} ... \end{conjecture}+ for Conjectures,
\item \verb+\begin{example} ... \end{example}+ for Examples,
\item \verb+\begin{problem} ... \end{problem}+ for Problems,
\item \verb+\begin{remark} ... \end{remark}+ for Remarks,
\item \verb+\begin{note} ... \end{note}+ for Notes,
\item \verb+\begin{claim} ... \end{claim}+ for Claims,
\item \verb+\begin{summary} ... \end{summary}+ for Summary,
\item \verb+\begin{case} ... \end{case}+ for Cases, and
\item \verb+\begin{ack} ... \end{ack}+ for Acknowledgements.
\end{itemize}

For example,

\begin{algorithm}[h]
\begin{alg}
  Step 1:  Write the paper\\
  Step 2: Format it with the ENTCS macro package\\
  Step 3:  Ship the whole thing to the Guest Editors\\
\end{alg}
\end{algorithm}

\section{References and Cross-references}
All the cross-referencing facilities of \LaTeX\ are supported, so one
can use \verb+\ref{}+ and \verb+\cite{}+ for cross-references within
the paper and for references to bibliographic items.  As is done in
this note, the \textbf{References} section~\ref{bibliography} can be
composed with \verb+\begin{thebibliography}...\end{thebibliography}+.
Alternatively, Bib\TeX\ can be used to compile the bibliography.
Whichever one is used, the references are to be numbered
consecutively, rather than by author-defined acronyms.  Of course you
can use your own acronyms for easy reference to each of the items in
the bibliography, as has been done with the listing for this short
note.

However, note that the references should \emph{not} be started with a
new \verb+\section+ command.

The package \texttt{hyperref} is automatically loaded by entcs.cls,
and this makes all the cross-references within the document ``active''
when the pdf file of the paper is viewed with Adobe's Acrobat$^{\tiny
  \copyright}$ Reader.  The format for including a link is simple:
simply insert \verb+\href{URL}+ \verb+{text}+ where \emph{URL} is the
URL to which you want the link to point, and \emph{text} is the text
you want to be highlighted, which when clicked upon will bring up the
desired web page.

\subsection{Particulars about {\normalshape \texttt{.pdf} files}}
We now require that \texttt{.pdf} files be provided for publication
online. A \texttt{.pdf} file is viewable by Adobe's Acrobat$^{\tiny
  \copyright}$ viewer, which can be configured to load automatically
within a browser. Viewing a properly formatted \texttt{.pdf} file with
Acrobat$^{\tiny \copyright}$ allows the cross-references and links to
URLs to be active. In fact, Elsevier utilizes \texttt{.pdf} files in
order to take better advantage of the web's capabilities.  

But one point we want to emphasize is that you should be sure to use
Type 1 fonts when you typeset your \LaTeX\ source file. These fonts
are scalable, meaning that they carry information that allows the
devise viewing the final output to scale the fonts to suit the viewer
being used -- from an onscreen viewer such as Adobe's Acrobat$^{\tiny
  \copyright}$ Reader, to printing the file on a printer. You can tell
if you have used the right fonts by viewing the final output on your
machine. It the fonts look grainy, then you have not used Type 1
fonts. They can be located at the CTAN archive
\href{http://www.ctan.org}{\tt http://www.ctan.org} -- they are public
domain fonts, and don't cost anything to add them to your system.

Assuming you have Type 1 fonts available, then there are there methods
for producing \texttt{.pdf} files.

\paragraph{Using \texttt{dvips} and \texttt{ps2pdf}}
We list this option first since it appears to be the most reliable and
the easiest to use, especially if you include embedded PostScript
graphics (\texttt{.eps} files) in your source file. Simply run \LaTeX
2e on your source file, then apply \texttt{dvips} to produce a
PostScript file, and finally apply \texttt{ps2pdf} to obtain a
\texttt{.pdf} file. 

\paragraph{The \texttt{DVIPDFM} utility}
Another easy method for producing acceptable \texttt{.pdf} files is
via the utility \texttt{dvipdfm}. This utility is included in
distributions of Mik\TeX, which runs on Windows machines, but it
probably needs to be added to your te\TeX\ distribution, if you are
running \LaTeX\ on a UNIX machine. The utility and precise information
about installing it on your system can be found at the web page
\href{http://gaspra.kettering.edu/dvipdfm/}{\tt
  http://gaspra.kettering.edu/dvipdfm/}. In essence, this utility
converts a \texttt{.dvi} file into a \texttt{.pdf} file. So, one can
first prepare the \texttt{.dvi} file using \LaTeX, and then apply the
utility \texttt{dvipdfm} to produce the needed \texttt{.pdf}
file.\footnote{ \emph{Beware}! The utility \texttt{dvipdf} does
  \emph{not} produce acceptable \texttt{.pdf} files, and should not be
  used. Only \texttt{dvipdfm} should be used to produce \texttt{.pdf}
  files.} This utility makes inclusion of graphics particularly simple
-- those that are included in the \LaTeX\ source file are simply
converted to the \texttt{.pdf} format. As we note below, things are
not so simple with the second alternative, which is to use pdf\LaTeX.

\paragraph{pdf\LaTeX}
An alternative to the first possibilities to produce \texttt{.pdf}
files is to process the source file with pdf\LaTeX. This format is
available from the standard CTAN sites \href{http://www.ctan.org}{\tt
  http://www.ctan.org}. It appears that pdf\LaTeX\ and
\texttt{hyperref} have some problems when used together. It is
necessary to use pdf\LaTeX\ version 14d or later in order to minimize
these issues. If your system has an earlier version (most te\TeX\ 
distributions have version 13d), then you can update your system by
retrieving the latest version of pdf\LaTeX\ from
\href{ftp://ftp.cstug.cz/pub/tex/local/cstug/thanh/pdftex/}{\tt
  ftp://ftp.cstug.cz/pub/tex/local/cstug/thanh/pdftex/}. Even if the
recent versions are used, pdf\LaTeX\ has the same dealing with
references embedded with the \texttt{frontmatter} section described
above for \LaTeX.

But there is one aspect of pdf\LaTeX\ that creates problems. Many
authors include EPS\footnote{EPS stands for \emph{embedded
    PostScript}, which affords a mechanism for including pre-prepared
  PostScript files within a \LaTeX\ document.} files within their
papers. While this is fairly straightforward with \LaTeX, there are a
couple of points to note when attempting this with pdf\LaTeX.

To include a PostScript image in a \texttt{.pdf} file produced
with pdf\LaTeX, you first have to convert the image to a
\texttt{.pdf} file, and then it can be included using the same
command sequence as above. The conversion can be accomplished
most easily using Ghostscript; you can simply view the file in
Ghostview and then print the image to a \texttt{.pdf} file using
the \verb+pdfwriter+ option within Ghostview. The result for a
standard chess board that is part of the Ghostview distribution
is the following image:

\centerline{\pdfximage width 4in height 4in
  {chess.pdf}\pdfrefximage\pdflastximage}
Here as well is a copy of a color image. While pdf\LaTeX\ can
handle image files in other formats, \LaTeX\ can only handle
\texttt{.eps} images reliably.

\centerline{\pdfximage height 3.5truein width 3truein
  {tigre.pdf}\pdfrefximage\pdflastximage}\ \medbreak

It also should be noted that you need to have two separate source
files -- one for \LaTeX\ and one for pdf\LaTeX\ -- \emph{only} in case
you wish to insert graphics images in your final paper. If your paper
does not include such images, then the same source file can be
formatted by both \LaTeX\ and by pdf\LaTeX.

\paragraph{Using ENTCS Macros with Mac OS X}
Of course, if your file doesn't require \texttt{.eps} or other
PostScript files, then you can create the required \texttt{.pdf} file
using any of the standard \TeX\ implementations for the Macintosh. If
you need to include PostScript files, and if you are using \TeX Shop,
then you can specify to use dvips and ghostview in processing your
file, and then you can apply \texttt{ps2pdf} to create the needed
\texttt{.pdf} file. Alternatibely, the Mac OS X operating system is
based on UNIX, so it supports the use of te\TeX\ as described above.

\section{Summary}  The ENTCS macro package is relatively easy to use
and provides a uniform layout for all the papers that appear in ENTCS.

\begin{problem}
  Finish your paper and get it to your Program Chairman on time!
\end{problem}

When you have finished preparing your paper, send a copy of the
\emph{source file}, together with any macro files that are needed to
your Program Chairman.  If the files are extensive, you can place
copies in the \texttt{pub/incoming} sub-directory of the ftp directory
on the machine indicated by your Program Chairman using anonymous ftp.
If you do this, please send me email to alert me that the file(s) are
here.

\paragraph{Assigning Volume / Issue Numbers}
One additional point worth mentioning is that ENTCS is moving to
ScienceDirect, Elsevier's main platform for publishing electronic
series, Because ScienceDirect must publish entire volumes at the same
time, we have changed the procedure for preparing final versions so
that volume numbers will not be assigned until the final versions are
ready. Guest Editors will now have to receive the final version of all
papers in their \emph{Proceedings} before a volume and issue number
will be assigned for the \emph{Proceedings}. Even with the move to
ScienceDirect, the reference scheme already used for publications in
ENTCS -- \texttt{http://www.elsevier/nl/locate/entcs/}
\texttt{NNnn.html} remains the valid way to cite papers published in
ENTCS, where \texttt{NN} denotes the number of the volume, and
\texttt{nn} denotes the issue number.  Publications consisting of an
entire volume should use \texttt{1} as the issue number.

\paragraph{Copyright Transfer Forms}
One result of the move to ScienceDirect is that the corresponding
author of each paper published in ENTCS must submit a signed Copyright
Transfer Form to Elsevier in order for their paper to be published. A
copy of this form will be sent to each author by the Guest Editors of
each volume. Details about this agreement specifying the rights of the
authors and the rights of Elsevier are available at
\href{http://authors.elsevier.com/PublisherInfoDetail.html?dc=AGI}{Elsevier's
  Author Gateway}. 

\paragraph{Publication of Final Versions}
Because ScienceDirect cannot easily accommodate changes to published
material, the Proceedings in its entirety must be ready before it can
be published. This is one reason why the volume and issue number is
not assigned until the final versions of all papers have been sent to
the Guest Editors for final processing. 

\section{Bibliographical references}\label{references}
ENTCS employs the \texttt{plain} style of bibliographic references in
which references are listed in alphabetical order, according the the
first author's last name, and are sequentially numbered. Please
utilize this style. We have a Bib\TeX\ style file, for those who wish
to use it. It is the file \texttt{entcs.bst} which in included in this
package. The basic rules we have employed are the following:
\begin{itemize}
\item Authors' names should be listed in alphabetical order, with the
  first author's last name being the first listing, followed by the
  author's initials or first name, and with the other authors names
  listed as \emph{first name, last name}.
\item Titles of articles in journals should be in \emph{emphasized}
  type.
\item Titles of books, monographs, etc.\ should be in quotations.
\item Journal names should be in plain roman type.
\item Journal volume numbers should be in boldface type, with the year
  of publication immediately following in roman type, and enclosed in
  parentheses.
\item References to URLs on the net should be ``active'' and the URL
  itself should be in typewriter font.
\item Articles should include page numbers.
\end{itemize}
The criteria are illustrated in the following.

\begin{thebibliography}{10}\label{bibliography}
\bibitem{cy} Civin, P., and B. Yood, \emph{Involutions on Banach
    algebras}, Pacific J. Math. \textbf{9} (1959), 415--436.
  
\bibitem{cp} Clifford, A. H., and G. B. Preston, ``The Algebraic
  Theory of Semigroups,'' Math. Surveys \textbf{7}, Amer. Math. Soc.,
  Providence, R.I., 1961.
  
\bibitem{f} Freyd, Peter, Peter O'Hearn, John Power, Robert Tennent
  and Makoto Takeyama, \emph{Bireflectivity}, Electronic Notes in
  Theoretical Computer Science {\bf 1} (1995), URL:
  \href{http://www.elsevier.nl/locate/entcs/volume1.html}
  {\texttt{http://www.elsevier.nl/locate/entcs/volume1.html}}.
  
\bibitem{em2} Easdown, D., and W. D. Munn, \emph{Trace functions on
    inverse semigroup algebras}, U. of Glasgow, Dept. of Math.,
  preprint 93/52.

\bibitem{r} Roscoe, A. W., ``The Theory and Practice of Concurrency,''
  Prentice Hall Series in Computer Science, Prentice Hall Publishers,
  London, New York (1198), 565pp. With associated web site\\  
  \href{http://www.comlab.ox.ac.uk/oucl/publications/books/concurrency/}
  {\texttt{http://www.comlab.ox.ac.uk/oucl/publications/books/concurrency/}}.
  
\bibitem{s} Shehadah, A. A., ``Embedding theorems for semigroups with
  involution, `` Ph.D.  thesis, Purdue University, Indiana, 1982.
  
\bibitem{w} Weyl, H., ``The Classical Groups,'' 2nd Ed., Princeton U.
  Press, Princeton, N.J., 1946.

\end{thebibliography}

\end{document}
