\section{Unlikely characters: spatial logic for
  knots}\label{sub:characteristic_formulae} % (fold)

Associated to the mobile process calculi are a family of logics known
as the Hennessy-Milner logics. These logics typically enjoy a
semantics interpreting formulae as sets of processes that when
factored through the encoding outlined above allows an identification
of classes of knots with logical formulae. In the context of this
encoding the sub-family known as the spatial logics \cite{CairesC03}
\cite{CairesC04} \cite{Caires04} are of particular interest providing
several important features for expressing and reasoning about
properties (i.e. classes) of knots. We hint here at how this may be done.

%\begin{description}
%\item [structural connectives] 
\subsubsection{Structural connectives} The spatial logics enjoy
structural connectives corresponding, at the logical level, to the
parallel composition ($P | Q$) and new name ($(\nu \; x)P$)
connectives for processes. As illustrated in the examples below, these
connectives are extremely expressive given the shape of our encoding.
%\item [decideable satisfaction]

\subsubsection{Decideable satisfaction}
In \cite{Caires04} the satisfaction relation is shown to be decideable
for a rich class of processes. It further turns out that the image of
the our encoding is a proper subset of that class. This result
provides the basis for an algorithm by which to search for knots
enjoying a given property.
%\item [characteristic formulae]

\subsubsection{Characteristic formulae}
In the same paper \cite{Caires04} , Caires presents a means of calculating
characteristic formulae, selecting equivalence classes of processes
up to a pre--specified depth limit on the support set of names. Composed with our
encoding, this characteristic formula can be used to select
characteristic formulae for knots.
%\end{description}

\subsubsection{Spatial logic formulae}

The grammar below (segmented for comprehension) summarizes the syntax
of spatial logic formulae. We employ illustrative examples in the
sequel to provide an intuitive understanding of their meaning
referring the reader to \cite{Caires04} for a more detailed explication
of the semantics.

\begin{mathpar}
  \inferrule* [lab=boolean] {} {{A,B} \bc T \;|\; \neg A \;|\; A \wedge B \;|\; \eta = \eta'}
  \and
  \inferrule* [lab=spatial] {} {|\; \pzero \;|\; A | B \;|\; x \text{\textregistered} A \;|\; \forall x . A \;|\;  H x . A}
  \and
  \inferrule* [lab=behavioral] {} {|\; \alpha . A}
  \and 
  \inferrule* [lab=recursion] {} {|\; X(\vec{u}) \;|\; \mu X(\vec{u}) . A}
  \and
  \inferrule* [lab=action] {} {\alpha \bc \langle x?(\vec{y}) \rangle \;|\; \langle x!(\vec{y}) \rangle \;|\; \langle \tau \rangle}
  \and 
  \inferrule* [lab=name] {} {\eta \bc x \;|\; \tau}
\end{mathpar} 

% subsection characteristic_formulae (end)   	 

\subsection{Example formulae}\label{sub:example_formulae_} % (fold)

\subsubsection{Crossing as formula.}
% 
% \begin{align*}
%   \frac{d}{dx} \sin x &= \cos x 
%   & \frac{d}{dx} e^x &= e^x \\
%   \frac{d}{dx} \cos x &= - \sin x 
%   & \frac{d}{dx} \log x &= \frac{1}{x} \\
% \end{align*} 

\begin{align*}
 \mu C(x_{0},x_{1},y_{0},y_{1},u).&(\langle x_{0}?(z) \rangle(\langle u! \rangle\langle y_{1}!z \rangle C(x_{0},x_{1},y_{0},y_{1},u)) & \\
  & \wedge \langle y_{1}?(z) \rangle (\langle u! \rangle \langle x_{0}!z \rangle C(x_{0},x_{1},y_{0},y_{1},u)) & \\
  & \wedge \langle x_{1}?(z) \rangle (\langle u? \rangle \langle y_{0}!z \rangle C(x_{0},x_{1},y_{0},y_{1},u)) & \\
  & \wedge \langle y_{0}?(z) \rangle (\langle u? \rangle \langle x_{1}!z \rangle C(x_{0},x_{1},y_{0},y_{1},u))) &
\end{align*}

The lexicographical similarity between the shape of this formulae and
the shape of definition of the process representing a crossing reveals
the intuitive meaning of this formulae. It describes the capabilities
of a process that has the right to represent a crossing. For example
it picks out processes that may perform an input on the port $x_0$ in
its initial menu of capabilities. What differentiates the formula
from the process, however, is that the crossing process is the
smallest candidate to satisfy the formula. Infinitely many other
processes -- with internal behavior hidden behind this interface, so
to speak -- also satisfy this formula. Even this simple formula,
then, can be seen to open a new view onto knots, providing a
computational interpretation of \emph{virtual} knots.

Note that this formula is derived by hand. A similar formula can be
derived by employing Caires' calculation of characteristic formula
\cite{Caires04} to the process representing a crossing. In light of
this discussion, we let
$\meaningof{C}_{\phi}(x0,x1,y0,y1,u)$ denote a formula specifying the
dynamics we wish to capture of a crossing. To guarantee we preserve
the shape of the interface and minimal semantics we demand that
$\meaningof{C}_{\phi}(x0,x1,y0,y1,u) \Rightarrow
\textbf{C}(x0,x1,y0,y1,u)$ where $\textbf{C}(x0,x1,y0,y1,u)$ denotes
the formula above.
                            
\subsubsection{Crossing number constraints.}
The moral content of the context lemma (Lemma \ref{context}) is that the notion of
``locality'' in the Reidemeister moves is effectively captured by the
parallel composition operator of the process calculus. This intuition
extends through the logic. Given a formula,
$\meaningof{C}_{\phi}(x0,x1,y0,y1,u)$, we can use the structural
connectives to specify constraints on crossing numbers, such as at
least $n$ crossings, or exactly $n$ crossings.
\begin{mathpar}
  \inferrule* [lab=at-least-n] {} { K^{\geq n}_{\phi}(\vec{xs},\vec{ys}) := \Pi_{i=0}^{n-1} Hu . \meaningof{C}_{\phi}(xs_i,ys_i,u) | T }
  \and 
  \inferrule* [lab=exactly-n] {} { K^{= n}_{\phi}(\vec{xs},\vec{ys}) := \Pi_{i=0}^{n-1} Hu . \meaningof{C}_{\phi}(xs_i,ys_i,u) | \neg (\forall x_0,y_0,x_1,y_1,u . \meaningof{C}_{\phi}(x_0,y_0,x_1,y_1,u) | T) }
\end{mathpar}

To round out this section, recall that the encoding of an $n$-crossing
knot decomposes into a parallel composition of $n$ \emph{copies} of a
crossing process together with a wiring harness. To specify different
knot classes with the same crossing number amounts to specifying
logical constraints on the wiring harness. In the interest of space,
we defer examples to a forthcoming paper. Suffice it to say that both
the conditions ``alternating knot'' and ``contains the tangle
corresponding to 5/3'' are expressible. For example, it is possible to
calculate the characteristic formula of a process corresponding to the
tangle 5/3 and conjoin it into the classifying formula via the
composition connective of the logic.

Finally, we wish to observe that it is entirely within reason to
contemplate a more domain-specific version of spatial logic tailored
to the shape of processes in the image of the encoding. Such a
domain-specific logic would have a better claim to the title formal
language of knot properties.

% subsection example_formulae_ (end)

% section knots_as_processes (end)