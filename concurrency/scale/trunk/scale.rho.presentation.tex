\subsection{The syntax and semantics of the notation system}\label{sub:the_syntax_and_semantics_of_the_notation_system} % (fold)

% [Ed. Note: stylistic alternative 1:]
% Having established an intuitive account of the target calculus via the
% encoding, we summarize its technical presentation below. These calculi
% are typically presented by giving a set of generators and relations on
% them. The grammar, below, describing term constructors, freely
% generates the set of processes $\mathcal{P}$ under the operations
% indicated.

% [Ed. Note: stylistic alternative 2:]

\section{A reflective calculus}

We observe that the calculus presented in the previous section is not
a closed theory, but rather a theory dependent upon some theory of
names. Operationally, one may think of the {\pic} as a procedure that
when handed a theory of names provides a theory of processes that
communicate over those names. In this section we present a closed
variation of the the {\pic} in which the \emph{codes} of processes
supply the set of names.

\subsubsection{Notation}

We let ${P, Q, R}$ range over processes and ${x, y, z}$ range over names.

\begin{mathpar}
  \inferrule* [lab=process] {} {{P, Q} \bc \pzero \\ \mbox{null process} \\\\ x(y).P \\ \mbox{input} \\\\ \lift{x}{P} \\ \mbox{lift} \\\\ \dropn{x} \\ \mbox{drop} \\\\ P|Q \\ \mbox{parallel}}
  \and \\
  \inferrule* [lab=name] {} {{x,y} \bc \quotep{P} \\ \mbox{quote}}
\end{mathpar}

\subsubsection{Quote}

Working in a bottom-up fashion, we begin with names. The technical
detail corresponding to the {\pic}' parametricity in a theory of
\textit{names} shows up in standard presentations in the grammar
describing terms of the language: there is no production for names;
names are taken to be terminals in the grammar. Our first point of
departure from a more standard presentation of an asynchronous mobile
process calculus is here. The grammar for the terms of the language
will include a production for names in the grammar. A name is a
\textit{quoted} process, $\quotep{P}$.

\subsubsection{Parallel}

This constructor is the usual parallel composition, denoting
concurrent execution of the composed processes.

\subsubsection{Lift and drop}
Despite the fact that names are built from (the codes of) processes,
we still maintain a careful disinction in kind between process and
name; thus, name construction is not process construction. So, if one
wants to be able to generate a name from a given process, there must
be a process constructor for a term that creates a name from a
process. This is the motivation for the production $\lift{x}{P}$,
dubbed here the \emph{lift} operator. The intuitive meaning of this
term is that the process $P$ will be packaged up as its code,
$\quotep{P}$, and ultimately made available as an output at the port
$x$.

A more formal motivation for the introduction of this operator will
become clear in the sequel. But, it will suffice to say now that
$\quotep{P}$ is impervious to substitution. In the $\rho$-calculus,
substitution does not affect the process body between quote marks. On
the other hand, $\lift{x}{P}$ is susceptible to substitution and as
such constitutes a dynamic form of quoting because the process body
ultimately quoted will be different depending on the context in which
the $\lift{x}{P}$ expression occurs.

Of course, when a name is a quoted process, it is very handy to have a
way of evaluating such an entity. Thus, the $\dropn{x}$ operator,
pronounced \textit{drop} $x$, (eventually) extracts the process from a
name. We say `eventually' because this extraction only happens when a
quoted process is substituted into this expression. A consequence of
this behavior is that $\dropn{x}$ is inert except under and input
prefix. One way of saying this is that if you want to get something
done, sometimes you need to drop a name, but it should be the name of
an agent you know.

\begin{remark} %{No $\nu$ is good news}
The lift operator turns out to play a role analogous to $(\nu \;
x)P$. As mentioned in the introduction, it is essential to the
computational completeness of the calculus, playing a key role in the
implementation of replication. It also provides an essential
ingredient in the compositional encoding of the $\nu$ operator.
\end{remark}

\begin{remark} %{Dropping replication}
It is well-known that replication is not required in a higher-order
process algebra \cite{SangiorgiWalker}. While our algebra is
\textit{not} higher-order in the traditional sense (there are not
formal process variables of a different type from names) it has all
the features of a higher-order process algebra. Thus, it turns out
that there is no need for a term for recursion. To illustrate this we
present below an encoding of $!P$ in this calculus. Intuitively, this
will amount to receiving a quoted form of a process, evaluating it,
while making the quoted form available again. The reader familiar with
the $\lambda$-calculus will note the formal similarity between the
crucial term in the encoding and the paradoxical combinator
\cite{Barendregt84}.
\end{remark}

\subsubsection{Input and output}

The input constructor is standard for an asynchronous name-passing
calculus. Input blocks its continuation from execution until it
receives a communication. Lift is a form of output which -- because
the calculus is asynchronous -- is allowed no continuation. It also
affords a convenient syntactic sugar, which we define here.

\begin{eqnarray}
	{x}\id{[}{y}\id{]} \triangleq \lift{x}{\dropn{y}} \nonumber
\end{eqnarray}

\subsubsection{The null process}

As we will see below, the null process has a more distinguished role
in this calculus. It provides the sole atom out of which all other
processes (and the names they use) arise much in the same way that the
number $0$ is the sole number out of which the natural numbers are
constructed; or the empty set is the sole set out of which all sets
are built in $ZF$-set theory \cite{Krivine-TheCurryHowardCorre}; or
the empty game is the sole game out of which all games are built in
Conway's theory of games and numbers \cite{ONAG}. This analogy to
these other theories draws attention, in our opinion, to the
foundational issues raised in the introduction regarding the design of
calculi of interaction.

\subsection{The name game}

Before presenting some of the more standard features of a mobile
process calculus, the calculation of free names, structural
equivalence, etc., we wish to consider some examples of processes and
names. In particular, if processes are built out of names, and names
are built out of processes, is it ever possible to get off the ground?
Fortunately, there is one process the construction of which involves
no names, the null process, $0$. Since we have at least one process,
we can construct at least one name, namely $\quotep{0}$ \footnote{pun
gratefully accepted ;-)}. Armed with one name we can now construct at
least two new processes that are evidently syntactically different
from the $0$, these are $\quotep{0}\id{[}{\quotep{0}}\id{]}$ and
$\quotep{0}\id{(}{\quotep{0}}\id{)} \concat 0$. As we might expect,
the intuitive operational interpretation of these processes is also
distinct from the null process. Intuitively, we expect that the first
outputs the name $\quotep{0}$ on the channel $\quotep{0}$, much like
the ordinary {\pic} process $x\id{[}x\id{]}$ outputs the name $x$ on
the channel $x$, and the second inputs on the channel $\quotep{0}$,
much like the ordinary {\pic} process $x\id{(}x\id{)} \concat 0$
inputs on the channel $x$.

Of course, now that we have two more processes, we have two more
names, $\quotep{\quotep{0}\id{[}{\quotep{0}}\id{]}}$ and
$\quotep{\quotep{0}\id{(}{\quotep{0}}\id{)} \concat 0}$. Having three
names at our disposal we can construct a whole new supply of processes
that generate a fresh supply of names, and we're off and running. It
should be pointed out, though, that as soon as we had the null process
we also had $0 \juxtap 0$ and $0 \juxtap 0 \juxtap 0$ and
consequently, we had the names $\quotep{ 0 \juxtap 0 }$, and $\quotep{
0 \juxtap 0 \juxtap 0}$, and .... But, since we ultimately wish to
treat these compositions as merely other ways of writing the null
process and not distinct from it, should we admit the codes of these
processes as distinct from $\quotep{0}$?

\subsection{Free and bound names}

The syntax has been chosen so that a binding occurrence of a name is
sandwiched between round braces, $\id{(} \cdot \id{)}$. Thus, the
calculation of the free names of a process, $P$, denoted
$\freenames{P}$ is given recursively by


\begin{eqnarray}
  \freenames{\pzero} & = & \emptyset \nonumber\\
  \freenames{x \id{(}y\id{)} \concat P} & = & \{ x \} \cup (\freenames{P} \setminus \{ y \}) \nonumber\\
  \freenames{\lift{x}{P}} & = & \{ x \} \cup \freenames{P} \nonumber\\
  \freenames{\binpar{P}{Q}} & = & \freenames{P} \cup \freenames{Q} \nonumber\\
  \freenames{\dropn{x}} & = & \{ x \} \nonumber
\end{eqnarray}

An occurrence of $x$ in a process $P$ is \textit{bound} if it is not
free. The set of names occurring in a process (bound or free) is
denoted by $\names{P}$.

\subsection{Structural congruence}

The {\em structural congruence} of processes, noted $\scong$, is the
least congruence, containing $\alpha$-equivalence, $\alphaeq$, that
satisfies the following laws:

\begin{eqnarray}
  {P} \juxtap \pzero	
  &  \scong \; {P} \; \scong & 
  \pzero \juxtap {P} \nonumber\\
  {P} \juxtap {Q}	
  & \scong & 
  {Q} \juxtap {P} \nonumber\\
  ({P} \juxtap {Q}) \juxtap {R}
  & \scong & 
  {P} \juxtap ({Q} \juxtap {R}) \nonumber
\end{eqnarray}

\subsection{Name equivalence}

We take name equivalence, written $\nameeq$, to be the smallest
equivalence relation generated by the following rules.

\infrule[Quote-drop]
{  }
{ \quotep{\dropn{x}} \nameeq x }

\infrule[Struct-equiv]
{ P \scong Q }
{ \quotep{P} \nameeq \quotep{Q} }

\subsection{Syntactic substitution}

Now we build the substitution used by $\alpha$-equivalence. We use
$\Proc$ for the set of processes, $\QProc$ for the set of names, and
$\id{\{}\vec{y} / \vec{x} \id{\}}$ to denote partial maps, $s : \QProc
\rightarrow \QProc$. A map, $s$ lifts, uniquely, to a map on process terms, $\widehat{s} :
\Proc \rightarrow \Proc$ by the following equations.

\begin{eqnarray}
  (0) \psubstp{Q}{P}         & = &    0 \nonumber\\
  (R \juxtap S) \psubstp{Q}{P}
  & = &    
  (R)\psubstp{Q}{P} \juxtap (S) \psubstp{Q}{P} \nonumber\\
  (x \id{(} y\id{)}\concat R) \psubstp{Q}{P}    
  & = &    
  (x)\substp{Q}{P} \id{(}z\id{)}\concat( (R \psubstn{z}{y}) \psubstp{Q}{P} ) \nonumber\\
  (\lift{x}{R}) \psubstp{Q}{P}  
  & = &
  \lift{(x)\substp{Q}{P}}{ R \psubstp{Q}{P} } \nonumber\\
  (\dropn{x})  \psubstp{Q}{P}       
  & = & 
  \left\{ 
    \begin{array}{ccc} 
      \dropn{\quotep{Q}} & & x \nameeq \quotep{P} \\
      \dropn{x} & & otherwise \\
    \end{array}
  \right. \nonumber
\end{eqnarray}
 

where

\begin{eqnarray}
  (x)\id{\{} \lpquote Q \rpquote / \lpquote P \rpquote \id{\}}            = 
  \left\{ 
    \begin{array}{ccc}
      \lpquote Q \rpquote & & x \nameeq \lpquote P \rpquote \\
      x & & otherwise \\
    \end{array}
  \right. \nonumber
\end{eqnarray}

and $z$ is chosen distinct from $\quotep{P}$, $\quotep{Q}$, the free
names in $Q$, and all the names in $R$. Our $\alpha$-equivalence will
be built in the standard way from this substitution.

\subsection{ Dynamic quote: an example }

Anticipating something of what's to come, consider applying the
substitution, $\widehat{\id{\{}u / z \id{\}}}$, to the following pair
of processes, $\lift{w}{y \id{[}z\id{]}}$ and $w\id{[} \lpquote
y\id{[}z\id{]} \rpquote \id{]}$.

\begin{eqnarray}
  \lift{w}{y \id{[}z\id{]}}\widehat{\id{\{}u / z \id{\}}}
  & = &
  \lift{w}{y \id{[}u\id{]}} \nonumber\\
  w\id{[} \lpquote y\id{[}z\id{]} \rpquote \id{]} \widehat{ \id{\{}u / z \id{\}} }
  & = &
  w\id{[} \lpquote y\id{[}z\id{]} \rpquote \id{]} \nonumber
\end{eqnarray}

Because the body of the process between quotes is impervious to
substitution, we get radically different answers. In fact, by
examining the first process in an input context, e.g. $x \id{(} z
\id{)}\concat \lift{w}{y
\id{[}z\id{]}}$, we see that the process under the lift operator may
be shaped by prefixed inputs binding a name inside it. In this sense,
the lift operator will be seen as a way to dynamically construct
processes before reifying them as names.

\subsection{Semantic substitution}

The substitution used in $\alpha$-equivalence is really only a device
to formally recognize that binding occurrences do not depend on the
specific names. It is not the engine of computation. The proposal here
is that while synchronization is the driver of that engine, the real
engine of computation is a semantic notion of substitution that
recognizes that a dropped name is a request to run a process. Which
process? Why the one whose code has been bound to the name being
dropped. Formally, this amounts to a notion of substitution that
differs from syntactic substitution in its application to a dropped
name.

\begin{eqnarray}
(\dropn{x})  \psubstp{Q}{P}       
		& = & 
		\left\{ 
			\begin{array}{ccc} 
				Q & & x \nameeq \quotep{P} \\
                              	\dropn{x} & & otherwise \\
			\end{array}
		\right. \nonumber
\end{eqnarray}

In the remainder of the paper we will refer to semantic and syntactic
substitutions simply as substitutions and rely on context to
distinguish which is meant. Similarly, we will abuse notation and
write $\substn{y}{x}$ for $\psubstn{y}{x}$.

Finally equipped with these standard features we can present the
dynamics of the calculus.

\subsection{Operational Semantics}
The reduction rules for {\rhoc}  are
% \infrule[Comm]
% { {{x}_0 \nameeq {x}_3} }
% {{{{x}_{0}\id{[}{x}_{2}\id{]}}} \juxtap {{x}_{3}\id{(}{x}_{1}\id{)} \concat {P}}
% \red {P}\id{\{}{x}_{2} \id{/} {x}_{1} \id{\}}}

%\infrule[Comm]
%{ {x}_{0} \nameeq {x}_{1} }
%{{{ x_{0} \id{[} z \id{]}}} \juxtap { x_{1} \id{(} y \id{)} \concat {P}}
%\red {P}\id{\{} z \id{/} y \id{\}}}

\infrule[Comm]
{ {x}_{0} \nameeq {x}_{1} }
{{\lift{x_{0}}{Q} \juxtap { x_{1} \id{(}{y} \id{)} \concat {P}}}
\red {P}\id{\{}\quotep{Q} \id{/} {y} \id{\}}}

In addition, we have the following context rules:

\infrule[Par]{{P} \red {P}'}{{{P} \juxtap {Q}} \red {{P}' \juxtap {Q}}}

\infrule[Equiv]{{{P} \scong {P}'} \andalso {{P}' \red {Q}'} \andalso {{Q}' \scong {Q}}}{{P} \red {Q}}

The context rules are entirely standard and we do not say much about
them, here. The communication rule does what was promised, namely make
it possible for agents to synchronize and communicate processes
packaged as names. For example, using the comm rule and name
equivalence we can now justify our syntactic sugar for output.

\begin{eqnarray}
  & {x}\id{[}{z}\id{]} \juxtap x \id{(}y\id{)} \concat P & \nonumber\\
  = 
  & \lift{x}{\dropn{z}} \juxtap x \id{(}y\id{)} \concat P & \nonumber\\
  \red 
  & P \substn{\quotep{\dropn{z}}}{y} & \nonumber\\
  \scong & P \substn{z}{y} & \nonumber
\end{eqnarray}

But, it also provides a scheme that identifies the role of name
equality in synchronization. We explore this family of calculi in a
forthcoming paper. For the rest of this paper, however, we restrict
our attention to the calculus with the less exotic communication rule,
using $\red$ for reduction according to that system and $\wred$ for
$\red^*$.

\section{Replication}

As mentioned before, it is known that replication (and hence
recursion) can be implemented in a higher-order process algebra
\cite{SangiorgiWalker}. As our first example of calculation with the
machinery thus far presented we give the construction explicitly in
the {\rhoc}.

\begin{eqnarray}
  D(x) & \triangleq & \prefix{x}{y}{(\binpar{\outputp{x}{y}}{\dropn{y}})} \nonumber\\
  \bangp{P}(x) & \triangleq & \binpar{\lift{x}{\binpar{D(x)}{P}}}{D(x)} \nonumber
\end{eqnarray}

\begin{eqnarray}
  \bangp{P}(x) & & \nonumber\\
  =
  & \lift{x}{(\prefix{x}{y}{(\outputp{x}{y} \juxtap \dropn{y})) \juxtap P}} 
  \juxtap \prefix{x}{y}{(\outputp{x}{y} \juxtap \dropn{y})} & \nonumber\\
  \red
  & (\outputp{x}{y} \juxtap \dropn{y})\substn{\quotep{(\prefix{x}{y}{(\dropn{y} \juxtap \outputp{x}{y})) \juxtap P}}}{y} & \nonumber\\
  =
  & \outputp{x}{\quotep{(\prefix{x}{y}{(\outputp{x}{y} \juxtap \dropn{y})) \juxtap P}}}
  \juxtap {(\prefix{x}{y}{(\outputp{x}{y} \juxtap \dropn{y})) \juxtap P}} & \nonumber\\
  \red
  & \ldots & \nonumber\\
  \red^*
  & P \juxtap P \juxtap \ldots & \nonumber
\end{eqnarray}

Of course, this encoding, as an implementation, runs away, unfolding
$\bangp{P}$ eagerly. A lazier and more implementable replication
operator, restricted to input-guarded processes, may be obtained as follows.

\begin{eqnarray}
\bangp{\prefix{u}{v}{P}} 
	\triangleq 
	\binpar{\lift{x}{\prefix{u}{v}{(\binpar{D(x)}{P})}}}{D(x)} \nonumber
\end{eqnarray}

It is worth noting that the lift operator is essential to get
computational completeness. A similar calculus equipped with only a
static quote enjoys a computational expressiveness at least equivalent
to context-free grammars, but short of context-sensitive. This fact is
established and exploited in a forthcoming paper on a type system for
the {\rhoc}.

\section{Bisimulation}

Having taken the notion of restriction out of the language, we
carefully place it back into the notion of observation, and hence into
the notion of program equality, i.e. bisimulation. That is, we
parameterize the notion of barbed bisimulation by a set of names over
which we are allowed to set the barbs. The motivation for this choice
is really comparison with other calculi. The set of names of the
{\rhoc} is \textit{global}. It is impossible, in the grammar of
processes, to guard terms from being placed into contexts that can
potentially observe communication. So, we provide a place for
reasoning about such limitations on the scope of observation in the
theory of bisimulation.

\begin{definition}
An \emph{observation relation}, $\downarrow_{\mathcal N}$, over a set
of names, $\mathcal N$, is the smallest relation satisfying the rules
below.

\infrule[Out-barb]{y \in {\mathcal N}, \; x \nameeq y}
		  {\outputp{x}{v} \downarrow_{\mathcal N} x}
\infrule[Par-barb]{\mbox{$P\downarrow_{\mathcal N} x$ or $Q\downarrow_{\mathcal N} x$}}
		  {\binpar{P}{Q} \downarrow_{\mathcal N} x}

We write $P \Downarrow_{\mathcal N} x$ if there is $Q$ such that 
$P \wred Q$ and $Q \downarrow_{\mathcal N} x$.
\end{definition}

Notice that $\prefix{x}{y}{P}$ has no barb.  Indeed, in {\rhoc} as well
as other asynchronous calculi, an observer has no direct means to
detect if a message sent has been received or not.

\begin{definition}
%\label{def.bbisim}
An  ${\mathcal N}$-\emph{barbed bisimulation} over a set of names, ${\mathcal N}$, is a symmetric binary relation 
${\mathcal S}_{\mathcal N}$ between agents such that $P\rel{S}_{\mathcal N}Q$ implies:
\begin{enumerate}
\item If $P \red P'$ then $Q \wred Q'$ and $P'\rel{S}_{\mathcal N} Q'$.
\item If $P\downarrow_{\mathcal N} x$, then $Q\Downarrow_{\mathcal N} x$.
\end{enumerate}
$P$ is ${\mathcal N}$-barbed bisimilar to $Q$, written
$P \wbbisim_{\mathcal N} Q$, if $P \rel{S}_{\mathcal N} Q$ for some ${\mathcal N}$-barbed bisimulation ${\mathcal S}_{\mathcal N}$.
\end{definition}

% subsection the_syntax_and_semantics_of_the_notation_system (end)   