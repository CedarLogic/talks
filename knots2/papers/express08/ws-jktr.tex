%%%%%%%%%%%%%%%%%%%%%%%%%%%%%%%%%%%%%%%%%%%%%%%%%%%%%%%%%%%%%%%%%%%%%%%%%%%%
%% Trim Size: 9.75in x 6.5in
%% Text Area: 8in (include Runningheads) x 5in
%% ws-jktr.tex   :   10-4-06
%% Tex file to use with ws-jktr.cls written in Latex2E. 
%% The content, structure, format and layout of this style file is the 
%% property of World Scientific Publishing Co. Pte. Ltd. 
%% Copyright 1995, 2002 by World Scientific Publishing Co. 
%% All rights are reserved.
%%%%%%%%%%%%%%%%%%%%%%%%%%%%%%%%%%%%%%%%%%%%%%%%%%%%%%%%%%%%%%%%%%%%%%%%%%%%
%

\documentclass{ws-jktr}

\begin{document}

\markboth{Authors' Names}
{Instructions for Typesetting Camera-Ready Manuscripts}

%%%%%%%%%%%%%%%%%%%%% Publisher's Area please ignore %%%%%%%%%%%%%%
\catchline{}{}{}{}{}
%%%%%%%%%%%%%%%%%%%%%%%%%%%%%%%%%%%%%%%%%%%%%%%%%%%%%%%%%%%%%%%%%%%

\title{INSTRUCTIONS FOR TYPESETTING CAMERA-READY\\
MANUSCRIPTS USING COMPUTER SOFTWARE\footnote{Typeset title in 
10~pt Times Roman uppercase and boldface. Please write
down in pencil a short title to be used as the running head.}}

\author{FIRST A. AUTHOR, SECOND B. AUTHOR and THIRD
C. AUTHOR\footnote{Typeset names in 8~pt Times Roman, uppercase
and lightface.  Use footnotes only to indicate if permanent and
present addresses are different. Funding information should go
in the Acknowledgement section.}}

\address{Full affiliations\footnote{Typeset 
affiliation and mailing addresses in 8pt Times italic.} \\
and mailing addresses}

\author{OTHER D. AUTHOR}

\address{Full affiliations \\
and mailing addresses}

\maketitle

\begin{abstract}
The abstract should summarize the context, content and
conclusions of the paper in less than 200 words. It should not
contain any references or displayed equations. Typeset the
abstract in 8 pt Times Roman with baselineskip of 10 pt, making
an indentation of $\frac14$ inch on the left and right margins.
Typeset similarly for keywords below.
\end{abstract}

\keywords{List four to six; Should characterize article.}

\ccode{Mathematics Subject Classification 2000: 57M25, 57M27}

\section{General Appearance}	%) A SECTION HEADING
Contributions to the {\it Journal of Knot Theory and Its 
Ramifications} will be reproduced by photographing the author's
submitted typeset manuscript.  It is therefore essential that
the manuscript be in its final form, and is an original computer
printout because it will be printed directly without any
editing. The manuscript should also be clean and unfolded. The
copy should be evenly printed on a high resolution printer (300 
dots/inch or higher). If typographical errors cannot be avoided,
use cut and paste methods to correct them. Smudged copy, pencil
or ink text corrections will not be accepted. Do not use 
cellophane or transparent tape on the surface as this interferes
with the picture taken by the publisher's camera.

\section{The Main Text}
Contributions are to be in English. Authors are 
encouraged to have their contribution checked for grammar. 
Abbreviations are allowed but should be spelt out in full when first
used.  Integers ten and below are to be spelt out. Italicize foreign
language phrases (e.g.~Latin, French).

The text should be in 10 pt Times Roman, single spaced with
baselineskip of 13~pt. Text area (excluding copyright block and folio)
is 6.9 inches high and 5 inches wide for the first page.  Text area
(excluding running title) is 7.7 inches high and 5 inches wide for
subsequent pages.  Final pagination and insertion of running titles
will be done by the publisher.

\section{Major Headings}
Major headings should be typeset in boldface with the first
letter of important words capitalized.

\subsection{Sub-headings}
Sub-headings should be typeset in boldface italics. 

\subsubsection{Sub-subheadings}
Sub-subheadings should be typeset in italics.

\subsection{Numbering and spacing}
Sections, sub-sections and sub-subsections are to be numbered in
Arabic.

\subsection{Lists of items}
Lists may be laid out with each item marked by a dot:

\begin{itemlist}
 \item item one,
 \item item two.
\end{itemlist}

\begin{romanlist}[(ii)]
\item item one
\item item two 
	\begin{romanlist}[(b)]
	\item Lists within lists can be numbered with lowercase 
              Roman letters,
	\item second item. 
	\end{romanlist}
\end{romanlist}

\section{Equations}
Displayed equations should be numbered consecutively in each
section, with the number set flush right and enclosed in
parentheses, 
\eject

\noindent
\begin{equation}
\mu(n, t) = \frac{\sum^\infty_{i=1} 1(d_i < t, 
N(d_i) = n)}{\int^t_{\sigma=0} 1(N(\sigma) = n)d\sigma}\,.
\label{this}
\end{equation}

Equations should be referred to in abbreviated form,
e.g.~``Eq.~(\ref{this})'' or ``(4.1)''. In multiple-line
equations, the number should be given on the last line.

Displayed equations are to be centered on the page width.
Standard English letters like x are to appear as $x$
(italicized) in the text if they are used as mathematical
symbols. Punctuation marks are used at the end of equations as
if they appeared directly in the text.

\begin{theorem}
Theorems$,$ lemmas$,$ etc. are to be numbered consecutively in the paper.
\end{theorem}

\begin{proof}
Proofs should end with a square.
\end{proof}

\section{Illustrations and Photographs}
Figures are to be inserted in the text nearest their first
reference. Original India ink drawings of glossy prints are
preferred. Please send one set of originals with copies. If the
author requires the publisher to reduce the figures, ensure that
the figures (including letterings and numbers) are large enough
to be clearly seen after reduction. If photographs are to be
used, only black and white ones are\break
acceptable.

\begin{figure}[th]
\centerline{\psfig{file=jktrf1.eps,width=2.2in}}
\vspace*{8pt}
\caption{Labeled tree {\it T}.\label{fig1}}
\end{figure}

Figures are to be sequentially numbered in Arabic numerals. The
caption must be placed below the figure. Typeset in 8~pt Times
Roman with baselineskip of 10~pt. Use double spacing between a
caption and the text that follows immediately.

Previously published material must be accompanied by written
permission from the author and publisher.

Figures should be referred to in the abbreviated form, 
e.g.~``$\ldots$ in Fig.~\ref{fig1}'' or ``$\ldots$ in   
Figs.~\ref{fig1} and 2''. Where the word ``Figure'' begins a 
sentence, it should be spelt\break
in full.

\section{Tables}
Tables should be inserted in the text as close to the point of
reference as possible. Some space should be left above and below
the table.

\begin{table}[ht]
\tbl{Comparison of acoustic for frequencies for piston-cylinder problem.}
{\begin{tabular}{@{}cccc@{}} \toprule
Piston mass & Analytical frequency & TRIA6-$S_1$ model &
\% Error \\
& (Rad/s) & (Rad/s) \\ \colrule
1.0\hphantom{00} & \hphantom{0}281.0 & \hphantom{0}280.81 & 0.07 \\
0.1\hphantom{00} & \hphantom{0}876.0 & \hphantom{0}875.74 & 0.03 \\
0.01\hphantom{0} & 2441.0 & 2441.0\hphantom{0} & 0.0\hphantom{0} \\
0.001 & 4130.0 & 4129.3\hphantom{0} & 0.16\\ \botrule
\end{tabular}}
\end{table}

Tables should be numbered sequentially in the text in Arabic
numerals. Captions are to be centralized above the tables.
Typeset tables and captions in 8~pt Times Roman with
baselineskip of 10~pt.

If tables need to extend over to a second page, the continuation
of the table should be preceded by a caption, 
e.g.~``Table~2. Cont'd.''

\section{References}
The format for references should be strictly followed.
References in the text are to be numbered in Arabic numerals.
They are to be cited in square brackets, \cite{2} before punctuation
marks. Standard journal abbreviations are preferred.

\section{Footnotes}
Footnotes should be numbered sequentially in superscript Arabic
numerals.\footnote{Footnotes should be typeset in 8~pt Times Roman 
at the bottom of the page.}

\appendix

\section{Appendices}

Appendices should be used only when absolutely necessary. They 
should come\break
before Acknowledgments. \forcebreak If there is more than one 
appendix, number them\break 
alphabetically.  Number displayed equations occurring in the Appendix
in this way, e.g.~(\ref{appeqn}), (A.2), etc.
\begin{equation}
f(j\delta, i\delta) \cong \frac{\pi}{M} \sum^M_{n-1}
Q_{\theta_n} (j\cos \theta_n + i\sin \theta_n)\, .\label{appeqn}
\end{equation}

\section*{Acknowledgments}
This section should come before the References. Funding 
information may also be included here.

\begin{thebibliography}{0}
\bibitem{1} R. Lorentz and D. B. Benson, Deterministic 
and nondeterministic flow-chart interpretations, 
{\it J. Comput. System Sci}.  {\bf 27} (1983) 400--433.

\bibitem{2} M. J. Beeson, {\it Foundations of Constructive 
Mathematics} (Springer, Berlin, 1985).

\bibitem{3} K. L. Clark, Negations as failure, in 
{\it Logic and Data Bases}, eds. H. Gallaire and\break 
J. Winker (Plenum Press, New York, 1973), pp.~293--306.

\bibitem{4} M. Joliat, A simple technique for partial 
elimination of unit productions from LR(k) parsers, 
{\it IEEE Trans. Comput}. {\bf C-27} (1976) 753--764.

\bibitem{5} D. Dolve, Unanimity in an unknown and 
unreliable environment, in {\it Proc. 22nd Annual Symposium on 
Foundations of Computer Science}, Nashville, TN (Oct. 1981),\break 
pp.~159--168.

\bibitem{6} R. Tamassia, C. Batini and M. Talamo, An  
algorithm for automatic layout of entity relationship diagrams, 
in {\it Entity-Relationship Approach to Software  
Engineering}, {\it Proc.  3rd Int. Conf. on Entity-Relationship 
Approach}, eds. C. G. Davis, S.  Jajodia, P. A. Ng and R. T. Yeh 
(North-Holland, Amsterdam, 1983), pp.~421--439.

\bibitem{7} W. L. Gewirtz, Investigations in the theory
of descriptive complexity, Ph. D. Thesis, New York University
(1974).
\end{thebibliography}

\end{document}

%%Can work for this too:-
\begin{lemma}[the name of the lemma]
Testing for lemma text only.
\end{lemma}

\begin{proposition}[the name of the proposition]
Testing for proposition text only.
\end{proposition}

