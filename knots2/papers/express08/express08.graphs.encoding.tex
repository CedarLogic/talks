\subsection{The general encoding}

Given a graph, $G$, a pair of injective maps,
$\widehat{\cdot},\cdot^{\circ}:V(G) \hookrightarrow \mathcal{N}$, satisfying
$\widehat{V(G)} \cap V(G)^{\circ} = \emptyset$, we encode $G$ by %$\meaningof{G}_{\pi}^H(V(G),\emptyset)$ where

% \begin{eqnarray}  
%   %\lefteqn{\meaningof{G}_{\pi}^H(V,E) \triangleq} % \nonumber \\
%   \meaningof{G}_{\pi}^H(V,E) \triangleq
%   % & & \mbox{}
%   \textsf{let} \; v \uplus V' \; \textsf{=} \; V \; \textsf{in} %\nonumber \\
%   % & & \mbox{}
%   \meaningof{v}_{\pi}(E) | \meaningof{G}_{\pi}^H(V',E \cup cone(v))
% \end{eqnarray}

% and

\begin{eqnarray}
  \meaningof{G}_{\pi} & \triangleq & \Pi_{v \in V(G)}\meaningof{v}_{\pi} \\
  \meaningof{v}_{\pi} & \triangleq & \Sigma_{e \in \textsf{slice(v)}} \hat{v}(s).\widehat{\textsf{trgt}(e)}[s].\meaningof{v}_{\pi} + (\nu \; t)v^{\circ}(s).s[t]
\end{eqnarray}

\begin{remark}
  An intuitively appealing encoding would, for each vertex, record its
  participation as the endpoint of an edge as potential
  interaction. Discrete graphs, however, have vertices that are
  distinguishable \emph{despite} the absence of any edge
  structure. Thus, the encoding must also accomodate this additional
  capacity for distinguishing vertices as additional possible behavior
  at each process representing a vertex. This fact is methodologically
  noteworthy: if the mathematical data makes exploitable distinctions
  a faithful process representation must reflect these distinctions as
  distinguishable behavior. Put another way, the world-view operative
  in process calculi dictates that any difference that makes a
  difference must show up as a distinct way of kicking or prodding
  some process.   
\end{remark}

On the other hand, it is easy to see that

\begin{lemma}
  If $G$ is such that for all $v \in V(G), \textsf{e}(v) \neq
  \emptyset$, then for any distinct $v_0, v_1 \in V(G)$ $(\nu \;
  v_0^{\circ})\meaningof{v_0}_{\pi} \not\simeq (\nu \;
  v_1^{\circ})\meaningof{v_1}_{\pi}$.
\end{lemma}

%\begin{proof}
That is, in every graph in which every vertex participates as the
endpoint of at least one edge, then even if we hide behavior at
$v_i^{\circ}$, the processes $(\nu \;
v_i^{\circ})\meaningof{v_i}_{\pi}$ are distinct for distinct nodes,
$v_0, v_1$. For, in the worst case, when the cone of $v_0$ equals the
cone of $v_1$ there can never be an edge in which they are
simultaneously the source or simultaneously the target. Hence, there
is always some distinguishing behavior. In fact, the adventurous
reader is encouraged to write down the distinguishing Hennessy-Milner
formula.
%\end{proof}

This fact allows us to employ a simpler encoding for graphs of this type, i.e.

\begin{eqnarray}\label{vcode}
  \meaningof{v}_{\pi} & \triangleq & \Sigma_{e \in \textsf{slice(v)}} \hat{v}(s).\widehat{\textsf{trgt}(e)}(s).\meaningof{v}_{\pi}
\end{eqnarray}

In a similar vein we have

\begin{theorem}\label{grapheqthm}
  $\meaningof{G}_{\pi} \simeq \meaningof{H}_{\pi} \iff  G \sim H$.
\end{theorem}

\begin{proof}
  By construction.
\end{proof}

\begin{remark}
  We observe that there is an dual encoding given by
  \begin{eqnarray}
  \meaningof{G}_{\pi}^{\bullet} & \triangleq & \Pi_{v \in V(G)}\meaningof{v}_{\pi}^{\bullet} \\
  \meaningof{v}_{\pi}^{\bullet} & \triangleq & \Sigma_{e \in \textsf{co-slice(v)}} \widehat{\textsf{src}(e)}(s).\hat{v}[s].\meaningof{v}_{\pi}^{\bullet} + (\nu \; t)v^{\circ}(s).s[t]
\end{eqnarray}
\end{remark}

\subsubsection{Examples}

\begin{example}
  \begin{enumerate}
    \item $\meaningof{G(n)}_{\pi} = \Pi_{i=0}^{n-1}(\nu \; t)i^{\circ}(s).s[t]$

    \item $\meaningof{C(n)}_{\pi} = \Pi_{i=0}^{n-2}\hat{i}(s).\widehat{i+1}[s]$

    \item $\meaningof{L(n)}_{\pi} = \Pi_{i=0}^{n-1}\hat{i}(s).\widehat{(i+1 \; mod \; n)}[s]$
  \end{enumerate}
\end{example}

