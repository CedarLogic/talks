\subsubsection{Probabilistic pay-offs}
This is much more than just a convenient refactoring. It represents a
fairly deep set of ontological categories that have natural
polymorphic interpretations, and natural extensions. To illustrate
this point consider Priami or Cardelli and Phillips' accounts of the
stochastic $\pi$-calculus \cite{DBLP:journals/cj/Priami95}
\cite{Phillips-Cardelli-Bioconcur04}. Both presentations make two
essential changes to the calculus. One is a syntactic change to the
shape of process constructors and the other is a change to the
reduction relation.

The first modification is based on a typing of sequence as $action
\times process \to process$. One way of seeing their account is that
they interpret this as a polymorphic operation in which actions may be
extended to have additional structure, $(x?(y),r) \; | \; (x!(z),r)$,
namely a rate from which a probability of interaction may be
calculated given a sampling window and global description of
contending candidates for reduction. To carry this interpretation out,
they modify the reduction relation to take an additional parameter,
namely the rate. (Priami's original account differs from the
Cardelli-Phillips account in how the probability is related to the
rate in certain contention cases -- a difference that has little
import in the discussion here.)

In an interpretation of sequence as typed $name \times (abstraction
\oplus concretion) \to process$ we see that there are two places to
consider as candidates for probabilistic or stochastic treatment:
synchronization and data-exchange. Both of these have natural physical
interpretations:

\begin{itemize} 
\item synchronization, like the docking of a protein on a gene site,
  is probabilistic, or fuzzy;
\item data-exchange, like the sharing of an electron in
  electron-bonded compounds, is probabilistic.
\end{itemize}

We can recover the Priami/Cardelli/Phillips interpretations by the
\emph{single} syntactic extension, $(x,r).A$. Like the systems of
Priami or Cardelli and Phillips we may interpret this as a polymorphic
interpretation of \emph{location} as having further articulated
structure, namely the rate, $r$. 

\begin{theorem}
  The variant
  \begin{mathpar}
    \inferrule* [lab=summation] {} {{M,N} \bc \pzero \;|\; l.A \;|\; M+N}
    \and
    \inferrule* [lab=agent] {} {{A} \bc (\vec{x})P \;| \; [\vec{x}]P}
    \and \\
    \inferrule* [lab=process] {} {{P,Q} \bc N \;| \;P|Q \;| \; (\nu \; \vec{x})P \;| \;X\langle \vec{y} \rangle \;| \; (\textsf{rec} \; X(\vec{x}).P)\langle \vec{y} \rangle }
    \and \\
    \inferrule* [lab=location] {} {l \bc (x,r)}
  \end{mathpar} 
  \begin{mathpar}
    \inferrule* [lab=$comm_{AC}$] {(x,r) = l} {l.A | l.C \to^{r} A \cdot C}
  \end{mathpar}
  fully and faithfully encodes the Cardelli-Phillips stochastic $\pi$-calculus
\end{theorem}

Additionally, we may explore presentation of probabilistic
data-exchange. It is outside of the scope of this paper to present
them, but there are very natural interpretations that admit quantum
computational interpretations.

The central point of this exegesis is that Milner's decomposition of
the sequence operator in \cite{milner91polyadicpi} -- which we stress
as something akin in scope to Girard's decomposition of implication
\cite{DBLP:journals/tcs/Abramsky93} -- has more pay-off than just a
hygenic account of scoping of binding operations. It also provides a
simpler account of existing probabilistic interpretations and also
points the way to additional and potentially useful probabilistic
theories. Thus, despite the lack of treatment of concrete examples of
interaction across scale, we are still accruing evidence in favor of a
certain declension of the theory; exposing a certain warp and woof to
it that favors certain usage patterns.

