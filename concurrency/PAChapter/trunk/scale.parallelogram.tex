We begin with a simple example of a meta-programming problem in a
concurrency setting. We would like to use reflection to implement
i/o-guarded replication. Here is a first attempt.

\begin{verbatim}

    rep?(A.K).
        {                                       
          x?({ a.k; dup }).
            { @a.{ @k; @dup }; 
              x![{ @a.@k; @dup }] 
            };
          x![{ @A.@K; 
               x?({ a.k; dup }).
                 { @a.{ @k; @dup }; 
                   x![{ @a.@k; @dup }] 
                 }
             }]
         }

\end{verbatim}

[Editor's note: alternate syntax 1]

\begin{verbatim}

    {
      rep?(A.K);
      {                                       
        x?({ a.k, dup });
        { 
          @a;
          { @k, @dup },
          x![{ @a.@k, @dup }] 
         },
        x![{ @A.@K, 
             x?({ a.k, dup });
             { 
               @a;
               { @k, @dup },
               x![{ @a.@k, @dup }] 
              }
           }]
         }
     }

\end{verbatim}

What we'd like this program to do is this. On \textsf{rep} receive an
\textsf{A}-guarded program with continuation \textsf{K} then run two
programs in parallel:
\begin{itemize}
\item one program, which we call R in the sequel, that receives the
  parallel composition of an \textsf{a}-guarded program with
  continuation \textsf{k}, and another program, \textsf{dup}, and then
  fires two threads:
  \begin{itemize}
  \item one that waits to do \textsf{a} and then runs \textsf{k} and
    \textsf{dup} in parallel and
  \item one that forwards what was originally received on \textsf{x}
    back on \textsf{x}
  \end{itemize}
\item the other program sends on x the parallel composition of what was
  received on \textsf{rep} along with R.
\end{itemize}

Unfortunately, there is still not enough machinery in the program to
guarantee this behavior. In particular, both \textsf{A.K} and R are
i/o-guarded programs. Thus, there is no guarantee that \textsf{A.K}
will be matched with \textsf{a.k} and \textsf{dup} with R. These could
be matched the other way around, and non-deterministically at
that. That is, on one `iteration' it might work, and another things go
horribly wrong.

Before proposing and analyzing any solutions, we simply wish to
observe that the problem, as simple as it is, exercises a number of
language features that are rarely brought together in a single
language. To wit:

\begin{itemize}
  \item passing programs as messages;
  \item pattern-matching (on the structure of programs as messages);
  \item parallel composition;
  \item sequencing
\end{itemize}

Note that the program-passing and pattern-matching introduce the
\textsf{@} syntax for turning (matched elements of) messages back into
programs. Implicit in this notation is that until such an element is
turned into program it is treated as a \emph{channel name}. Moreover,
program code is the \emph{only} source of channel names, which are, in
turn the core of the synchronization and message-passing machinery.

It turns out that this collection of features is sufficient for the
language to be fully expressive. This puts us in a relatively ideal
position: we have a small manageable laboratory for investigating
meta-programming in a concurrency setting and we have a small, but
compelling, problem that exercises virtually all of those features.

Among the many possible alternatives to modify the program above to
ensure the desired behavior we consider one whose primary purpose is
to motivate a discussion of the tension between reflection and
pattern-matching and various possible resolutions. The key idea in the
approach to resolving the problem identified above is to make the
match with the replicating behavior so specific that it is easy to
differentiate it from the i/o-guarded behavior. \footnote{Will that
  cause problems when replicating replicated programs? Well... in
  general we want $!P \equiv !!P$.} We can then select for this in
our pattern-matching. 

The replicating machinery is expecting to receive on the channel named
\textsf{x} a program of the form \textsf{ \{ a.k; ... \} } where the
\textsf{a.k} is supposed to match (some copy of) the i/o-guarded
program and what's in the \textsf{...} is supposed to match the
replicating machinery. What does this machinery do? It receives on the
channel named \textsf{x} a program of the form \textsf{ \{ a.k; ... \}
}. To break this cycle we capture the pattern form needed by the
replicating machines in a name, call it \textsf{p}, and make it
available to rest of the code in the scope of the input guard. The
code below reduces these reflections (;-) to syntax.

This problem and solution illustrates the very core of the tension
between reflection and pattern-matching. To reflect one needs to be
able to turn every program language construct into data -- including
patterns. Rather than introducing pattern-patterns and
pattern-pattern-patterns, simply reusing the ability to turn program
elements -- including patterns -- into names appears to be sufficient
to address this particular programming problem. 

One key difference, however, is that the access to programmatic
reflection is made available without appeal to message
passing. \footnote{Can this be eliminated as syntactic sugar?}

\begin{verbatim}

    rep?(A.K).
        {                                       
          x?(rec p.{ a.k; x?(p).dup }).
            { @a.{ @k; x?(@p).@dup }; 
              x![{ @a.@k; x?(@p).@dup }] 
            };
          x![{ @A.@K; 
               x?(rec p.{ a.k; x?(@p).dup }).
                 { @a.{ @k; x?(@p).@dup }; 
                   x![{ @a.@k; x?(@p).@dup }] 
                 }
             }]
         }

\end{verbatim}


\section{Grammar}

% P           ::= select{ B0; ...; BN }
%                 @u
%                 { P0;...;PN }    

% B           ::= u A

% A           ::= F | C

% F           ::= ?(a0,...,aN).P
% C           ::= !(a0,...,aN) := (P0,...,PN).P

% a           ::= x
%                 q

% q           ::= _
%                 v
%                 s t
%                 @<<a>>
%                 { a0;...;an }

% t           ::= ?(a0,...,an).q
%                 !(a0,....,an) := (b0,...,bn).q

% s           ::= _
%                 u

% u           ::= x
%                 v

% x           ::= <<P>>
% v           ::= ['A'-'Z' 'a'-'z' '0'-'9' '_'] *

\begin{mathpar}
  \inferrule* [lab=Process,leftskip=2em] {} {P \bc
    \textsf{case}\textsf{\{} u_1 A_1 \textsf{;} \;\ldots \textsf{;}\;
    u_n A_n \textsf{\}} \;\;|\;\; \textsf{\{} P_1 \textsf{;} \;\ldots
    \textsf{;}\; P_n \textsf{\}} \;\;|\;\; \textsf{@}u }
  \and \\
  \inferrule* [lab=Agent] {} {A \bc \textsf{?} \textsf{(} a_1
    \textsf{,} \ldots \textsf{,} a_n \textsf{)}\textsf{.}P \;\;|\;\;
    \textsf{!} \textsf{(} a_1 \textsf{,} \ldots \textsf{,} a_n
    \textsf{)} \textsf{:=} \textsf{(} P_1 \textsf{,} \ldots \textsf{,}
    P_n \textsf{)}\textsf{.}P}
  \and \\
  \inferrule* [lab=Pattern] {} {a \bc e \;\;|\;\; q}
  \and
  \inferrule* [lab=Query] {} {q \bc s \;\;|\;\; \textsf{\{}a_1 \textsf{;} \ldots
    \textsf{;} a_n\textsf{\}} \;\;|\;\; \textsf{@}\langle\langle
    a\rangle\rangle \;\;|\;\; \textsf{rec}\;id \textsf{.}a }
  \and \inferrule* [lab=Sum] {} {s \bc \textsf{\{}\textsf{\}}
    \;\;|\;\; v \;\;|\;\; u \; g + s}
  \and
  \inferrule* [lab=Guard] {} {g \bc \textsf{?} t\textsf{.}q \;\;|\;\;
    \textsf{!} t \textsf{:=} t'\textsf{.}q \;\;|\;\;
    \textsf{*}v\textsf{.}q}
  \and \\
  \inferrule* [lab=Tuple] {} {t \bc \textsf{(}\textsf{)} \;\;|\;\; v
    \;\;|\;\;\textsf{(} m_1 \textsf{,} \ldots \textsf{,} m_n
    \textsf{)} \;\;|\;\; a \textsf{::} t} \and \inferrule*
  [lab=Nesting] {} {m \bc a \;\;|\;\; t}
  \and \\
  \inferrule* [lab=Site] {} {u \bc x \;\;|\;\; v} \and \inferrule*
  [lab=Code] {} {x \bc \langle\langle P \rangle\rangle}
  \and
  \inferrule* [lab=Variable] {} {v \bc id \;\;|\;\; \textsf{\_}}
  \and \\
  \inferrule* [lab=Value] {} {e \bc x \;\;|\;\; bool \;\;|\;\; nat \;\;|\;\; \ldots} 
  % \inferrule* [lab=Expression] {} {e \bc l \;\;|\;\; logical \;\;|\;\; numeric \;\;|\;\; \ldots} 
%   \and \\
%   \inferrule* [lab=Literal] {} {l \bc x \;\;|\;\; bool \;\;|\;\; nat \;\;|\;\; \ldots}
%   \and \\
%   \inferrule* [lab=Identifier] {} {id \bc n c_1 \ldots c_n}
%   \and \\
%   \inferrule* [lab=AlphaNum] {} { n \bc \textsf{'A'}\;|\;\ldots\;|\;\textsf{'z'} \;|\; \textsf{'0'}\;|\;\ldots\;|\;\textsf{'9'} }
%   \and \\
%   \inferrule* [lab=AlphaPlus] {} {c \bc l \;\;|\;\; \textsf{\_} \;\;|\;\; \textsf{-} \;\;|\;\; \ldots}
\end{mathpar}

\newpage

\subsubsection{Description}

\paragraph{Process} A program is one of three forms:

\begin{itemize}
\item a \emph{selection}, collection of i/o-guarded programs called
  \emph{branches}. A selection may be thought of as a parallel switch,
  or case statement in which the choice made amongst the branches is
  non-deterministic amongst the possible i/o matches in a
  communicating i/o-guarded statement (c.f. the Comm rule).
\item a \emph{drop}, turning the code of a program back into a program.
\item a \emph{composition}, a collection of programs running in
  parallel.
\end{itemize}

\paragraph{Action} I/o is represented syntactically in terms of actions.

\subsection{Structural equivalence}

% select{ B; {} } == select{ B }

% select{ B; select{ C; D } } == select{ B; C; D } == select{ select{ B; C }; D }
% select{ B; C } == select{ C; B }


% { P; {} } == { P }
% { P; { Q; R } } == { P; Q; R } == { { P; Q }; R }
% { P; Q } == { Q; P }

% P{w/v} == P (v \in BN(P), w \not\in FN(P))

\begin{eqnarray}
  \textsf{select}\textsf{\{} B \textsf{;} \textsf{\{}\textsf{\}} \textsf{\}} & == & \textsf{select}\textsf{\{} B \textsf{\}} \\
  \textsf{select}\textsf{\{} B \textsf{;} \textsf{select}\textsf{\{} C \textsf{;} D \textsf{\}}\textsf{\}} & == \textsf{select}\textsf{\{} B \textsf{;} C \textsf{;} D \textsf{\}} == & \textsf{select}\textsf{\{} \textsf{select}\textsf{\{} B \textsf{;} C \textsf{\}}\textsf{;} D \textsf{\}} \\
  \textsf{select}\textsf{\{} B \textsf{;} C \textsf{\}} & == & \textsf{select}\textsf{\{} C \textsf{;} B \textsf{\}} \\
%
\mbox{ } \\
  \textsf{\{} P \textsf{;} \textsf{\{} \textsf{\}} \textsf{\}} & == & \textsf{\{} P \textsf{\}} \\
  \textsf{\{} P \textsf{;} \textsf{\{} Q \textsf{;} R \textsf{\}} \textsf{\}} & == \textsf{\{} P \textsf{;} Q \textsf{;} R \textsf{\}} == & \textsf{\{} \textsf{\{} P \textsf{;} Q \textsf{\}}\textsf{;} R \textsf{\}} \\
  \textsf{\{} P \textsf{;} Q \textsf{\}} & == & \textsf{\{} Q \textsf{;} P \textsf{\}}
\end{eqnarray}

\subsubsection{Channel compatibility}

% <<P>> ~ <<Q>> <=> P == Q

\begin{equation}
  \langle\langle P \rangle\rangle \; \sim \; \langle\langle Q \rangle\rangle \iff P == Q
\end{equation}

\section{Semantics}

\subsection{Reduction}

% x_s ~ x_t, s |- a_i *= P_i, t |- b_i *= P_i
% =>
% { select{ B0;...;Bm; x_s(a0,...,an)?P }; select{ x_t(b0,...,bn) := (P0,...,Pn).Q } -> { Ps; Qt }


% P -> P' => { P; Q } -> { P'; Q}

% P == P', P' -> Q', Q' == Q => P -> Q

The core of these rules is the Comm rule. It provides a virtually
symmetric form of interaction. Input provides to its continuation a
view of the incoming data via the binding to the patterns in the input
action. Interestingly, output provides a markedly similar mechanism by
providing bindings of the data before sending it out.

\begin{mathpar}
  \inferrule* [Right=Comm] {x_s \sim x_t \\ s \vdash a_i \; \assign P_i \\ t \vdash b_i \; \assign P_i} 
 { \textsf{\{} \textsf{select} \textsf{\{} B_0 \textsf{;} \ldots \textsf{;} B_m \textsf{;} x_s \textsf{(} a_0 \textsf{,} \ldots \textsf{,} a_n \textsf{)} \textsf{?} P \textsf{\}} \textsf{;} \; \textsf{select} \textsf{\{} x_t \textsf{(} b_0 \textsf{,} \ldots \textsf{,} b_n \textsf{)} \textsf{:=} \textsf{(} P_0 \textsf{,} \ldots \textsf{,} P_n \textsf{)} \textsf{.}Q \textsf{;} B'_0 \textsf{;} \ldots \textsf{;} B'_n \textsf{\}} \textsf{\}} \\\\ \to \textsf{\{} Ps \textsf{;} Qt \textsf{\}} } \\\\

 \inferrule* [Right=Par] { P \to P' } { \textsf{\{} P \textsf{;} Q \textsf{\}} \to \textsf{\{} P' \textsf{;} Q \textsf{\}} } \\\\

 \inferrule* [Right=Equiv] { P == P' \\ P' \to Q' \\ Q' == Q } {  P \to Q }

\end{mathpar}

\subsection{Pattern-matching}

% [] |- _ *= P

% [<<P>>/v] |- v *= P

% s |- a *= P  => s |- _(a0,...,an)?a *= x(a0,...,an)?P
% y ~ x, s |- a *= P  => s |- y(a0,...,an)?a *= x(a0,...,an)?P
% s |- a *= P  => s[u/v] |- v(a0,...,an)?a *= u(a0,...,an)?P

% s |- b_i *= P_i, a *= P  => s |- _(a0,...,an) := (b0,...,bn).a *= x(a0,...,an) := (P0,...,Pn).P
% y ~ x, s |- b_i *= P_i, a *= P  => s |- y(a0,...,an) := (b0,...,bn).a *= x(a0,...,an) := (P0,...,Pn).P
% s |- b_i *= P_i, a *= P  => s[u/v] |- v(a0,...,an) := (b0,...,bn).a *= u(a0,...,an) := (P0,...,Pn).P


% s |- a *= P  => s |- _.a *= u?(a0,...,an).P
% s |- a *= P  => s |- _.a *= u!(a0,...,an) := (P0,...,Pn).P

% s |- a *= P  => s[<<u?(a0,...,an)>>/v] |- v.a *= u?(a0,...,an).P
% s |- a *= P  => s[<<u!(a0,...,an) := (P0,...,Pn) >>/v] |- v.a *= u!(a0,...,an) := (P0,...,Pn).P

% s |- a *= P  => s |- _?(a0,...,an).a *= u?(a0,...,an).P

% y ~ x, s |- a *= P  => s |- y?(a0,...,an).a *= x?(a0,...,an).P
% s |- a *= P  => s[u/v] |- v?(a0,...,an).a *= u?(a0,...,an).P


% s |- b_i *= P_i, a *= P  => s |- _!(a0,...,an) := (b0,...,bn).a *= x!(a0,...,an) := (P0,...,Pn).P
% y ~ x, s |- b_i *= P_i, a *= P  => s |- y!(a0,...,an) := (b0,...,bn).a *= x!(a0,...,an) := (P0,...,Pn).P
% s |- b_i *= P_i, a *= P  => s[u/v] |- v!(a0,...,an) := (b0,...,bn).a *= u!(a0,...,an) := (P0,...,Pn).P


% s |- a *= P => s |- @<<a>> *= @<<P>>

% s |- a_i *= P_i => s |- { a0;...;an } *= { P0;...;Pn }

In some sense pattern-matching is the workhorse of the language. It is
the means by which the program structure as data structure gets
realized so that computation can be carried out. The form of the rules
below indicates what sort of substitution a match results in. More
precisely, the statement $ s \vdash a \; \assign P $ means that the
subsitution, $s$, is the result of matching $a$ against
$P$. Alternatively, we say $s$ supports the match $a \; \assign P$.

\begin{mathpar}
  \inferrule* [Right=Wildcard] {} { [] \vdash \_ \; \assign P } \\
  \inferrule* [Right=Variable] {} { [\langle\langle P \rangle\rangle / v] \vdash v \; \assign P } \\\\
  \inferrule* [Right=WildInpSubject] { s \vdash a \; \assign P } 
  { s \vdash \textsf{\_}\textsf{?}\textsf{(}a_0 \textsf{,}\ldots\textsf{,}a_n\textsf{)}\textsf{.}a \; \assign x\textsf{?}\textsf{(}a_0\textsf{,}\ldots\textsf{,}a_n\textsf{)}\textsf{.}P } \\\\
  \inferrule* [Right=CompatInpSubject] { y \sim x \\ s \vdash a \; \assign P } { s \vdash y\textsf{?}\textsf{(}a_0\textsf{,}\ldots\textsf{,}a_n\textsf{)}\textsf{.}a \; \assign x\textsf{?}\textsf{(}a_0\textsf{,}\ldots\textsf{,}a_n\textsf{)}\textsf{.}P } \\\\
  \inferrule* [Right=VarInpSubject] { s \vdash a \; \assign P } { s[u/v] \vdash v\textsf{?}\textsf{(}a_0\textsf{,}\ldots\textsf{,}a_n\textsf{)}\textsf{.}a \; \assign u\textsf{?}\textsf{(}a_0\textsf{,}\ldots\textsf{,}a_n\textsf{)}\textsf{.}P } \\\\
  \inferrule* [Right=WildInpAction] { s \vdash a \; \assign P } { s \vdash \textsf{\_}\textsf{?}a \; \assign u\textsf{?}\textsf{(}a_0\textsf{,}\ldots\textsf{,}a_n\textsf{)} \textsf{.}P } \\\\
  \inferrule* [Right=InpActionCode] { s \vdash a \; \assign P }
  { s[\langle\langle u\textsf{?}\textsf{(}a_0\textsf{,}\ldots\textsf{,}a_n\textsf{)}/v\rangle\rangle] \vdash v\textsf{.}a \; \assign u\textsf{?}\textsf{(}a_0\textsf{,}\ldots\textsf{,}a_n\textsf{)} \textsf{.}P }
\end{mathpar}
  
\begin{mathpar}
  \inferrule* [Right=WildOutpSubject] { s \vdash b_i \; \assign P_i \\ s \vdash a \; \assign P } 
  { s \vdash \textsf{\_}\textsf{!}\textsf{(}a_0\textsf{,}\ldots\textsf{,}a_n\textsf{)} \textsf{:=} \textsf{(}b_0\textsf{,}\ldots\textsf{,}b_n\textsf{)}\textsf{.}a \; \assign x\textsf{!}\textsf{(}a_0\textsf{,}\ldots\textsf{,}a_n\textsf{)} \textsf{:=} \textsf{(}P_0\textsf{,}\ldots\textsf{,}P_n\textsf{)}\textsf{.}P } \\\\
  \inferrule* [Right=CompatOutpSubject] { y \sim x \\ s \vdash b_i \; \assign P_i \\ s \vdash a \; \assign P } 
  { s \vdash y\textsf{!}\textsf{(}a_0\textsf{,}\ldots\textsf{,}a_n\textsf{)} \textsf{:=} \textsf{(}b_0\textsf{,}\ldots\textsf{,}b_n\textsf{)}\textsf{.}a \; \assign x\textsf{!}\textsf{(}a_0\textsf{,}\ldots\textsf{,}a_n\textsf{)} \textsf{:=} \textsf{(}P_0\textsf{,}\ldots\textsf{,}P_n\textsf{)}\textsf{.}P } \\\\
  \inferrule* [Right=VarOutpSubject] { s \vdash b_i \; \assign P_i \\ s \vdash a \; \assign P }
  { s[u/v] \vdash v\textsf{!}\textsf{(}a_0\textsf{,}\ldots\textsf{,}a_n\textsf{)} \textsf{:=} \textsf{(}b_0\textsf{,}\ldots\textsf{,}b_n\textsf{)}\textsf{.}a \; \assign u\textsf{!}\textsf{(}a_0\textsf{,}\ldots\textsf{,}a_n\textsf{)} \textsf{:=} \textsf{(}P_0\textsf{,}\ldots\textsf{,}P_n\textsf{)}\textsf{.}P } \\\\
  \inferrule* [Right=WildOutpAction] { s \vdash a \; \assign P } { s \vdash \textsf{\_}\textsf{!}a \; \assign u\textsf{!}\textsf{(}a_0\textsf{,}\ldots\textsf{,}a_n\textsf{)} \textsf{.}P } \\\\
  \inferrule* [Right=OutpActionCode] { s \vdash a \; \assign P }
  { s[\langle\langle u\textsf{!}\textsf{(}a_0\textsf{,}\ldots\textsf{,}a_n\textsf{)} \textsf{:=} \textsf{(}P_0\textsf{,}\ldots\textsf{,}P_n\textsf{)}/v\rangle\rangle] \vdash v.a \; \assign u\textsf{!}\textsf{(}a_0\textsf{,}\ldots\textsf{,}a_n\textsf{)} \textsf{:=} \textsf{(}P_0\textsf{,}\ldots\textsf{,}P_n\textsf{)}\textsf{.}P }
\end{mathpar}

\begin{mathpar}       
  \inferrule* [Right=Code] { s \vdash a \; \assign P }
  { s \vdash \textsf{@}\langle\langle a \rangle\rangle \; \assign \textsf{@}\langle\langle P \rangle\rangle } \\\\
  \inferrule* [Right=Par] { s \vdash a_i \; \assign P_i }
  { s \vdash \textsf{\{} a_0 \textsf{;} \ldots \textsf{;} a_n \textsf{\}} \; \assign \textsf{\{} P_0 \textsf{;} \ldots \textsf{;} P_n \textsf{\}} } \\\\
  \inferrule* [Right=Rec] { s \vdash a \; \assign P \\ \mathcal{FV}(a) = \{ v \} }
  { s[\langle\langle \textsf{rec} \; v \textsf{.} a \rangle\rangle / v] \vdash \textsf{rec} \; v \textsf{.} a \; \assign  P }
\end{mathpar}

\subsection{Semantic substitution}

% {}s = {}
% { P; Q }s = { Ps; Qs }
% select{ B0; B1 }s = select{ (B0)s; (B1)s }

% (u(a0,...,an)?P)s = s(u)(s(a0),...,s(an)).(Ps) // must avoid capturing variables in a_i
% (u(a0,...,an) := (P0,...,Pn).P)s = s(u)(s(a0),...,s(an)) := ((P0)s,...,(Pn)s).(Ps) // must avoid capturing variables in a_i
% @v[w/v] = @w
% @v[<<P>>/v] = P

% s[u/v](v) = u
% s[u/v](v') = v'

\begin{eqnarray}
  \lefteqn{ \textsf{\{} \textsf{\}}s \triangleq } \\
  & & \mbox{} \textsf{\{} \textsf{\}} \\
  \lefteqn{ \textsf{\{} P \textsf{;} Q \textsf{\}}s \triangleq } \\
  & & \mbox{} \textsf{\{} Ps \textsf{;} Qs \textsf{\}} \\
  \lefteqn{ \textsf{select} \textsf{\{} B_0 \textsf{;} B_1 \textsf{\}}s \triangleq } \\
  & & \mbox{} \textsf{select} \textsf{\{} B_{0}s \textsf{;} B_{1}s \textsf{\}} \\
  \lefteqn{ (u \textsf{?} \textsf{(} a_0 \textsf{,} \ldots \textsf{,} a_n \textsf{)} \textsf{.} P)s \triangleq } \\
  & & \mbox{} s(u) \textsf{?} \textsf{(} s(a_0) \textsf{,} \ldots \textsf{,} s(a_n) \textsf{)} \textsf{.} (Ps) \\
  \lefteqn{ (u \textsf{!} \textsf{(} a_0 \textsf{,} \ldots \textsf{,} a_n \textsf{)} \textsf{:=} \textsf{(} P_0 \textsf{,} \ldots \textsf{,} P_n \textsf{)} \textsf{.} P)s \triangleq } \\
  & & s(u) \textsf{!} \textsf{(} s(a_0) \textsf{,} \ldots \textsf{,} s(a_n) \textsf{)} \textsf{:=} \textsf{(} s(P_0) \textsf{,} \ldots \textsf{,} s(P_n) \textsf{)} \textsf{.} (Ps) \\
  \lefteqn{ \textsf{@}v[w/v] \triangleq } \\
  & & \mbox{} \textsf{@}w \\
  \lefteqn{ \textsf{@}[\langle\langle P \rangle\rangle/v] \triangleq } \\
  & & \mbox{} P \\
  \lefteqn{ s[u/v](v) \triangleq } \\
  & & \mbox{} u \\
  \lefteqn{ s[u/v](v') \triangleq } \\
  & & \mbox{} v'
\end{eqnarray}

\section{Syntactic sugar}

% u(a0,...,an)? \triangleeq u(a0,...,an)?{}
% u(a0,...,an) := (P0,...,Pn). \triangleeq u(a0,...,an) := (P0,...,Pn).{}
% u(a0,...,an)?P \triangleeq select{ u(a0,...,an)?P }
% u(a0,...,an) := (P0,...,Pn).P \triangleeq select{ u(a0,...,an) := (P0,...,Pn).P }
% u[P0,...,Pn].P \triangleeq u(a0,...,an) := (P0,...,Pn).P', with s |- a_i *= P_i, P = P's
% let (a0,...,an) := (P0,...,Pn) in P \triangleeq { u(_,...,_)?; u(a0,...,an) := (P0,...,Pn).P } with u fresh

\section{Builtin processes}

\subsection{Data types}

\subsubsection{Numeric}

Numbers and numerals -- a number is a process, a numeral is its
name. For example, 1 is a process, <<1>> is the numeral denoting 1.

\paragraph{Integer}

% integer ::= ...

\paragraph{Rational}

\paragraph{Real}

\paragraph{Boolean}

\subsubsection{Char}

\subsection{Collections}

\subsubsection{String}

\subsubsection{Combinators}

Work over all data types

Add(+)
Subtract(-)
Multiply(*)
Divide(/)

\section{Typing}

\subsection{Judgments}

% A judgment will be of the form

% v0 : T0, ..., vM : TM |- e : T

% with e a process, a name or a pattern

\subsubsection{Notation}

% In what follows below we adopt

% T -- ranges over ProcType
% N -- ranges over NameType
% A -- ranges over ActionType
% M -- ranges over PatternType

% e -- ranges over process, name and pattern (expressions)
% u -- ranges over variables and names
% v+ -- ranges over variables and _

% Thus, the statement

% G |- e : T

% means that the assumptions G establish that the expression e is of type T
% and T is a ProcType

% while the statement

% G |- u : N

% means that the assumptions G establish that some variable or name, u, is of type
% N and N is a NameType.

\paragraph{Aim}

This system is designed primarily to establish that a well-typed process never
puts a variable typed ProcType into a position v.P.

\subsection{Types}

\subsubsection{Process Types}

% ProcType  ::= 0
%               BranchType0 +...+ BranchTypeN
%               @NameType
%               ProcType0 | ... | ProcTypeN

% BranchType ::= ActionType.ProcType

% ActionType ::= NameType?(PatternType0,...,PatternTypeN)
%                NameType!(PatternType0,...,PatternTypeN)
%                   := (ProcType0,...,ProcTypeN)


% CodeType ::= NameType
%              <<ActionType>>

% NameType ::= <<ProcType>>

% PatternType ::= CodeType
%                 NameType?(PatternType0,...,PatternTypeN).PatternType
%                 NameType!(PatternType0,...,PatternTypeN)
%                    := (PatternType0,...,PatternTypeN).PatternType
%                 @<<PatternType>>
%                 PatternType0|...|PatternTypeN

\subsection{Typing}

% |- {} : 0, |- select{} : 0


%           G |- e_i : T_i
% ------------------------------------

%  G |- { e0;...;eN } : T0 | ... | TN

%            G |- e_i : T_i
% -------------------------------------------

%  G |- select{ e0;...;eN } : T0 + ... + TN

%      G |- e : T
% ----------------------

%   G |- <<e>> : <<T>>

%    G |- e : N
% ------------------
%   G |- @e : @N

%    G |- u : N, G |- P : T, G |- a_i : M_i
% ----------------------------------------------
%    G |- u?(a0,...,an).P : N?(M0,...,Mn).T

%        G |- u : N, G |- P : T, G |- a_i : M_i, G |- P_i : T_i
% ----------------------------------------------------------------------
%  G |- u!(a0,...,an) := (P0,...,Pn).P : N!(M0,...,Mn) := (T0,...,Tn).T

%   G |- v : <<A>>, G |- P : T
% ------------------------------
%       G |- @v.P : A.T

% v+ : N |- v+ : N, v+ : A |- v+ : A

%   G |- v+ : N, G |- a_i : M_i, G |- a : M
% --------------------------------------------
%   G |- v+?(a0,...,an).a : N?(M0,...,Mn).M

%         G |- v+ : N, G |- a_i : M_i, G |- b_i : M'_i, G |- a : M
% ---------------------------------------------------------------------------
%   G |- v+!(a0,...,an) := (b0,...,bn).a : N?(M0,...,Mn) := (M'0,...,M'n).M

%      G |- a : M
% ----------------------
%  G |- @<<a>> : @<<M>>

%         G |- a_i : M_i
% --------------------------------
%  G |- { a0;...;an } : M0|...|Mn