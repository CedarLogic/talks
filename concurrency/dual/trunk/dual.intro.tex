\section{Introduction}\label{sec:introduction} % (fold)

At the heart of this paper is an attempt to tease out the content of
the intuition that scale and composition are somehow related. Any
engineer will tell you, you build ``bigger'' components out of
``smaller'' ones. The question is how this notion of bigger, in the
sense of system complexity, is related to bigger in the sense of
physical scale. To tell this story properly, it needs to be told in
installments because the technical apparatus, while not particularly
large or unwieldy, is sufficiently complex that it is impossible to
unpack it properly and simultaneously provide convincingly detailed
examples. Part of the complexity is the build up of an enormous
backlog of conceptual context informing operational
knowledge. Therefore, the primary aim of this paper is to lay out the
necessary technical apparatus together with a more explicit account of
its context to provide a base rationale for a subsequent detailed
treatment of examples of processes involving interaction across scale
(which, in its turn provides the justification for the development of
the technical apparatus).

What is at stake, what makes this sort of investigation worth the
effort, in absence of concrete, detailed examples, is the dream of a
compositional account of physical dynamics. To date, the most
successful mathematical apparatus for specifying and reasoning about
physical dynamical systems (as opposed to symbolic dynamical systems)
is Newton's calculus. Both as a body of mathematics and as a language
in which to express physical theories it is impressive, and literally
unrivalled -- in the sense that there is no serious competitor in
which to express, reason about, simulate or otherwise calculate
dynamical aspects of physical systems, despite what we have learned
about computational dynamices over the last 40 years. And, both the
predominant application and the apparatus itself \emph{fails} to be
compositional. 

To bring this into sharp relief imagine the standard specification,
via ODEs, of the dynamics of some physical system, for example the
dynamics of some gene regulatory network, like the p53-Mdm2 feedback
cycle in the cell \cite{Lahav2004}. Now imagine the standard
specification, via ODEs, of the dynamics of some ``nearby'' physical
system, for example another gene regulatory network, or a cytoplasmic
process in the same cell (type). It is not possible to compose the two
models (as proxies for the physical subsystems they represent) to get
some understanding of the global system behavior. Even in the cases
where a more global account of the dynamics is possible, the systems
of ODEs has to be reworked from the ground up, not built from the
component system of equations. More often than not, however, for
example in the case of cytoplasmic processes interacting with
processes in the nucleus, issues of spatio-temporal scale represent
significant obstacles in obtaining more global accounts of dynamics
from more local ones. Not to put too fine a point on it, if we widen
the scope a bit, though Newton's wonderful machinery is used in
reasoning about chemical processes, and in reasoning about nuclear
ones, there is a great divide between these processes from the
perspective of modeling their dynamics within Newton's calculus. Thus,
while the Gecko has made a survival strategy, at least in part,
exploiting VanDerWaal's forces \cite{Gecko}, we are still a long way
from a similarly practical and \emph{reductive} account of physical
dynamics.

This facet of Newton's calculus needs to be contrasted with
compositional accounts of symbolic dynamics, as may be seen in systems
like the $\lambda$-calculus, or the $\pi$-calculus where the way to
get at global system behavior -- the primary means, in fact -- is
through composition of the dynamics of the components. Like
``experiments'' in ``laboratory'', one mixes components, via
application, in the case of $\lambda$, or parallel composition, in the
case of $\pi$, to find the behavior of the composite. Like Newton's
framework, but unlike purely empirical methods, these systems are
genuine mathematical theories equipped with powerful tools for
analysis, such as bisimulation
\cite{DBLP:conf/lics/Sangiorgi04}. Moreover, these systems, via a
precise, algebraic treatment of \emph{context}, blithely dispatch huge
disparities in scale -- in the sense of system complexity -- as we
will make clear in the sequel. For the computer scientist who is both
trained in these techniques and concerned with the issues facing the
modeling of physical systems it is natural to ask whether there might
be a way to export or exploit these methods of addressing symbolic
dynamics to address physical dynamics. This question is made all the
more vivid in the light of the work by Cardelli, Priami, Regev, and
others who have used such compositional systems (like the
$\pi$-calculus and the ambient calculus) to give direct, quantitative
and predictive accounts of physical processes such as gene regulatory
networks, as well as signaling and metabolic processes in the
cell \cite{Cardelli04} \cite{DanosL03} \cite{RegevS03}
\cite{PriamiRSS01}.

\subsection{Summary of contributions and overview}

We propose to lay out a conceptual framework, realizable in a wide
variety of mobile process calculi, in which to relate interaction and
scale. To do this we will review some of the development of the
paradigmatic mobile process calculus, Milner's $\pi$-calculus,
summarizing the essential points of the theory and comparing and
contrasting these with standard accounts of physical dynamics. In the
process we will uncover some of the less well-known subtleties of
Milner's design choices in presenting the calculus. Careful analysis
of these choices exposes a set of principles and ontological
commitments that imply a precise characterization of the relationship
between interaction and scale. Following this analysis, we will
exhibit a series of mild extensions of Milner's original calculus,
resulting in a technical apparatus for expressing interaction across
scale. Unfortunately, to really make good on the claim we need to
illustrate it with example processes and these must be deferred to a
forthcoming paper.

\paragraph{Warning: the appearance of Newton-bashing}
The arguments that follow may appear to be heavily slanted in favor of
the computational account of dynamics over the Newtonian account. Part
of this is due to the author's clear affinity for computation and
concurrency theory. However, it is now 320 years since Philosophiae
Naturalis Principia Mathematica was published. The mathematical
apparatus and what it engendered dominates the field, without
question. Because this framework is so pervasive in modern scientific
and mathematical culture is not necessary to recount its approach to
dynamics or its successes, but it is necessary to stress the core
features of alternative view. This leads to a greater appearance of
bias than there actually is. 

% section introduction (end)