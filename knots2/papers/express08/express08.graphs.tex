\section{Graphs}

\begin{definition}
  A \emph{graph} consists of the following data $(V,E,\textsf{src},\textsf{trgt})$:
  \begin{enumerate}
    \item a set, $V$, of \emph{vertices} (alternately, \emph{nodes});
    \item a set, $E$, of \emph{edges} (alternately, \emph{arrows} or \emph{links});
    \item a map, $\textsf{src}:E \to V$, determining the \emph{source} of an edge;
    \item a map, $\textsf{trgt}:E \to V$, determining the \emph{target} of an edge;
  \end{enumerate}
\end{definition}

This definition is general enough to capture both directed and
undirected graphs. In particular, nothing in the definition dictates
that an edge is oriented from source to target. Rather, each edge,
$e$, has two endpoints, source and target, accessed by the functions,
$\textsf{src}$, and $\textsf{trgt}$, respectively. And, speaking of
accessing, we will avail ourselves of standard `accessor' function
notation. Thus, if $G = (V,E,\textsf{src},\textsf{trgt})$, then $V(G)
= V$ and $E(G) = E$, etc.

\begin{example} \label{seqex}
  \begin{enumerate}
  \item The \emph{discrete graph}, $G(n)$, is given by
  $(\{0,...,n-1\},\emptyset,\bot,\bot)$ where $\bot$ is the function
  with empty domain.

  \item A \emph{chain}, $C(n)$, is given by 
  \begin{enumerate}
    \item $V(C(n)) = \{0,...,n-1\}$;
    \item $E(C(n)) = \{e_0,...,e_{n-2}\}$;
    \item $\textsf{src}(e_i)=e_{i-1}, 0 \leq i \leq n-2$;
    \item $\textsf{trgt}(e_i)=e_{i+1},0 \leq i \leq n-2$.
  \end{enumerate}

  \item A \emph{Loop}, $L(n)$, is constructed from a chain,
  $C(n)$, by adding an edge, $e_{n-1}$ with
  $\textsf{src}(e_{n-1})=e_{n-2}$ and $\textsf{trgt}(e_{n-1})=e_{0}$.
  \end{enumerate}
\end{example}

\begin{example} \label{knotex}
  A \emph{knot} is an embedding, $K : S^0 \to \Real^3$, of the circle
  into 3-space. A \emph{knot diagram} is a projection of a knot onto
  the plane such that every crossing is of the form .... A \emph{knot
    shadow} is a graph derived from a knot diagram by associating
  vertices to crossings and edges between connected crossings. Note
  that a knot shadow is always a tetravalent graph, i.e. for any
  vertex, $v$, the cardinality of the cone of $v$ is $|\textsf{e}(v)|
  = 4$.
\end{example}

\begin{example} \label{chemex}
  \begin{eqnarray}
    E + S \rightleftharpoons ES \rightarrow E + P
  \end{eqnarray}
\end{example}

\begin{definition}
  A \emph{subgraph}, $H$, of a graph, $G$, is a graph such that
  \begin{enumerate}
     \item $V(H) \subseteq V(G)$;
     \item $E(H) \subseteq E(G)$;
     \item $\textsf{src}(H) = \textsf{src}(G) \backslash E(H)$;
     \item $\textsf{trgt}(H) = \textsf{trgt}(G) \backslash E(H)$;
  \end{enumerate}
  More generally, given graphs, $G, H$, a graph \emph{homomorphism},
  $m: G \to H$, from $G$ to $H$ is a pair of functions, $m_V : V(G)
  \to V(H)$, $m_E : E(G) \to E(H)$ such that $\textsf{src}(m_E(e)) =
  m_V(\textsf{src}(e))$ and $\textsf{trgt}(m_E(e)) =
  m_V(\textsf{trgt}(e))$. And a graph \emph{isomorphism} is a graph
  homomorphism in which the vertex and edge maps are bijections.
\end{definition}

As usual, we overload $\subseteq$ for the subgraph relation. Thus,
$G(n) \subseteq C(n) \subseteq L(n)$ holds for the examples above. We
write $G \sim H$ when there exists a graph isomorphism between $G$ and
$H$.

\begin{definition}
  A \emph{path}, $p$, in graph, $G$, is an ordered collection of
  edges, $p = (e_0, ..., e_N)$, such that $\textsf{src}(e_i) =
  \textsf{trgt}(e_{i-1})$. We use the notation $p[i]$ to access the
  $i$th edge in the path, $p$, and $|p|$ to denote the length of the
  path. Like an edge, a path, $p$, has a source, $\textsf{src}(p)$,
  and a target, $\textsf{trgt}(p)$, calculated as $\textsf{src}(p) =
  \textsf{src}(p[0])$, and $\textsf{trgt}(p) =
  \textsf{trgt}(p[|p|-1])$.
\end{definition}

\begin{definition}
  A \emph{connected component}, $C$, of a graph, $G$, is a subgraph of
  $G$ such that there is a path between every distinct pair of
  vertices. That is, for all $v_0, v_1$ such that $v_0 \neq v_1$ there
  exists a $p$ in $C$ such that either $\textsf{src}(p) = v_0$ and
  $\textsf{trgt}(p) = v_1$ or $\textsf{src}(p) = v_1$ and
  $\textsf{trgt}(p) = v_0$.
\end{definition}

\begin{remark}
  Note that while chains and loops have a single connected component
  the discrete graph $G(n)$ enjoys $n$ connected components.
\end{remark}

\begin{definition}
  The \emph{slice} of a vertex, $v$, in a graph, $G$, is given by
  \begin{equation}
    \textsf{slice}_{G}(v) \triangleq \{ e \; | \; \textsf{src}(e) = v \}
  \end{equation}
  and the \emph{co-slice} is given by
  \begin{equation}
    \textsf{co-slice}_{G}(v) \triangleq \{ e \; | \; \textsf{trgt}(e) = v \}
  \end{equation}
  and the \emph{cone} is given by $\textsf{e}(v) \triangleq \textsf{slice}_{G}(v) \cup \textsf{co-slice}_{G}(v)$.
\end{definition}

As usual the subscript, $G$, will be dropped when context minimizes
the risk of confusion.


