\subsection{Reflective rewrite systems}\label{sub:the_syntax_and_semantics_of_the_notation_system} % (fold)

% [Ed. Note: stylistic alternative 1:]
% Having established an intuitive account of the target calculus via the
% encoding, we summarize its technical presentation below. These calculi
% are typically presented by giving a set of generators and relations on
% them. The grammar, below, describing term constructors, freely
% generates the set of processes $\mathcal{P}$ under the operations
% indicated.

% [Ed. Note: stylistic alternative 2:]

\subsubsection{Ground terms and reflections}

\begin{definition}[ground terms]
  Given a signature $\Sigma = (\Sigma_{i})_{i \in \mathcal{N}}$ with
  $\Sigma_i \cap \Sigma_j = \emptyset$ when $i \neq j$, we define
  $\mathcal{L}(\Sigma)$, the \emph{ground terms} of $\Sigma$, by the
  following rules

  \begin{mathpar}
    \inferrule* [lab=constants] {s \in \Sigma_{0}} {s \in \mathcal{L}(\Sigma)}
    \and \\
    \inferrule* [lab=ground terms] {s_j \in \mathcal{L}(\Sigma) \\ 0 \leq j < i \\ s \in \Sigma_{i}} {s(s_{0},...,s_{i-1}) \in \mathcal{L}(\Sigma)}
  \end{mathpar}

  If $s \in \Sigma_i$ we use $|s| \triangleq i$ to denote the
  \emph{arity} of $s$.
\end{definition}

\begin{remark}
  For purposes of this note we ignore issues of polymorphism and
  associativity and commutativity, etc. The $\mathcal{L}(\Sigma)$ is
  describing the set of terms before any such relations are
  imposed. That is, $\mathcal{L}(\Sigma)$ is the free construction
  over some signature, $\Sigma$.
\end{remark}

\begin{remark}
  As usual, we abuse notation, appearing to conflate $\Sigma$ with
  $\bigcup_i\Sigma_i$ and write expressions such as $s \in \Sigma$.
\end{remark}

Next, we define $\mathcal{L}_{Q}(\Sigma)$, the \emph{reflection} of $\Sigma$, by 

\begin{definition}[terms and codes]
  \begin{mathpar}
    \inferrule* [lab=constants] {s \in \Sigma_{0}} {s \in \mathcal{L}_{Q}(\Sigma)}
    \and \\
    \inferrule* [lab=terms] {((s_j \in \mathcal{L}_{Q}(\Sigma)) or (s_j = \quotep{t}, t \in \mathcal{L}_{Q}(\Sigma))) \\ 0 \leq j < i \\ s \in \Sigma_{i}} {s(s_{0},...,s_{i-1}) \in \mathcal{L}_{Q}(\Sigma)}
    \and \\
    \inferrule* [lab=dequoted terms] { s = \quotep{t}, t \in \mathcal{L}_{Q}(\Sigma)} { \dropn{s} \in \mathcal{L}_{Q}(\Sigma)}
  \end{mathpar}

  and impose the relation $\quotep{\dropn{s}} \equiv s$.

  We call $\quotep{s}$ the \emph{code} of $s$ and set
  $\quotep{\mathcal{L}_{Q}(\Sigma)} \triangleq \{ \quotep{s} \; | \; s
  \in \mathcal{L}_{Q}(\Sigma) \}$, to denote the set of
  \emph{codes}. Similarly, we call $\dropn{\quotep{t}}$ a decoding of
  $\quotep{t}$.

\end{definition}

\begin{remark}
  We do \emph{not} require $\dropn{\quotep{s}} \equiv s$.
\end{remark}
\begin{remark}
  Intuitively, the set of codes corresponds to a set of names or
  variables. The set of terms does not intersect the set of codes,
  i.e. $\mathcal{L}_{Q}(\Sigma) \cap \quotep{\mathcal{L}_{Q}(\Sigma)}
  = \emptyset$. Further, if we let $\mathcal{T}(s)$ denote the evident
  syntax tree of a term, then nodes of $\mathcal{T}(s)$ labeled with
  $\quotep{-}$ cannot have children labeled with $\quotep{-}$; rather,
  there must be an intervening term constructor.

  Said, if we define a labeling of the tree with $O$ for nodes labled
  with term constructors and $P$ we see that a node labeled $P$ must
  always have a child labeled $O$. Moreover, terms correspond to trees
  with roots labeled $O$ and codes correspond to trees with roots
  labeled $P$. This alternation discipline is why we cannot achieve
  the construction merely by enriching $\Sigma_{1}$ with another
  operator.
\end{remark}

Following this intuition of codes as names or variables we define the
set of \emph{free codes} of a term.

\begin{definition}[free codes]
  \begin{mathpar}
    \inferrule* [lab=constants] {|s| = 0}{\mathcal{FC}(s) \triangleq \emptyset}
    \and \\
    \inferrule* [lab=terms] {}{\mathcal{FC}(s(t_{0},...,t_{i-1})) \triangleq \{ t_j \; | \; t_j = \quotep{u}\} \cup \bigcup_{\{ t_k \; | \; t_k \neq \quotep{u}\}} \mathcal{FC}(t_k)}
    \and \\
    \inferrule* [lab=decodes]{s = \dropn{\quotep{t}}}{\mathcal{FC}(s) \triangleq \{ \quotep{t} \}}
  \end{mathpar}
\end{definition}

\begin{theorem}
  $\quotep{s} \not\in \mathcal{FC}(s)$
\end{theorem}
\begin{proof}
  By structural induction.
\end{proof}

\begin{definition}[open term constructors]
  Let $C$ be a set of codes. Suppose $\exists k \in \Sigma$ such that
  $0 < |k| < |C|$. We can form the set of terms $\mathcal{O}(C) = \{
  k(t_0,...,t_{|k|-1}) \; | \; t_i = k(c_0,...,c_{|k|-1}), \bigcup_i
  \mathcal{FC}(t_i) = C\}$.
\end{definition}

\begin{theorem}
  Let $s$ be a term. $\forall t \in \mathcal{O}(\mathcal{FC}(s)). \quotep{t} \not\in \mathcal{FC}(s)$.
\end{theorem}
\begin{proof}
  By structural induction.
\end{proof}

\begin{remark}
  As the free codes correspond to the notion of free names or free
  variables of a term, these two theorems give us a supply of `fresh'
  codes analogous to fresh names or variables.
\end{remark}

\subsubsection{Substitution}

\begin{definition}[renaming -- the noun]
  A \emph{renaming}, $\sigma$, is a partial map, $\sigma :
  \quotep{\mathcal{L}_{Q}(\Sigma)} \to
  \quotep{\mathcal{L}_{Q}(\Sigma)}_{\bot}$, from codes to codes.
\end{definition}

\begin{definition}[substitution -- the verb]
  We define, $s \cdot \sigma$, the application of a renaming to a
  term, via recursive traversal of the term structure.

  \begin{mathpar}
    \inferrule* [lab=constants]{|s| = 0}{s \cdot \sigma \triangleq s}
    \and \\
    \inferrule* [lab=terms]{}{s(t_{0},...,t_{i-1}) \cdot \sigma \triangleq s(t_{0}\cdot \sigma,...,t_{i-1}\cdot \sigma)}
    \and \\
    \inferrule* [lab=codes]{\sigma(\quotep{t}) = \bot}{\quotep{t} \cdot \sigma \triangleq \quotep{t}}
    \and
    \inferrule* [lab=codes]{\sigma(\quotep{t}) = s}{\quotep{t} \cdot \sigma \triangleq s}
    \and
    \inferrule* [lab=decodes]{\sigma(\quotep{t}) = \bot}{\dropn{\quotep{t}} \cdot \sigma \triangleq \dropn{\quotep{t}}}
    \and
    \inferrule* [lab=decodes]{\sigma(\quotep{t}) = \quotep{u}}{\dropn{\quotep{t}} \cdot \sigma \triangleq u}
  \end{mathpar}
\end{definition}

\begin{remark}
  The application of a renaming does more than shuffle names: when it
  encounters a decoding it substitutes in a term. 
\end{remark}

\subsection{Basic rewriting}

\begin{definition}
  Let $\rightarrow \subset \mathcal{L}_{Q}(\Sigma) \times
  \mathcal{L}_{Q}(\Sigma)$. We say a term $s$ rewrites to a term $t$
  ($s \rightarrow t$) when $\exists (l,r) \in \rightarrow,
  \sigma. l \cdot \sigma = s, r \cdot \sigma = t$. We also use $s
  \stackrel{(l,\sigma,r)}{\rightarrow} t$ to label the rewrite with
  the witness.
\end{definition}

\subsection{Binders}

\begin{definition}[binders]
  Given a signature, $\Sigma$, we define
  $\mathcal{L}_{Q}^{\lambda}(\Sigma)$ by the following rules

  \begin{mathpar}
    \inferrule* [lab=constants] {s \in \Sigma_{0}} {s \in \mathcal{L}_{Q}^{\lambda}(\Sigma)}
    \and \\
    \inferrule* [lab=terms] {((s_j \in \mathcal{L}_{Q}^{\lambda}(\Sigma)) or (s_j = \quotep{t}, t \in \mathcal{L}_{Q}^{\lambda}(\Sigma))) \\ 0 \leq j < i \\ s \in \Sigma_{i}} {s(s_{0},...,s_{i-1}) \in \mathcal{L}_{Q}^{\lambda}(\Sigma)}
    \and \\
    \inferrule* [lab=dequoted terms] { s = \quotep{t}, t \in \mathcal{L}_{Q}^{\lambda}(\Sigma)} { \dropn{s} \in \mathcal{L}_{Q}^{\lambda}(\Sigma)}
    \and \\
    \inferrule* [lab=abstractions] { s = \quotep{s'}, s' \in \mathcal{L}_{Q}^{\lambda}(\Sigma) \\ t \in \mathcal{L}_{Q}^{\lambda}(\Sigma)} {\lambda s.t \in \mathcal{L}_{Q}^{\lambda}(\Sigma)}
  \end{mathpar}
\end{definition}

\begin{definition}[free codes]
  We extend the previous definition with the following rule.
  \begin{mathpar}
    \inferrule* [lab=abstractions] {} {\mathcal{FC}(\lambda s.t) \triangleq \mathcal{FC}(t) \setminus \{ s \}}
  \end{mathpar}
\end{definition}

\begin{definition}[beta-reduction]
  $(\lambda s.t) u \to_{\beta} t \{ \quotep{u} / s \}$
\end{definition}

\begin{definition}[closed terms]
  A term, $t \in \mathcal{L}_{Q}^{\lambda}(\Sigma)$, is closed
  $\Leftrightarrow \mathcal{FC}(t) = \emptyset$. We denote the set of
  closed terms with $\overline{\mathcal{L}_{Q}^{\lambda}(\Sigma)}$.
\end{definition}

\subsection{Duality}

\begin{definition}[dual system]
  $\overline{\mathcal{L}_{Q}^{\lambda}(\Sigma)}^{\perp} \subset [\overline{\mathcal{L}_{Q}^{\lambda}(\Sigma)} \to \quotep{\overline{\mathcal{L}_{Q}^{\lambda}(\Sigma)}}]$ is defined by
  $t^{\perp}(u) \triangleq \quotep{(t u)}$
\end{definition}

% subsection the_syntax_and_semantics_of_the_notation_system (end)   